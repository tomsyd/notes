% Generated by Sphinx.
\def\sphinxdocclass{report}
\documentclass[letterpaper,10pt,english]{sphinxmanual}
\usepackage[utf8]{inputenc}
\DeclareUnicodeCharacter{00A0}{\nobreakspace}
\usepackage{cmap}
\usepackage[T1]{fontenc}
\usepackage{babel}
\usepackage{times}
\usepackage[Bjarne]{fncychap}
\usepackage{longtable}
\usepackage{sphinx}
\usepackage{multirow}

\addto\captionsenglish{\renewcommand{\figurename}{Fig. }}
\addto\captionsenglish{\renewcommand{\tablename}{Table }}
\floatname{literal-block}{Listing }



\title{technotes Documentation}
\date{April 02, 2015}
\release{1}
\author{Tom JIANG}
\newcommand{\sphinxlogo}{}
\renewcommand{\releasename}{Release}
\makeindex

\makeatletter
\def\PYG@reset{\let\PYG@it=\relax \let\PYG@bf=\relax%
    \let\PYG@ul=\relax \let\PYG@tc=\relax%
    \let\PYG@bc=\relax \let\PYG@ff=\relax}
\def\PYG@tok#1{\csname PYG@tok@#1\endcsname}
\def\PYG@toks#1+{\ifx\relax#1\empty\else%
    \PYG@tok{#1}\expandafter\PYG@toks\fi}
\def\PYG@do#1{\PYG@bc{\PYG@tc{\PYG@ul{%
    \PYG@it{\PYG@bf{\PYG@ff{#1}}}}}}}
\def\PYG#1#2{\PYG@reset\PYG@toks#1+\relax+\PYG@do{#2}}

\expandafter\def\csname PYG@tok@gd\endcsname{\def\PYG@tc##1{\textcolor[rgb]{0.63,0.00,0.00}{##1}}}
\expandafter\def\csname PYG@tok@gu\endcsname{\let\PYG@bf=\textbf\def\PYG@tc##1{\textcolor[rgb]{0.50,0.00,0.50}{##1}}}
\expandafter\def\csname PYG@tok@gt\endcsname{\def\PYG@tc##1{\textcolor[rgb]{0.00,0.27,0.87}{##1}}}
\expandafter\def\csname PYG@tok@gs\endcsname{\let\PYG@bf=\textbf}
\expandafter\def\csname PYG@tok@gr\endcsname{\def\PYG@tc##1{\textcolor[rgb]{1.00,0.00,0.00}{##1}}}
\expandafter\def\csname PYG@tok@cm\endcsname{\let\PYG@it=\textit\def\PYG@tc##1{\textcolor[rgb]{0.25,0.50,0.56}{##1}}}
\expandafter\def\csname PYG@tok@vg\endcsname{\def\PYG@tc##1{\textcolor[rgb]{0.73,0.38,0.84}{##1}}}
\expandafter\def\csname PYG@tok@m\endcsname{\def\PYG@tc##1{\textcolor[rgb]{0.13,0.50,0.31}{##1}}}
\expandafter\def\csname PYG@tok@mh\endcsname{\def\PYG@tc##1{\textcolor[rgb]{0.13,0.50,0.31}{##1}}}
\expandafter\def\csname PYG@tok@cs\endcsname{\def\PYG@tc##1{\textcolor[rgb]{0.25,0.50,0.56}{##1}}\def\PYG@bc##1{\setlength{\fboxsep}{0pt}\colorbox[rgb]{1.00,0.94,0.94}{\strut ##1}}}
\expandafter\def\csname PYG@tok@ge\endcsname{\let\PYG@it=\textit}
\expandafter\def\csname PYG@tok@vc\endcsname{\def\PYG@tc##1{\textcolor[rgb]{0.73,0.38,0.84}{##1}}}
\expandafter\def\csname PYG@tok@il\endcsname{\def\PYG@tc##1{\textcolor[rgb]{0.13,0.50,0.31}{##1}}}
\expandafter\def\csname PYG@tok@go\endcsname{\def\PYG@tc##1{\textcolor[rgb]{0.20,0.20,0.20}{##1}}}
\expandafter\def\csname PYG@tok@cp\endcsname{\def\PYG@tc##1{\textcolor[rgb]{0.00,0.44,0.13}{##1}}}
\expandafter\def\csname PYG@tok@gi\endcsname{\def\PYG@tc##1{\textcolor[rgb]{0.00,0.63,0.00}{##1}}}
\expandafter\def\csname PYG@tok@gh\endcsname{\let\PYG@bf=\textbf\def\PYG@tc##1{\textcolor[rgb]{0.00,0.00,0.50}{##1}}}
\expandafter\def\csname PYG@tok@ni\endcsname{\let\PYG@bf=\textbf\def\PYG@tc##1{\textcolor[rgb]{0.84,0.33,0.22}{##1}}}
\expandafter\def\csname PYG@tok@nl\endcsname{\let\PYG@bf=\textbf\def\PYG@tc##1{\textcolor[rgb]{0.00,0.13,0.44}{##1}}}
\expandafter\def\csname PYG@tok@nn\endcsname{\let\PYG@bf=\textbf\def\PYG@tc##1{\textcolor[rgb]{0.05,0.52,0.71}{##1}}}
\expandafter\def\csname PYG@tok@no\endcsname{\def\PYG@tc##1{\textcolor[rgb]{0.38,0.68,0.84}{##1}}}
\expandafter\def\csname PYG@tok@na\endcsname{\def\PYG@tc##1{\textcolor[rgb]{0.25,0.44,0.63}{##1}}}
\expandafter\def\csname PYG@tok@nb\endcsname{\def\PYG@tc##1{\textcolor[rgb]{0.00,0.44,0.13}{##1}}}
\expandafter\def\csname PYG@tok@nc\endcsname{\let\PYG@bf=\textbf\def\PYG@tc##1{\textcolor[rgb]{0.05,0.52,0.71}{##1}}}
\expandafter\def\csname PYG@tok@nd\endcsname{\let\PYG@bf=\textbf\def\PYG@tc##1{\textcolor[rgb]{0.33,0.33,0.33}{##1}}}
\expandafter\def\csname PYG@tok@ne\endcsname{\def\PYG@tc##1{\textcolor[rgb]{0.00,0.44,0.13}{##1}}}
\expandafter\def\csname PYG@tok@nf\endcsname{\def\PYG@tc##1{\textcolor[rgb]{0.02,0.16,0.49}{##1}}}
\expandafter\def\csname PYG@tok@si\endcsname{\let\PYG@it=\textit\def\PYG@tc##1{\textcolor[rgb]{0.44,0.63,0.82}{##1}}}
\expandafter\def\csname PYG@tok@s2\endcsname{\def\PYG@tc##1{\textcolor[rgb]{0.25,0.44,0.63}{##1}}}
\expandafter\def\csname PYG@tok@vi\endcsname{\def\PYG@tc##1{\textcolor[rgb]{0.73,0.38,0.84}{##1}}}
\expandafter\def\csname PYG@tok@nt\endcsname{\let\PYG@bf=\textbf\def\PYG@tc##1{\textcolor[rgb]{0.02,0.16,0.45}{##1}}}
\expandafter\def\csname PYG@tok@nv\endcsname{\def\PYG@tc##1{\textcolor[rgb]{0.73,0.38,0.84}{##1}}}
\expandafter\def\csname PYG@tok@s1\endcsname{\def\PYG@tc##1{\textcolor[rgb]{0.25,0.44,0.63}{##1}}}
\expandafter\def\csname PYG@tok@gp\endcsname{\let\PYG@bf=\textbf\def\PYG@tc##1{\textcolor[rgb]{0.78,0.36,0.04}{##1}}}
\expandafter\def\csname PYG@tok@sh\endcsname{\def\PYG@tc##1{\textcolor[rgb]{0.25,0.44,0.63}{##1}}}
\expandafter\def\csname PYG@tok@ow\endcsname{\let\PYG@bf=\textbf\def\PYG@tc##1{\textcolor[rgb]{0.00,0.44,0.13}{##1}}}
\expandafter\def\csname PYG@tok@sx\endcsname{\def\PYG@tc##1{\textcolor[rgb]{0.78,0.36,0.04}{##1}}}
\expandafter\def\csname PYG@tok@bp\endcsname{\def\PYG@tc##1{\textcolor[rgb]{0.00,0.44,0.13}{##1}}}
\expandafter\def\csname PYG@tok@c1\endcsname{\let\PYG@it=\textit\def\PYG@tc##1{\textcolor[rgb]{0.25,0.50,0.56}{##1}}}
\expandafter\def\csname PYG@tok@kc\endcsname{\let\PYG@bf=\textbf\def\PYG@tc##1{\textcolor[rgb]{0.00,0.44,0.13}{##1}}}
\expandafter\def\csname PYG@tok@c\endcsname{\let\PYG@it=\textit\def\PYG@tc##1{\textcolor[rgb]{0.25,0.50,0.56}{##1}}}
\expandafter\def\csname PYG@tok@mf\endcsname{\def\PYG@tc##1{\textcolor[rgb]{0.13,0.50,0.31}{##1}}}
\expandafter\def\csname PYG@tok@err\endcsname{\def\PYG@bc##1{\setlength{\fboxsep}{0pt}\fcolorbox[rgb]{1.00,0.00,0.00}{1,1,1}{\strut ##1}}}
\expandafter\def\csname PYG@tok@mb\endcsname{\def\PYG@tc##1{\textcolor[rgb]{0.13,0.50,0.31}{##1}}}
\expandafter\def\csname PYG@tok@ss\endcsname{\def\PYG@tc##1{\textcolor[rgb]{0.32,0.47,0.09}{##1}}}
\expandafter\def\csname PYG@tok@sr\endcsname{\def\PYG@tc##1{\textcolor[rgb]{0.14,0.33,0.53}{##1}}}
\expandafter\def\csname PYG@tok@mo\endcsname{\def\PYG@tc##1{\textcolor[rgb]{0.13,0.50,0.31}{##1}}}
\expandafter\def\csname PYG@tok@kd\endcsname{\let\PYG@bf=\textbf\def\PYG@tc##1{\textcolor[rgb]{0.00,0.44,0.13}{##1}}}
\expandafter\def\csname PYG@tok@mi\endcsname{\def\PYG@tc##1{\textcolor[rgb]{0.13,0.50,0.31}{##1}}}
\expandafter\def\csname PYG@tok@kn\endcsname{\let\PYG@bf=\textbf\def\PYG@tc##1{\textcolor[rgb]{0.00,0.44,0.13}{##1}}}
\expandafter\def\csname PYG@tok@o\endcsname{\def\PYG@tc##1{\textcolor[rgb]{0.40,0.40,0.40}{##1}}}
\expandafter\def\csname PYG@tok@kr\endcsname{\let\PYG@bf=\textbf\def\PYG@tc##1{\textcolor[rgb]{0.00,0.44,0.13}{##1}}}
\expandafter\def\csname PYG@tok@s\endcsname{\def\PYG@tc##1{\textcolor[rgb]{0.25,0.44,0.63}{##1}}}
\expandafter\def\csname PYG@tok@kp\endcsname{\def\PYG@tc##1{\textcolor[rgb]{0.00,0.44,0.13}{##1}}}
\expandafter\def\csname PYG@tok@w\endcsname{\def\PYG@tc##1{\textcolor[rgb]{0.73,0.73,0.73}{##1}}}
\expandafter\def\csname PYG@tok@kt\endcsname{\def\PYG@tc##1{\textcolor[rgb]{0.56,0.13,0.00}{##1}}}
\expandafter\def\csname PYG@tok@sc\endcsname{\def\PYG@tc##1{\textcolor[rgb]{0.25,0.44,0.63}{##1}}}
\expandafter\def\csname PYG@tok@sb\endcsname{\def\PYG@tc##1{\textcolor[rgb]{0.25,0.44,0.63}{##1}}}
\expandafter\def\csname PYG@tok@k\endcsname{\let\PYG@bf=\textbf\def\PYG@tc##1{\textcolor[rgb]{0.00,0.44,0.13}{##1}}}
\expandafter\def\csname PYG@tok@se\endcsname{\let\PYG@bf=\textbf\def\PYG@tc##1{\textcolor[rgb]{0.25,0.44,0.63}{##1}}}
\expandafter\def\csname PYG@tok@sd\endcsname{\let\PYG@it=\textit\def\PYG@tc##1{\textcolor[rgb]{0.25,0.44,0.63}{##1}}}

\def\PYGZbs{\char`\\}
\def\PYGZus{\char`\_}
\def\PYGZob{\char`\{}
\def\PYGZcb{\char`\}}
\def\PYGZca{\char`\^}
\def\PYGZam{\char`\&}
\def\PYGZlt{\char`\<}
\def\PYGZgt{\char`\>}
\def\PYGZsh{\char`\#}
\def\PYGZpc{\char`\%}
\def\PYGZdl{\char`\$}
\def\PYGZhy{\char`\-}
\def\PYGZsq{\char`\'}
\def\PYGZdq{\char`\"}
\def\PYGZti{\char`\~}
% for compatibility with earlier versions
\def\PYGZat{@}
\def\PYGZlb{[}
\def\PYGZrb{]}
\makeatother

\renewcommand\PYGZsq{\textquotesingle}

\begin{document}

\maketitle
\tableofcontents
\phantomsection\label{index::doc}



\chapter{Kid's activity}
\label{index:technotes-by-tom-jiang-fr}\label{index:kid-s-activity}

\section{Minecraft Pi Edition}
\label{kid/minecraft::doc}\label{kid/minecraft:minecraft-pi-edition}

\subsection{Basic commands}
\label{kid/minecraft:basic-commands}
\begin{tabulary}{\linewidth}{|L|L|}
\hline

\textbf{W}
 & 
move forward
\\
\hline
\textbf{S}
 & 
move backward
\\
\hline
\textbf{A}
 & 
move left
\\
\hline
\textbf{D}
 & 
move right
\\
\hline
\textbf{E}
 & 
show inventory of blocks
\\
\hline
\textbf{1}-\textbf{8}
 & 
select items in the quick bar
\\
\hline
\textbf{Space} / \textbf{Ctrl} + \textbf{Space}
 & 
jump (ascend in fly-mode)
\\
\hline
\textbf{Shift} / \textbf{Ctrl} + \textbf{Shift}
 & 
sneak (descend in fly-mode)
\\
\hline
\textbf{ESC}
 & 
pause / menu
\\
\hline
left mouse
 & 
destroy blocks
\\
\hline
right mouse
 & 
place blocks
\\
\hline
double \textbf{Space}
 & 
fly / fall
\\
\hline
\textbf{Tab}
 & 
release mouse
\\
\hline\end{tabulary}



\subsection{List of python programs}
\label{kid/minecraft:list-of-python-programs}

\subsubsection{Short-cuts}
\label{kid/minecraft:short-cuts}
\begin{tabulary}{\linewidth}{|L|L|}
\hline

\textbf{Ctrl} + \textbf{S}
 & 
save
\\
\hline
\textbf{F5}
 & 
run
\\
\hline\end{tabulary}



\subsubsection{Display the player's position}
\label{kid/minecraft:display-the-player-s-position}
\begin{Verbatim}[commandchars=\\\{\},numbers=left,firstnumber=1,stepnumber=1]
\PYG{k+kn}{from} \PYG{n+nn}{mcpi} \PYG{k+kn}{import} \PYG{n}{minecraft}

\PYG{n}{mc} \PYG{o}{=} \PYG{n}{minecraft}\PYG{o}{.}\PYG{n}{Minecraft}\PYG{o}{.}\PYG{n}{create}\PYG{p}{(}\PYG{p}{)}

\PYG{n}{x}\PYG{p}{,}\PYG{n}{y}\PYG{p}{,}\PYG{n}{z} \PYG{o}{=} \PYG{n}{mc}\PYG{o}{.}\PYG{n}{player}\PYG{o}{.}\PYG{n}{getTilePos}\PYG{p}{(}\PYG{p}{)}
\PYG{n}{mc}\PYG{o}{.}\PYG{n}{postToChat}\PYG{p}{(}\PYG{l+s}{\PYGZdq{}}\PYG{l+s}{x=}\PYG{l+s}{\PYGZdq{}}\PYG{o}{+}\PYG{n+nb}{str}\PYG{p}{(}\PYG{n}{x}\PYG{p}{)}\PYG{o}{+}\PYG{l+s}{\PYGZdq{}}\PYG{l+s}{, y=}\PYG{l+s}{\PYGZdq{}}\PYG{o}{+}\PYG{n+nb}{str}\PYG{p}{(}\PYG{n}{y}\PYG{p}{)}\PYG{o}{+}\PYG{l+s}{\PYGZdq{}}\PYG{l+s}{, z=}\PYG{l+s}{\PYGZdq{}}\PYG{o}{+}\PYG{n+nb}{str}\PYG{p}{(}\PYG{n}{z}\PYG{p}{)}\PYG{p}{)}
\end{Verbatim}


\subsubsection{Teleport (change the player's position)}
\label{kid/minecraft:teleport-change-the-player-s-position}
In the following program, the player will be teleported 100 higher.

\begin{Verbatim}[commandchars=\\\{\},numbers=left,firstnumber=1,stepnumber=1]
\PYG{k+kn}{from} \PYG{n+nn}{mcpi} \PYG{k+kn}{import} \PYG{n}{minecraft}

\PYG{n}{mc} \PYG{o}{=} \PYG{n}{minecraft}\PYG{o}{.}\PYG{n}{Minecraft}\PYG{o}{.}\PYG{n}{create}\PYG{p}{(}\PYG{p}{)}

\PYG{n}{x}\PYG{p}{,}\PYG{n}{y}\PYG{p}{,}\PYG{n}{z} \PYG{o}{=} \PYG{n}{mc}\PYG{o}{.}\PYG{n}{player}\PYG{o}{.}\PYG{n}{getTilePos}\PYG{p}{(}\PYG{p}{)}
\PYG{n}{mc}\PYG{o}{.}\PYG{n}{player}\PYG{o}{.}\PYG{n}{setPos}\PYG{p}{(}\PYG{n}{x}\PYG{p}{,}\PYG{n}{y}\PYG{o}{+}\PYG{l+m+mi}{100}\PYG{p}{,}\PYG{n}{z}\PYG{p}{)}
\end{Verbatim}


\subsubsection{Build a huge block of activated TNTs}
\label{kid/minecraft:build-a-huge-block-of-activated-tnts}
When you click one TNT, there will be an explosion around that block of TNTs.

\begin{Verbatim}[commandchars=\\\{\},numbers=left,firstnumber=1,stepnumber=1]
\PYG{k+kn}{from} \PYG{n+nn}{mcpi} \PYG{k+kn}{import} \PYG{n}{minecraft}

\PYG{n}{mc} \PYG{o}{=} \PYG{n}{minecraft}\PYG{o}{.}\PYG{n}{Minecraft}\PYG{o}{.}\PYG{n}{create}\PYG{p}{(}\PYG{p}{)}

\PYG{n}{x}\PYG{p}{,}\PYG{n}{y}\PYG{p}{,}\PYG{n}{z} \PYG{o}{=} \PYG{n}{mc}\PYG{o}{.}\PYG{n}{player}\PYG{o}{.}\PYG{n}{getTilePos}\PYG{p}{(}\PYG{p}{)}

\PYG{n}{tnt} \PYG{o}{=} \PYG{l+m+mi}{46}
\PYG{n}{activated} \PYG{o}{=} \PYG{l+m+mi}{1}
\PYG{n}{mc}\PYG{o}{.}\PYG{n}{setBlocks}\PYG{p}{(}\PYG{n}{x}\PYG{o}{+}\PYG{l+m+mi}{1}\PYG{p}{,}\PYG{n}{y}\PYG{o}{+}\PYG{l+m+mi}{1}\PYG{p}{,}\PYG{n}{z}\PYG{o}{+}\PYG{l+m+mi}{1}\PYG{p}{,}\PYG{n}{x}\PYG{o}{+}\PYG{l+m+mi}{5}\PYG{p}{,}\PYG{n}{y}\PYG{o}{+}\PYG{l+m+mi}{5}\PYG{p}{,}\PYG{n}{z}\PYG{o}{+}\PYG{l+m+mi}{5}\PYG{p}{,}\PYG{n}{tnt}\PYG{p}{,}\PYG{n}{activated}\PYG{p}{)}
\end{Verbatim}


\subsubsection{Put a flower on the path}
\label{kid/minecraft:put-a-flower-on-the-path}
We will leave a flower when we are on a block of grass. Otherwise we will change the beneath block to a grass block.

\begin{Verbatim}[commandchars=\\\{\},numbers=left,firstnumber=1,stepnumber=1]
\PYG{k+kn}{from} \PYG{n+nn}{mcpi} \PYG{k+kn}{import} \PYG{n}{minecraft}
\PYG{k+kn}{from} \PYG{n+nn}{time} \PYG{k+kn}{import} \PYG{n}{sleep}

\PYG{n}{mc} \PYG{o}{=} \PYG{n}{minecraft}\PYG{o}{.}\PYG{n}{Minecraft}\PYG{o}{.}\PYG{n}{create}\PYG{p}{(}\PYG{p}{)}

\PYG{n}{grass} \PYG{o}{=} \PYG{l+m+mi}{2}
\PYG{n}{flower} \PYG{o}{=} \PYG{l+m+mi}{38}
\PYG{k}{while} \PYG{n+nb+bp}{True}\PYG{p}{:}
    \PYG{n}{x}\PYG{p}{,}\PYG{n}{y}\PYG{p}{,}\PYG{n}{z} \PYG{o}{=} \PYG{n}{mc}\PYG{o}{.}\PYG{n}{player}\PYG{o}{.}\PYG{n}{getTilePos}\PYG{p}{(}\PYG{p}{)}
    \PYG{n}{block\PYGZus{}beneath} \PYG{o}{=} \PYG{n}{mc}\PYG{o}{.}\PYG{n}{getBlock}\PYG{p}{(}\PYG{n}{x}\PYG{p}{,}\PYG{n}{y}\PYG{o}{\PYGZhy{}}\PYG{l+m+mi}{1}\PYG{p}{,}\PYG{n}{z}\PYG{p}{)}
    \PYG{k}{if} \PYG{n}{block\PYGZus{}beneath} \PYG{o}{==} \PYG{n}{grass}\PYG{p}{:}
        \PYG{n}{mc}\PYG{o}{.}\PYG{n}{setBlock}\PYG{p}{(}\PYG{n}{x}\PYG{p}{,}\PYG{n}{y}\PYG{p}{,}\PYG{n}{z}\PYG{p}{,}\PYG{n}{flower}\PYG{p}{)}
    \PYG{k}{else}\PYG{p}{:}
        \PYG{n}{mc}\PYG{o}{.}\PYG{n}{setBlock}\PYG{p}{(}\PYG{n}{x}\PYG{p}{,}\PYG{n}{y}\PYG{o}{\PYGZhy{}}\PYG{l+m+mi}{1}\PYG{p}{,}\PYG{n}{z}\PYG{p}{,}\PYG{n}{grass}\PYG{p}{)}
    \PYG{n}{sleep}\PYG{p}{(}\PYG{l+m+mf}{0.1}\PYG{p}{)}
\end{Verbatim}


\subsubsection{Clear space with input size}
\label{kid/minecraft:clear-space-with-input-size}
We will clear space for a given \textbf{size}. To do so, we will build a cube of \textbf{size} x \textbf{size} x \textbf{size} blocks, filled with the AIR block.

\begin{Verbatim}[commandchars=\\\{\},numbers=left,firstnumber=1,stepnumber=1]
\PYG{k+kn}{from} \PYG{n+nn}{mcpi} \PYG{k+kn}{import} \PYG{n}{minecraft}\PYG{p}{,} \PYG{n}{block}

\PYG{n}{mc} \PYG{o}{=} \PYG{n}{minecraft}\PYG{o}{.}\PYG{n}{Minecraft}\PYG{o}{.}\PYG{n}{create}\PYG{p}{(}\PYG{p}{)}

\PYG{n}{x}\PYG{p}{,}\PYG{n}{y}\PYG{p}{,}\PYG{n}{z} \PYG{o}{=} \PYG{n}{mc}\PYG{o}{.}\PYG{n}{player}\PYG{o}{.}\PYG{n}{getTilePos}\PYG{p}{(}\PYG{p}{)}
\PYG{n}{size} \PYG{o}{=} \PYG{n+nb}{int}\PYG{p}{(}\PYG{n+nb}{raw\PYGZus{}input}\PYG{p}{(}\PYG{l+s}{\PYGZdq{}}\PYG{l+s}{size of area to clear? }\PYG{l+s}{\PYGZdq{}}\PYG{p}{)}\PYG{p}{)}
\PYG{k}{if} \PYG{n}{size} \PYG{o}{\PYGZgt{}} \PYG{l+m+mi}{0}\PYG{p}{:}
    \PYG{n}{mc}\PYG{o}{.}\PYG{n}{setBlocks}\PYG{p}{(}\PYG{n}{x}\PYG{p}{,}\PYG{n}{y}\PYG{p}{,}\PYG{n}{z}\PYG{p}{,}\PYG{n}{x}\PYG{o}{+}\PYG{n}{size}\PYG{p}{,}\PYG{n}{y}\PYG{o}{+}\PYG{n}{size}\PYG{p}{,}\PYG{n}{z}\PYG{o}{+}\PYG{n}{size}\PYG{p}{,}\PYG{n}{block}\PYG{o}{.}\PYG{n}{AIR}\PYG{o}{.}\PYG{n}{id}\PYG{p}{)}
\end{Verbatim}

Challenge: Change a little the above program so that the player is in the middle of the cleared space (and also dig down a few blocks).


\subsubsection{Build a house, then a street}
\label{kid/minecraft:build-a-house-then-a-street}
\begin{Verbatim}[commandchars=\\\{\},numbers=left,firstnumber=1,stepnumber=1]
\PYG{k+kn}{from} \PYG{n+nn}{mcpi} \PYG{k+kn}{import} \PYG{n}{minecraft}\PYG{p}{,} \PYG{n}{block}

\PYG{n}{mc} \PYG{o}{=} \PYG{n}{minecraft}\PYG{o}{.}\PYG{n}{Minecraft}\PYG{o}{.}\PYG{n}{create}\PYG{p}{(}\PYG{p}{)}
\PYG{n}{SIZE} \PYG{o}{=} \PYG{l+m+mi}{20}

\PYG{k}{def} \PYG{n+nf}{house}\PYG{p}{(}\PYG{p}{)}\PYG{p}{:}
    \PYG{n}{midx} \PYG{o}{=} \PYG{n}{x} \PYG{o}{+} \PYG{n}{SIZE}\PYG{o}{/}\PYG{l+m+mi}{2}
    \PYG{n}{midy} \PYG{o}{=} \PYG{n}{y} \PYG{o}{+} \PYG{n}{SIZE}\PYG{o}{/}\PYG{l+m+mi}{2}
    \PYG{n}{mc}\PYG{o}{.}\PYG{n}{setBlocks}\PYG{p}{(}     \PYG{n}{x}\PYG{p}{,}       \PYG{n}{y}\PYG{p}{,}  \PYG{n}{z}\PYG{p}{,}  \PYG{n}{x}\PYG{o}{+}\PYG{n}{SIZE}\PYG{p}{,}  \PYG{n}{y}\PYG{o}{+}\PYG{n}{SIZE}\PYG{p}{,}  \PYG{n}{z}\PYG{o}{+}\PYG{n}{SIZE}\PYG{p}{,}\PYG{n}{block}\PYG{o}{.}\PYG{n}{COBBLESTONE}\PYG{o}{.}\PYG{n}{id}\PYG{p}{)}
    \PYG{n}{mc}\PYG{o}{.}\PYG{n}{setBlocks}\PYG{p}{(}   \PYG{n}{x}\PYG{o}{+}\PYG{l+m+mi}{1}\PYG{p}{,}     \PYG{n}{y}\PYG{o}{+}\PYG{l+m+mi}{1}\PYG{p}{,}\PYG{n}{z}\PYG{o}{+}\PYG{l+m+mi}{1}\PYG{p}{,}\PYG{n}{x}\PYG{o}{+}\PYG{n}{SIZE}\PYG{o}{\PYGZhy{}}\PYG{l+m+mi}{1}\PYG{p}{,}\PYG{n}{y}\PYG{o}{+}\PYG{n}{SIZE}\PYG{o}{\PYGZhy{}}\PYG{l+m+mi}{1}\PYG{p}{,}\PYG{n}{z}\PYG{o}{+}\PYG{n}{SIZE}\PYG{o}{\PYGZhy{}}\PYG{l+m+mi}{1}\PYG{p}{,}        \PYG{n}{block}\PYG{o}{.}\PYG{n}{AIR}\PYG{o}{.}\PYG{n}{id}\PYG{p}{)}
    \PYG{c}{\PYGZsh{} left window}
    \PYG{n}{mc}\PYG{o}{.}\PYG{n}{setBlocks}\PYG{p}{(}   \PYG{n}{x}\PYG{o}{+}\PYG{l+m+mi}{3}\PYG{p}{,}\PYG{n}{y}\PYG{o}{+}\PYG{n}{SIZE}\PYG{o}{\PYGZhy{}}\PYG{l+m+mi}{3}\PYG{p}{,}  \PYG{n}{z}\PYG{p}{,}  \PYG{n}{midx}\PYG{o}{\PYGZhy{}}\PYG{l+m+mi}{3}\PYG{p}{,}  \PYG{n}{midy}\PYG{o}{+}\PYG{l+m+mi}{3}\PYG{p}{,}       \PYG{n}{z}\PYG{p}{,}      \PYG{n}{block}\PYG{o}{.}\PYG{n}{GLASS}\PYG{o}{.}\PYG{n}{id}\PYG{p}{)}
    \PYG{c}{\PYGZsh{} right window}
    \PYG{n}{mc}\PYG{o}{.}\PYG{n}{setBlocks}\PYG{p}{(}\PYG{n}{midx}\PYG{o}{+}\PYG{l+m+mi}{3}\PYG{p}{,}\PYG{n}{y}\PYG{o}{+}\PYG{n}{SIZE}\PYG{o}{\PYGZhy{}}\PYG{l+m+mi}{3}\PYG{p}{,}  \PYG{n}{z}\PYG{p}{,}\PYG{n}{x}\PYG{o}{+}\PYG{n}{SIZE}\PYG{o}{\PYGZhy{}}\PYG{l+m+mi}{3}\PYG{p}{,}  \PYG{n}{midy}\PYG{o}{+}\PYG{l+m+mi}{3}\PYG{p}{,}       \PYG{n}{z}\PYG{p}{,}      \PYG{n}{block}\PYG{o}{.}\PYG{n}{GLASS}\PYG{o}{.}\PYG{n}{id}\PYG{p}{)}
    \PYG{c}{\PYGZsh{} door}
    \PYG{n}{mc}\PYG{o}{.}\PYG{n}{setBlocks}\PYG{p}{(}\PYG{n}{midx}\PYG{o}{\PYGZhy{}}\PYG{l+m+mi}{3}\PYG{p}{,}       \PYG{n}{y}\PYG{p}{,}  \PYG{n}{z}\PYG{p}{,}  \PYG{n}{midx}\PYG{o}{+}\PYG{l+m+mi}{3}\PYG{p}{,}    \PYG{n}{midy}\PYG{p}{,}       \PYG{n}{z}\PYG{p}{,}  \PYG{n}{block}\PYG{o}{.}\PYG{n}{DOOR\PYGZus{}WOOD}\PYG{o}{.}\PYG{n}{id}\PYG{p}{)}
    \PYG{n}{mc}\PYG{o}{.}\PYG{n}{setBlocks}\PYG{p}{(}     \PYG{n}{x}\PYG{p}{,}  \PYG{n}{y}\PYG{o}{+}\PYG{n}{SIZE}\PYG{p}{,}  \PYG{n}{z}\PYG{p}{,}  \PYG{n}{x}\PYG{o}{+}\PYG{n}{SIZE}\PYG{p}{,}  \PYG{n}{y}\PYG{o}{+}\PYG{n}{SIZE}\PYG{p}{,}  \PYG{n}{z}\PYG{o}{+}\PYG{n}{SIZE}\PYG{p}{,}       \PYG{n}{block}\PYG{o}{.}\PYG{n}{SNOW}\PYG{o}{.}\PYG{n}{id}\PYG{p}{)}
    \PYG{n}{mc}\PYG{o}{.}\PYG{n}{setBlocks}\PYG{p}{(}   \PYG{n}{x}\PYG{o}{+}\PYG{l+m+mi}{1}\PYG{p}{,}     \PYG{n}{y}\PYG{o}{+}\PYG{l+m+mi}{1}\PYG{p}{,}\PYG{n}{z}\PYG{o}{+}\PYG{l+m+mi}{1}\PYG{p}{,}\PYG{n}{x}\PYG{o}{+}\PYG{n}{SIZE}\PYG{o}{\PYGZhy{}}\PYG{l+m+mi}{1}\PYG{p}{,}     \PYG{n}{y}\PYG{o}{+}\PYG{l+m+mi}{1}\PYG{p}{,}\PYG{n}{z}\PYG{o}{+}\PYG{n}{SIZE}\PYG{o}{\PYGZhy{}}\PYG{l+m+mi}{1}\PYG{p}{,}       \PYG{n}{block}\PYG{o}{.}\PYG{n}{WOOL}\PYG{o}{.}\PYG{n}{id}\PYG{p}{,}\PYG{l+m+mi}{7}\PYG{p}{)}

\PYG{n}{x}\PYG{p}{,}\PYG{n}{y}\PYG{p}{,}\PYG{n}{z} \PYG{o}{=} \PYG{n}{mc}\PYG{o}{.}\PYG{n}{player}\PYG{o}{.}\PYG{n}{getTilePos}\PYG{p}{(}\PYG{p}{)}

\PYG{c}{\PYGZsh{} build a house}
\PYG{n}{house}\PYG{p}{(}\PYG{p}{)}

\PYG{c}{\PYGZsh{} build a street}
\PYG{k}{for} \PYG{n}{h} \PYG{o+ow}{in} \PYG{n+nb}{range}\PYG{p}{(}\PYG{l+m+mi}{5}\PYG{p}{)}\PYG{p}{:}
    \PYG{n}{house}\PYG{p}{(}\PYG{p}{)}
    \PYG{n}{x} \PYG{o}{=} \PYG{n}{x}\PYG{o}{+}\PYG{n}{SIZE}
\end{Verbatim}


\section{Pygame}
\label{kid/pygame::doc}\label{kid/pygame:pygame}

\subsection{List of pygame programs}
\label{kid/pygame:list-of-pygame-programs}

\subsubsection{Draw a circle}
\label{kid/pygame:draw-a-circle}
\begin{Verbatim}[commandchars=\\\{\},numbers=left,firstnumber=1,stepnumber=1]
\PYG{k+kn}{import} \PYG{n+nn}{pygame}

\PYG{n}{width}\PYG{p}{,}\PYG{n}{height} \PYG{o}{=} \PYG{l+m+mi}{640}\PYG{p}{,}\PYG{l+m+mi}{480}
\PYG{n}{radius} \PYG{o}{=} \PYG{l+m+mi}{100}
\PYG{n}{fill} \PYG{o}{=} \PYG{l+m+mi}{1}

\PYG{n}{pygame}\PYG{o}{.}\PYG{n}{init}\PYG{p}{(}\PYG{p}{)}
\PYG{n}{window} \PYG{o}{=} \PYG{n}{pygame}\PYG{o}{.}\PYG{n}{display}\PYG{o}{.}\PYG{n}{set\PYGZus{}mode}\PYG{p}{(}\PYG{p}{(}\PYG{n}{width}\PYG{p}{,}\PYG{n}{height}\PYG{p}{)}\PYG{p}{)}
\PYG{n}{window}\PYG{o}{.}\PYG{n}{fill}\PYG{p}{(}\PYG{n}{pygame}\PYG{o}{.}\PYG{n}{Color}\PYG{p}{(}\PYG{l+m+mi}{255}\PYG{p}{,}\PYG{l+m+mi}{255}\PYG{p}{,}\PYG{l+m+mi}{255}\PYG{p}{)}\PYG{p}{)} \PYG{c}{\PYGZsh{} white}

\PYG{k}{while} \PYG{n+nb+bp}{True}\PYG{p}{:}
    \PYG{n}{pygame}\PYG{o}{.}\PYG{n}{draw}\PYG{o}{.}\PYG{n}{circle}\PYG{p}{(}\PYG{n}{window}\PYG{p}{,}
                       \PYG{n}{pygame}\PYG{o}{.}\PYG{n}{Color}\PYG{p}{(}\PYG{l+m+mi}{255}\PYG{p}{,}\PYG{l+m+mi}{0}\PYG{p}{,}\PYG{l+m+mi}{0}\PYG{p}{)}\PYG{p}{,} \PYG{c}{\PYGZsh{} red}
                       \PYG{p}{(}\PYG{n}{width}\PYG{o}{/}\PYG{l+m+mi}{2}\PYG{p}{,}\PYG{n}{height}\PYG{o}{/}\PYG{l+m+mi}{2}\PYG{p}{)}\PYG{p}{,}
                       \PYG{n}{radius}\PYG{p}{,}
                       \PYG{n}{fill}\PYG{p}{)}
    \PYG{n}{pygame}\PYG{o}{.}\PYG{n}{display}\PYG{o}{.}\PYG{n}{update}\PYG{p}{(}\PYG{p}{)}
    \PYG{k}{if} \PYG{n}{pygame}\PYG{o}{.}\PYG{n}{QUIT} \PYG{o+ow}{in} \PYG{p}{[}\PYG{n}{e}\PYG{o}{.}\PYG{n}{type} \PYG{k}{for} \PYG{n}{e} \PYG{o+ow}{in} \PYG{n}{pygame}\PYG{o}{.}\PYG{n}{event}\PYG{o}{.}\PYG{n}{get}\PYG{p}{(}\PYG{p}{)}\PYG{p}{]}\PYG{p}{:}
        \PYG{k}{break}
\end{Verbatim}


\subsubsection{Draw circles based on mouse move / position}
\label{kid/pygame:draw-circles-based-on-mouse-move-position}
\begin{Verbatim}[commandchars=\\\{\},numbers=left,firstnumber=1,stepnumber=1]
\PYG{k+kn}{import} \PYG{n+nn}{pygame}
\PYG{k+kn}{from} \PYG{n+nn}{pygame.locals} \PYG{k+kn}{import} \PYG{o}{*}

\PYG{n}{width}\PYG{p}{,}\PYG{n}{height} \PYG{o}{=} \PYG{l+m+mi}{640}\PYG{p}{,}\PYG{l+m+mi}{640}
\PYG{n}{radius} \PYG{o}{=} \PYG{l+m+mi}{0}
\PYG{n}{fill} \PYG{o}{=} \PYG{l+m+mi}{1}
\PYG{n}{mouseX}\PYG{p}{,}\PYG{n}{mouseY} \PYG{o}{=} \PYG{l+m+mi}{0}\PYG{p}{,}\PYG{l+m+mi}{0}

\PYG{n}{pygame}\PYG{o}{.}\PYG{n}{init}\PYG{p}{(}\PYG{p}{)}
\PYG{n}{window} \PYG{o}{=} \PYG{n}{pygame}\PYG{o}{.}\PYG{n}{display}\PYG{o}{.}\PYG{n}{set\PYGZus{}mode}\PYG{p}{(}\PYG{p}{(}\PYG{n}{width}\PYG{p}{,}\PYG{n}{height}\PYG{p}{)}\PYG{p}{)}
\PYG{n}{window}\PYG{o}{.}\PYG{n}{fill}\PYG{p}{(}\PYG{n}{pygame}\PYG{o}{.}\PYG{n}{Color}\PYG{p}{(}\PYG{l+m+mi}{255}\PYG{p}{,}\PYG{l+m+mi}{255}\PYG{p}{,}\PYG{l+m+mi}{255}\PYG{p}{)}\PYG{p}{)} \PYG{c}{\PYGZsh{} white}
\PYG{n}{fps} \PYG{o}{=} \PYG{n}{pygame}\PYG{o}{.}\PYG{n}{time}\PYG{o}{.}\PYG{n}{Clock}\PYG{p}{(}\PYG{p}{)} \PYG{c}{\PYGZsh{} FPS = Frame Per Second}

\PYG{k}{while} \PYG{n+nb+bp}{True}\PYG{p}{:}  \PYG{c}{\PYGZsh{} one frame per loop}
    \PYG{k}{for} \PYG{n}{event} \PYG{o+ow}{in} \PYG{n}{pygame}\PYG{o}{.}\PYG{n}{event}\PYG{o}{.}\PYG{n}{get}\PYG{p}{(}\PYG{p}{)}\PYG{p}{:}
        \PYG{k}{if} \PYG{n}{event}\PYG{o}{.}\PYG{n}{type} \PYG{o}{==} \PYG{n}{MOUSEMOTION}\PYG{p}{:}
            \PYG{n}{mouseX}\PYG{p}{,}\PYG{n}{mouseY} \PYG{o}{=} \PYG{n}{event}\PYG{o}{.}\PYG{n}{pos}
        \PYG{k}{if} \PYG{n}{event}\PYG{o}{.}\PYG{n}{type} \PYG{o}{==} \PYG{n}{MOUSEBUTTONDOWN}\PYG{p}{:} \PYG{c}{\PYGZsh{} mouse click}
            \PYG{n}{window}\PYG{o}{.}\PYG{n}{fill}\PYG{p}{(}\PYG{n}{pygame}\PYG{o}{.}\PYG{n}{Color}\PYG{p}{(}\PYG{l+m+mi}{255}\PYG{p}{,}\PYG{l+m+mi}{255}\PYG{p}{,}\PYG{l+m+mi}{255}\PYG{p}{)}\PYG{p}{)} \PYG{c}{\PYGZsh{} clear screen}
        \PYG{n}{radius} \PYG{o}{=} \PYG{p}{(}\PYG{n+nb}{abs}\PYG{p}{(}\PYG{n}{width}\PYG{o}{/}\PYG{l+m+mi}{2} \PYG{o}{\PYGZhy{}} \PYG{n}{mouseX}\PYG{p}{)} \PYG{o}{+} \PYG{n+nb}{abs}\PYG{p}{(}\PYG{n}{height}\PYG{o}{/}\PYG{l+m+mi}{2} \PYG{o}{\PYGZhy{}} \PYG{n}{mouseY}\PYG{p}{)}\PYG{p}{)}\PYG{o}{/}\PYG{l+m+mi}{2} \PYG{o}{+} \PYG{l+m+mi}{1}
        \PYG{n}{pygame}\PYG{o}{.}\PYG{n}{draw}\PYG{o}{.}\PYG{n}{circle}\PYG{p}{(}\PYG{n}{window}\PYG{p}{,}
                           \PYG{n}{pygame}\PYG{o}{.}\PYG{n}{Color}\PYG{p}{(}\PYG{l+m+mi}{255}\PYG{p}{,}\PYG{l+m+mi}{0}\PYG{p}{,}\PYG{l+m+mi}{0}\PYG{p}{)}\PYG{p}{,} \PYG{c}{\PYGZsh{} red}
                           \PYG{p}{(}\PYG{n}{mouseX}\PYG{p}{,}\PYG{n}{mouseY}\PYG{p}{)}\PYG{p}{,}
                           \PYG{n}{radius}\PYG{p}{,}
                           \PYG{n}{fill}\PYG{p}{)}
    \PYG{n}{pygame}\PYG{o}{.}\PYG{n}{display}\PYG{o}{.}\PYG{n}{update}\PYG{p}{(}\PYG{p}{)}
    \PYG{k}{if} \PYG{n}{pygame}\PYG{o}{.}\PYG{n}{QUIT} \PYG{o+ow}{in} \PYG{p}{[}\PYG{n}{e}\PYG{o}{.}\PYG{n}{type} \PYG{k}{for} \PYG{n}{e} \PYG{o+ow}{in} \PYG{n}{pygame}\PYG{o}{.}\PYG{n}{event}\PYG{o}{.}\PYG{n}{get}\PYG{p}{(}\PYG{p}{)}\PYG{p}{]}\PYG{p}{:}
        \PYG{k}{break}
    \PYG{n}{fps}\PYG{o}{.}\PYG{n}{tick}\PYG{p}{(}\PYG{l+m+mi}{30}\PYG{p}{)} \PYG{c}{\PYGZsh{} wait so that frame rate is 30 fps}
\end{Verbatim}


\section{Scratch}
\label{kid/scratch:scratch}\label{kid/scratch::doc}

\chapter{Hardware}
\label{index:hardware}

\section{Raspberry Pi}
\label{hardware/raspberrypi::doc}\label{hardware/raspberrypi:raspberry-pi}

\subsection{Default settings}
\label{hardware/raspberrypi:default-settings}
\begin{tabulary}{\linewidth}{|L|L|}
\hline

login
 & 
\textbf{pi}
\\
\hline
password
 & 
\textbf{raspberry}
\\
\hline
hostname
 & 
\textbf{raspberrypi}
\\
\hline
keyboard
 & 
UK
\\
\hline\end{tabulary}



\subsection{Basic commands}
\label{hardware/raspberrypi:basic-commands}

\subsubsection{Config}
\label{hardware/raspberrypi:config}
\begin{Verbatim}[commandchars=\\\{\}]
\PYG{n+nv}{\PYGZdl{} }sudo raspi\PYGZhy{}config
\end{Verbatim}


\subsubsection{Start X server}
\label{hardware/raspberrypi:start-x-server}
\begin{Verbatim}[commandchars=\\\{\}]
\PYG{n+nv}{\PYGZdl{} }startx
\end{Verbatim}


\subsubsection{Reboot}
\label{hardware/raspberrypi:reboot}
\begin{Verbatim}[commandchars=\\\{\}]
\PYG{n+nv}{\PYGZdl{} }sudo reboot
\end{Verbatim}


\subsubsection{Shutdown}
\label{hardware/raspberrypi:shutdown}
\begin{Verbatim}[commandchars=\\\{\}]
\PYG{n+nv}{\PYGZdl{} }sudo shutdown \PYGZhy{}h now
\end{Verbatim}


\subsubsection{Change datetime}
\label{hardware/raspberrypi:change-datetime}
\begin{Verbatim}[commandchars=\\\{\}]
\PYG{n+nv}{\PYGZdl{} }sudo date \PYGZhy{}\PYGZhy{}set\PYG{o}{=}\PYG{l+s+s2}{\PYGZdq{}Sun Nov 18 1:55:16 EDT 2012\PYGZdq{}}
\end{Verbatim}


\subsubsection{Update}
\label{hardware/raspberrypi:update}
\begin{Verbatim}[commandchars=\\\{\}]
\PYG{n+nv}{\PYGZdl{} }sudo apt\PYGZhy{}get update
\PYG{n+nv}{\PYGZdl{} }sudo apt\PYGZhy{}get upgrade
\end{Verbatim}


\subsection{Information}
\label{hardware/raspberrypi:information}

\subsubsection{Check OS version}
\label{hardware/raspberrypi:check-os-version}
\begin{Verbatim}[commandchars=\\\{\}]
\PYG{n+nv}{\PYGZdl{} }cat /proc/version
\end{Verbatim}


\subsubsection{Check board version}
\label{hardware/raspberrypi:check-board-version}
\begin{Verbatim}[commandchars=\\\{\}]
\PYG{n+nv}{\PYGZdl{} }cat /proc/cpuinfo
\end{Verbatim}


\subsubsection{Display network interface and associated IP addresses}
\label{hardware/raspberrypi:display-network-interface-and-associated-ip-addresses}
\begin{Verbatim}[commandchars=\\\{\}]
\PYG{n+nv}{\PYGZdl{} }ifconfig
\end{Verbatim}


\subsection{Short-cuts}
\label{hardware/raspberrypi:short-cuts}
\begin{tabulary}{\linewidth}{|L|L|}
\hline

\textbf{Ctrl} + \textbf{C}
 & 
kill currently running program
\\
\hline
\textbf{Ctrl} + \textbf{D}
 & 
exit shell
\\
\hline
\textbf{Ctrl} + \textbf{A}
 & 
move cursor to the beginning of the line
\\
\hline
\textbf{Ctrl} + \textbf{E}
 & 
move cursor to the end of the line
\\
\hline
\textbf{Ctrl} + \textbf{Alt} + \textbf{Backspace}
 & 
{[}optional{]} terminate the X server
\\
\hline\end{tabulary}



\subsection{Setup Keyboard}
\label{hardware/raspberrypi:setup-keyboard}
The default keyboard is UK. Let's change it to AU keyboard.

The trick is that Australia is not listed in the country list for the keyboard, we need to setup a US keyboard instead.


\subsubsection{Change the keyboard config}
\label{hardware/raspberrypi:change-the-keyboard-config}
\begin{Verbatim}[commandchars=\\\{\}]
\PYG{n+nv}{\PYGZdl{} }sudo vi /etc/default/keyboard
\end{Verbatim}

\begin{Verbatim}[commandchars=\\\{\}]
XKBMODEL =\PYGZdq{}pc105\PYGZdq{}
XKBLAYOUT=\PYGZdq{}us\PYGZdq{}
XKBVARIANT=\PYGZdq{}\PYGZdq{}
XKBOPTIONS=\PYGZdq{}\PYGZdq{}

BACKSPACE=\PYGZdq{}guess\PYGZdq{}
\end{Verbatim}


\subsubsection{Then run the following commands and reboot}
\label{hardware/raspberrypi:then-run-the-following-commands-and-reboot}
\begin{Verbatim}[commandchars=\\\{\}]
\PYG{n+nv}{\PYGZdl{} }sudo setxkbmap \PYGZhy{}layout us
\PYG{n+nv}{\PYGZdl{} }sudo udevadm trigger \PYGZhy{}\PYGZhy{}subsysstem\PYGZhy{}match\PYG{o}{=}input \PYGZhy{}\PYGZhy{}action\PYG{o}{=}change
\end{Verbatim}


\subsection{Utilities / Softwares}
\label{hardware/raspberrypi:utilities-softwares}

\subsubsection{raspi-config tool}
\label{hardware/raspberrypi:raspi-config-tool}
\begin{Verbatim}[commandchars=\\\{\}]
\PYG{n+nv}{\PYGZdl{} }sudo apt\PYGZhy{}get install raspi\PYGZhy{}config
\end{Verbatim}


\subsubsection{Minecraft}
\label{hardware/raspberrypi:minecraft}
\begin{Verbatim}[commandchars=\\\{\}]
\PYG{n+nv}{\PYGZdl{} }sudo apt\PYGZhy{}get install minecraft\PYGZhy{}pi
\end{Verbatim}


\subsubsection{Screenshot : scrot}
\label{hardware/raspberrypi:screenshot-scrot}
\begin{Verbatim}[commandchars=\\\{\}]
\PYG{n+nv}{\PYGZdl{} }sudo apt\PYGZhy{}get install scrot
\end{Verbatim}


\subsubsection{Mercurial}
\label{hardware/raspberrypi:mercurial}
\begin{Verbatim}[commandchars=\\\{\}]
\PYG{n+nv}{\PYGZdl{} }sudo apt\PYGZhy{}get install mercurial
\end{Verbatim}


\section{Arduino}
\label{hardware/arduino::doc}\label{hardware/arduino:arduino}

\chapter{System}
\label{index:system}

\section{Linux}
\label{system/linux::doc}\label{system/linux:linux}

\section{Windows}
\label{system/windows:windows}\label{system/windows::doc}

\subsection{Connect to Internet via Ethernet cable (from PC/laptop)}
\label{system/windows:connect-to-internet-via-ethernet-cable-from-pc-laptop}
\textbf{Control Panel} --\textgreater{} \textbf{Network and Internet} --\textgreater{} \textbf{Network Connections}

\textbf{Ctrl} + select local and wireless connections, right click \textbf{Bridge Connections}


\chapter{Editor}
\label{index:editor}

\section{VIM (Vi IMproved)}
\label{editor/vim:vim-vi-improved}\label{editor/vim::doc}

\subsection{Basic commands}
\label{editor/vim:basic-commands}

\subsubsection{Read only (use \textbf{:wq!} to force the modification)}
\label{editor/vim:read-only-use-wq-to-force-the-modification}
\begin{Verbatim}[commandchars=\\\{\}]
\PYG{n+nv}{\PYGZdl{} }vim \PYGZhy{}R file
\end{Verbatim}


\subsubsection{Running shell commands}
\label{editor/vim:running-shell-commands}
\begin{Verbatim}[commandchars=\\\{\}]
!command
\end{Verbatim}

e.g. \textbf{!ls} will launch \textbf{ls}

if you wants to go directly to shell without quitting from VI editor you can go by executing \textbf{!sh} / \textbf{!bash} / \textbf{!ksh} from VI and then come back to VI editor by just executing command \textbf{exit} from shell.
for Cygwin, \textbf{!bash} and \textbf{exit} seems to be the best choice


\subsubsection{Launch VIM from command line}
\label{editor/vim:launch-vim-from-command-line}
\begin{Verbatim}[commandchars=\\\{\}]
\PYG{n+nv}{\PYGZdl{} }vi file.txt                        open and edit file file.txt
\PYG{n+nv}{\PYGZdl{} }vi file1.txt file2.txt file3.txt   open several files
\PYG{n+nv}{\PYGZdl{} }vi +25 file.txt                    edit from the 25th line
\PYG{n+nv}{\PYGZdl{} }vi + file.txt                      edit at the end of file
\PYG{n+nv}{\PYGZdl{} }vi +/text file.txt                 edit from the first line containing the word \PYG{n+nb}{test}
\PYG{n+nv}{\PYGZdl{} }vi \PYGZhy{}r file.txt                     restore a crashed file
\PYG{n+nv}{\PYGZdl{} }view file.txt                      vi in \PYG{n+nb}{read}\PYGZhy{}only mode
\PYG{n+nv}{\PYGZdl{} }vimtutor                           VIM tutorial
\end{Verbatim}


\subsubsection{Saving and quiting commands}
\label{editor/vim:saving-and-quiting-commands}
\begin{Verbatim}[commandchars=\\\{\}]
:w            save the current file (before quit)
:w file.txt   save the modified file with another file name
              (even if the file was opened in read\PYGZhy{}only mode)
:wq           save and quit
ZZ            save and quit
:q!           quit without saving
:wq!          save change in the current file opened in read\PYGZhy{}only mode, and then quit
:w!           save change in the current file opened in read\PYGZhy{}only mode
\end{Verbatim}


\subsubsection{Checking history and help}
\label{editor/vim:checking-history-and-help}
\begin{Verbatim}[commandchars=\\\{\}]
:history        vim commands history
:help           all helps
:help command   help on one command
\end{Verbatim}


\subsubsection{Recording and replaying commands}
\label{editor/vim:recording-and-replaying-commands}
Recoding in vim or VI editor can be done by using \textbf{q} and the executing recorded comment by using \textbf{q@1}


\subsection{Options}
\label{editor/vim:options}
Here are the major VIM editor options

\begin{tabulary}{\linewidth}{|L|L|}
\hline

:set nu
 & 
This will display line number in front of each line quite useful if you want line by line information.
\\
\hline & 
You can turn it off by executing \textbf{set nonu}. Remember for turning it off put ``no'' in front of option, like here option is ``nu'' so for turning it off use ``nonu''.
\\
\hline
:set nonu
 & 
removing line number display
\\
\hline
:set hlsearch
 & 
This will highlight the matching word when we do search in VI editor, quite useful but if you find it annoying or not able to see sometime due to your color scheme you can turn it off by executing \textbf{set nohlsearch}.
\\
\hline
:set wrap
 & 
If your file has contains some long lines and you want them to wrap use this option, if its already on and you just don't want them to wrap use \textbf{set nowrap}.
\\
\hline
:colorscheme
 & 
color scheme is used to change color of VIM editor, my favorite color scheme is murphy so if you want to change color scheme of VI editor you can do by executing \textbf{colorscheme murphy}.
\\
\hline
:syntax on
 & 
syntax can be turn on and off based on your need, if it's on it will display color syntax for .xml, .html and .perl files.
\\
\hline
:set ignorecase
 & 
This VI editor option allows you do case insensitive search because if it's set VI will not distinguish between two words which are just differ in case.
\\
\hline
:set smartcase
 & 
Another VI editor option which allows case-sensitive search if the word you are searching contains an uppercase character.
\\
\hline\end{tabulary}



\subsection{Navigation}
\label{editor/vim:navigation}
Here are some navigating commands

\begin{tabulary}{\linewidth}{|L|L|}
\hline

gg
 & 
goes to start of file
\\
\hline
shift g
 & 
goes to end of file
\\
\hline
0
 & 
goes to beginning of the line
\\
\hline
\$
 & 
goes to end of the line
\\
\hline
nG
 & 
goes to nth line
\\
\hline
:n
 & 
another way of going to nth line
\\
\hline\end{tabulary}



\subsection{Editing}
\label{editor/vim:editing}

\subsubsection{Editing commands}
\label{editor/vim:editing-commands}
\begin{tabulary}{\linewidth}{|L|L|}
\hline

yy
 & 
equivalent to cut also called yank
\\
\hline
p
 & 
paste below line
\\
\hline
Shift p
 & 
paste above line
\\
\hline
dd
 & 
deletes the current line
\\
\hline
5dd
 & 
deletes 5 lines
\\
\hline
u
 & 
undo last change
\\
\hline
Ctrl + R
 & 
Re do last change
\\
\hline\end{tabulary}



\subsubsection{Copy (or cut) / paste (without strange indent)}
\label{editor/vim:copy-or-cut-paste-without-strange-indent}\begin{enumerate}
\item {} 
move the mouse pointer to the beginning of your desired copy text

\item {} 
type `v' (visual) for Visual mode, then using mouse pointer move to the end of selected text

\item {} 
type `y' (yank) for Copy or `d' (delete) for Cut

\item {} 
move to your paste location, then type `p' (paste)

\end{enumerate}


\subsubsection{Tabulation}
\label{editor/vim:tabulation}\begin{enumerate}
\item {} 
define TAB as 2 spaces

\end{enumerate}

\begin{Verbatim}[commandchars=\\\{\}]
:set tabstop=2 shiftwidth=2 expandtab
\end{Verbatim}
\begin{enumerate}
\setcounter{enumi}{1}
\item {} 
replace TAB by 4 spaces

\end{enumerate}

\begin{Verbatim}[commandchars=\\\{\}]
:\PYGZpc{}s/\PYGZbs{}t/    /g
\end{Verbatim}


\subsection{Multi-files, multi-windows}
\label{editor/vim:multi-files-multi-windows}

\subsubsection{Opening multi-files / another file}
\label{editor/vim:opening-multi-files-another-file}
\begin{Verbatim}[commandchars=\\\{\}]
\PYG{n+nv}{\PYGZdl{} }vim file1 file2 file3 ...
\end{Verbatim}

\begin{Verbatim}[commandchars=\\\{\}]
:n        edit next file among multi\PYGZhy{}files
          (with respect to the order given in the command line)
:wn       save the modification and edit the next file
:n!       edit the next file without saving the ongoing modification
:e        reload the current file
:e file   load file in the current window
\end{Verbatim}


\subsubsection{Multi-windows}
\label{editor/vim:multi-windows}
\begin{Verbatim}[commandchars=\\\{\}]
:sp(lit) file   split horizontally the window and load file in the splitted window
:vsplit file    split vertically the window and load file in the splitted window
:vs             vertically split window
CTRL + w + w    switch among all (sub\PYGZhy{})windows
:q              close the current (sub\PYGZhy{})window
\end{Verbatim}


\subsection{Search and Replace}
\label{editor/vim:search-and-replace}

\subsubsection{Searching commands}
\label{editor/vim:searching-commands}
\begin{Verbatim}[commandchars=\\\{\}]
/Exception   will search for word \PYGZdq{}Exception\PYGZdq{} from top to bottom and stop when it got
             first match, to go to next match type \PYGZdq{}n\PYGZdq{} and for coming back to previous
             match press \PYGZdq{}Shift + N\PYGZdq{}
?Exception   will search for word \PYGZdq{}Exception\PYGZdq{} from bottom to top and stop when it got
             first match, to go to next match type \PYGZdq{}n\PYGZdq{} and for coming back to previous
             match press \PYGZdq{}Shift + N\PYGZdq{}, remember for next match it will go towards top
             of file.
\end{Verbatim}


\subsubsection{Find and replace}
\label{editor/vim:find-and-replace}
\begin{Verbatim}[commandchars=\\\{\}]
:\PYGZpc{}s/Old/New/g     This is an example of global search it will replace all occurrence
                  of word \PYGZdq{}Old\PYGZdq{} in file with word \PYGZdq{}New\PYGZdq{}. Its also equivalent to
                  following command \PYGZdq{}: 0,\PYGZdl{} s/Old/New/g\PYGZdq{} which actually tells that
                  search from fist to last line.
:\PYGZpc{}s/Old/New/gc    This is similar to first command but with the introduction of \PYGZdq{}c\PYGZdq{}
                  it will ask for confirmation
:\PYGZpc{}s/Old/New/gci   This is command is global, case insensitive and ask for confirmation.
                  to make it case Sensitive use \PYGZdq{}I\PYGZdq{}
\end{Verbatim}


\subsubsection{Substitution}
\label{editor/vim:substitution}
Substitution is very useful when working with text. Below you have some example. For more information, you could check the link : \href{http://vim.wikia.com/wiki/Search\_and\_replace}{http://vim.wikia.com/wiki/Search\_and\_replace}

\begin{Verbatim}[commandchars=\\\{\}]
:s/abc/def/           change the first \PYGZsq{}abc\PYGZsq{} of the line to \PYGZsq{}def\PYGZsq{}
:s/abc/def/g          change all \PYGZsq{}abc\PYGZsq{} of the line to \PYGZsq{}def\PYGZsq{}
:\PYGZpc{}s/abc/def/g         change all \PYGZsq{}abc\PYGZsq{} of all lines to \PYGZsq{}def\PYGZsq{}
:\PYGZpc{}s/\PYGZbs{}\PYGZlt{}abc\PYGZbs{}\PYGZgt{}/def/g     change all words \PYGZsq{}abc\PYGZsq{} of all lines to \PYGZsq{}def\PYGZsq{}
:\PYGZpc{}s/\PYGZbs{}\PYGZlt{}abc\PYGZbs{}\PYGZgt{}/def/gI    change all words \PYGZsq{}abc\PYGZsq{} (case sensitive) of all lines to \PYGZsq{}def\PYGZsq{}
:\PYGZpc{}s/\PYGZbs{}\PYGZlt{}abc\PYGZbs{}\PYGZgt{}/def/gci   change all words \PYGZsq{}abc\PYGZsq{} (case insensitive) of all lines to \PYGZsq{}def\PYGZsq{},
                      ask for confirmation
:5,10s/abc/def/g      change all \PYGZsq{}abc\PYGZsq{} to \PYGZsq{}def\PYGZsq{}, from line 5 to line 10 inclusive
:.,+5s/abc/def/g      change all \PYGZsq{}abc\PYGZsq{} to \PYGZsq{}def\PYGZsq{}, for the current line and the 5 next
                      lines
:.,\PYGZdl{}s/abc/def/g       change all \PYGZsq{}abc\PYGZsq{} to \PYGZsq{}def\PYGZsq{}, from the current line to the last line
:g/\PYGZca{}a/s/abc/def/g     change all \PYGZsq{}abc\PYGZsq{} to \PYGZsq{}def\PYGZsq{}, for each line starting with \PYGZsq{}a\PYGZsq{}
\end{Verbatim}


\section{JOE (Joe’s Own Editor)}
\label{editor/joe:joe-joes-own-editor}\label{editor/joe::doc}

\subsection{Basic commands}
\label{editor/joe:basic-commands}

\subsubsection{Launch JOE from command line}
\label{editor/joe:launch-joe-from-command-line}
\begin{Verbatim}[commandchars=\\\{\}]
\PYG{n+nv}{\PYGZdl{} }joe file.txt                     open and edit file.txt
\PYG{n+nv}{\PYGZdl{} }joe \PYGZhy{}wordwrap file.txt           option wordwrap
\PYG{n+nv}{\PYGZdl{} }joe \PYGZhy{}lmargin \PYG{l+m}{5} \PYGZhy{}tab \PYG{l+m}{5} file.txt   left \PYG{n+nv}{margin} \PYG{o}{=} \PYG{l+m}{5} chars and \PYG{n+nv}{TAB} \PYG{o}{=} \PYG{l+m}{5} chars
\PYG{n+nv}{\PYGZdl{} }joe +25 file.txt                 edit from 25th line
\PYG{n+nv}{\PYGZdl{} }jmacs file.txt                   variant : simulate GNU\PYGZhy{}EMACS
\PYG{n+nv}{\PYGZdl{} }jstar file.txt                   variant : simulate WordStar
\PYG{n+nv}{\PYGZdl{} }jpico file.txt                   variant : simulate the Pine mailer editor PICO
\PYG{n+nv}{\PYGZdl{} }rjoe file.txt                    variant : restraint the edit to the file file.txt
                                   only
\end{Verbatim}


\subsubsection{Saving and quiting commands}
\label{editor/joe:saving-and-quiting-commands}
\begin{tabulary}{\linewidth}{|L|L|}
\hline

CTRL + k + d
 & 
save the file
\\
\hline
CTRL + k + x
 & 
save and exit
\\
\hline
CTRL + c
 & 
exit without save
\\
\hline
CTRL + k + z
 & 
exit and leave JOE in background (fg to go back)
\\
\hline\end{tabulary}



\subsubsection{Orthographe}
\label{editor/joe:orthographe}
\begin{tabulary}{\linewidth}{|L|L|}
\hline

CTRL + {[} + n
 & 
check one word
\\
\hline
CTRL + {[} + l
 & 
check one file
\\
\hline\end{tabulary}



\subsubsection{Misc}
\label{editor/joe:misc}
\begin{tabulary}{\linewidth}{|L|L|}
\hline

CTRL + k + a
 & 
move to the middle
\\
\hline
CTRL + t
 & 
display and choose the options
\\
\hline
CTRL + r
 & 
refresh the display
\\
\hline
CTRL + k + h
 & 
display or close the online help
\\
\hline\end{tabulary}



\subsection{Navigation}
\label{editor/joe:navigation}

\subsubsection{Cursor / Move}
\label{editor/joe:cursor-move}
\begin{tabulary}{\linewidth}{|L|L|}
\hline

CTRL + b
 & 
move to left
\\
\hline
CTRL + p
 & 
move to top
\\
\hline
CTRL + f
 & 
move to right
\\
\hline
CTRL + n
 & 
move to down
\\
\hline
CTRL + z
 & 
move to the previous word
\\
\hline
CTRL + x
 & 
move to the next word
\\
\hline\end{tabulary}



\subsubsection{Navigation}
\label{editor/joe:id1}
\begin{tabulary}{\linewidth}{|L|L|}
\hline

CTRL + u
 & 
previous screen
\\
\hline
CTRL + v
 & 
next screen
\\
\hline
CTRL + a
 & 
beginning of the line
\\
\hline
CTRL + e
 & 
end of the line
\\
\hline
CTRL + k + u
 & 
beginning of the file
\\
\hline
CTRL + k + v
 & 
end of the file
\\
\hline
CTRL + k + l
 & 
go to line n
\\
\hline\end{tabulary}



\subsection{Editing}
\label{editor/joe:editing}

\subsubsection{Blocs operations}
\label{editor/joe:blocs-operations}
\begin{tabulary}{\linewidth}{|L|L|}
\hline

CTRL + k + b
 & 
beginning of the bloc
\\
\hline
CTRL + k + k
 & 
end of the bloc
\\
\hline
CTRL + k + m
 & 
move of the bloc
\\
\hline
CTRL + k + c
 & 
copy the bloc
\\
\hline
CTRL + k + w
 & 
write the bloc in a file
\\
\hline
CTRL + k + y
 & 
delete the bloc
\\
\hline
CTRL + k + /
 & 
filter the bloc
\\
\hline\end{tabulary}



\subsubsection{Deletion}
\label{editor/joe:deletion}
\begin{tabulary}{\linewidth}{|L|L|}
\hline

CTRL + d
 & 
delete one character
\\
\hline
CTRL + y
 & 
delete one line
\\
\hline
CTRL + w
 & 
delete one word on the right of the cursor
\\
\hline
CTRL + o
 & 
delete one word on the left of the cursor
\\
\hline
CTRL + j
 & 
delete the rest of the line (i.e. the right side of the cursor)
\\
\hline
CTRL + \_
 & 
cancel the operation
\\
\hline
CTRL + 6
 & 
redo the cancelled operation
\\
\hline\end{tabulary}



\subsubsection{Files}
\label{editor/joe:files}
\begin{tabulary}{\linewidth}{|L|L|}
\hline

CTRL + k + e
 & 
open / edit a new file
\\
\hline
CTRL + k + r
 & 
insert one file at the cursor position
\\
\hline\end{tabulary}



\subsection{Search}
\label{editor/joe:search}
\begin{tabulary}{\linewidth}{|L|L|}
\hline

CTRL + k + f
 & 
search one text
\\
\hline
CTRL + l
 & 
search the next
\\
\hline\end{tabulary}



\section{NANO (Nano’s ANOther editor)}
\label{editor/nano::doc}\label{editor/nano:nano-nanos-another-editor}

\subsection{Basic commands}
\label{editor/nano:basic-commands}
NANO is the open source clone of the editor PICO, distributed as part of the mail client Pine.


\subsubsection{Launch NANO from command line}
\label{editor/nano:launch-nano-from-command-line}
\begin{Verbatim}[commandchars=\\\{\}]
\PYG{n+nv}{\PYGZdl{} }nano file.txt       open and edit file.txt
\PYG{n+nv}{\PYGZdl{} }nano \PYGZhy{}B file.txt    save the original file as file.txt\PYGZti{} or \PYGZti{}.file
\PYG{n+nv}{\PYGZdl{} }nano \PYGZhy{}m file.txt    activate the mouse cursor \PYG{o}{(}\PYG{k}{if} supported\PYG{o}{)}
\PYG{n+nv}{\PYGZdl{} }nano +25 file.txt   edit from the 25th line
\PYG{n+nv}{\PYGZdl{} }jpico file.txt      simulator JOE of PICO
\end{Verbatim}


\subsubsection{Short-cuts Fn}
\label{editor/nano:short-cuts-fn}
\begin{tabulary}{\linewidth}{|L|L|L|}
\hline

F1
 & 
CTRL + g
 & 
display online help (CTRL + x to quit)
\\
\hline
F2
 & 
CTRL + x
 & 
quit NANO (or cloase ongoing buffer)
\\
\hline
F3
 & 
CTRL + o
 & 
save ongoing file
\\
\hline
F4
 & 
CTRL + j
 & 
reformat the text of paragraph
\\
\hline
F5
 & 
CTRL + r
 & 
insert one file
\\
\hline
F6
 & 
CTRL + w
 & 
search one text
\\
\hline
F7
 & 
CTRL + y
 & 
previous screen
\\
\hline
F8
 & 
CTRL + v
 & 
next screen
\\
\hline
F9
 & 
CTRL + k
 & 
cut (and copy) the line (or the chosen text)
\\
\hline
F10
 & 
CTRL + u
 & 
paste the cut text
\\
\hline
F11
 & 
CTRL + c
 & 
display the cursor position
\\
\hline
F12
 & 
CTRL + t
 & 
start the orthograph verification
\\
\hline\end{tabulary}



\subsubsection{Misc}
\label{editor/nano:misc}
\begin{tabulary}{\linewidth}{|L|L|}
\hline

CTRL + 6
 & 
choose one text from the cursor (CTRL + 6 to cancel the action)
\\
\hline\end{tabulary}



\chapter{Programming language}
\label{index:programming-language}

\section{Shell}
\label{language/shell:shell}\label{language/shell::doc}

\section{Python}
\label{language/python:python}\label{language/python::doc}

\chapter{Mathmatics}
\label{index:mathmatics}

\section{Algebra}
\label{math/algebra::doc}\label{math/algebra:algebra}

\section{Geometry}
\label{math/geometry:geometry}\label{math/geometry::doc}


\renewcommand{\indexname}{Index}
\printindex
\end{document}
