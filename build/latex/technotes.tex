% Generated by Sphinx.
\def\sphinxdocclass{report}
\documentclass[letterpaper,10pt,english]{sphinxmanual}
\usepackage[utf8]{inputenc}
\DeclareUnicodeCharacter{00A0}{\nobreakspace}
\usepackage{cmap}
\usepackage[T1]{fontenc}
\usepackage{babel}
\usepackage{times}
\usepackage[Bjarne]{fncychap}
\usepackage{longtable}
\usepackage{sphinx}
\usepackage{multirow}

\addto\captionsenglish{\renewcommand{\figurename}{Fig. }}
\addto\captionsenglish{\renewcommand{\tablename}{Table }}
\floatname{literal-block}{Listing }



\title{Tech for fun!}
\date{April 24, 2015}
\release{1}
\author{Tom}
\newcommand{\sphinxlogo}{}
\renewcommand{\releasename}{Release}
\makeindex

\makeatletter
\def\PYG@reset{\let\PYG@it=\relax \let\PYG@bf=\relax%
    \let\PYG@ul=\relax \let\PYG@tc=\relax%
    \let\PYG@bc=\relax \let\PYG@ff=\relax}
\def\PYG@tok#1{\csname PYG@tok@#1\endcsname}
\def\PYG@toks#1+{\ifx\relax#1\empty\else%
    \PYG@tok{#1}\expandafter\PYG@toks\fi}
\def\PYG@do#1{\PYG@bc{\PYG@tc{\PYG@ul{%
    \PYG@it{\PYG@bf{\PYG@ff{#1}}}}}}}
\def\PYG#1#2{\PYG@reset\PYG@toks#1+\relax+\PYG@do{#2}}

\expandafter\def\csname PYG@tok@gd\endcsname{\def\PYG@tc##1{\textcolor[rgb]{0.63,0.00,0.00}{##1}}}
\expandafter\def\csname PYG@tok@gu\endcsname{\let\PYG@bf=\textbf\def\PYG@tc##1{\textcolor[rgb]{0.50,0.00,0.50}{##1}}}
\expandafter\def\csname PYG@tok@gt\endcsname{\def\PYG@tc##1{\textcolor[rgb]{0.00,0.27,0.87}{##1}}}
\expandafter\def\csname PYG@tok@gs\endcsname{\let\PYG@bf=\textbf}
\expandafter\def\csname PYG@tok@gr\endcsname{\def\PYG@tc##1{\textcolor[rgb]{1.00,0.00,0.00}{##1}}}
\expandafter\def\csname PYG@tok@cm\endcsname{\let\PYG@it=\textit\def\PYG@tc##1{\textcolor[rgb]{0.25,0.50,0.56}{##1}}}
\expandafter\def\csname PYG@tok@vg\endcsname{\def\PYG@tc##1{\textcolor[rgb]{0.73,0.38,0.84}{##1}}}
\expandafter\def\csname PYG@tok@m\endcsname{\def\PYG@tc##1{\textcolor[rgb]{0.13,0.50,0.31}{##1}}}
\expandafter\def\csname PYG@tok@mh\endcsname{\def\PYG@tc##1{\textcolor[rgb]{0.13,0.50,0.31}{##1}}}
\expandafter\def\csname PYG@tok@cs\endcsname{\def\PYG@tc##1{\textcolor[rgb]{0.25,0.50,0.56}{##1}}\def\PYG@bc##1{\setlength{\fboxsep}{0pt}\colorbox[rgb]{1.00,0.94,0.94}{\strut ##1}}}
\expandafter\def\csname PYG@tok@ge\endcsname{\let\PYG@it=\textit}
\expandafter\def\csname PYG@tok@vc\endcsname{\def\PYG@tc##1{\textcolor[rgb]{0.73,0.38,0.84}{##1}}}
\expandafter\def\csname PYG@tok@il\endcsname{\def\PYG@tc##1{\textcolor[rgb]{0.13,0.50,0.31}{##1}}}
\expandafter\def\csname PYG@tok@go\endcsname{\def\PYG@tc##1{\textcolor[rgb]{0.20,0.20,0.20}{##1}}}
\expandafter\def\csname PYG@tok@cp\endcsname{\def\PYG@tc##1{\textcolor[rgb]{0.00,0.44,0.13}{##1}}}
\expandafter\def\csname PYG@tok@gi\endcsname{\def\PYG@tc##1{\textcolor[rgb]{0.00,0.63,0.00}{##1}}}
\expandafter\def\csname PYG@tok@gh\endcsname{\let\PYG@bf=\textbf\def\PYG@tc##1{\textcolor[rgb]{0.00,0.00,0.50}{##1}}}
\expandafter\def\csname PYG@tok@ni\endcsname{\let\PYG@bf=\textbf\def\PYG@tc##1{\textcolor[rgb]{0.84,0.33,0.22}{##1}}}
\expandafter\def\csname PYG@tok@nl\endcsname{\let\PYG@bf=\textbf\def\PYG@tc##1{\textcolor[rgb]{0.00,0.13,0.44}{##1}}}
\expandafter\def\csname PYG@tok@nn\endcsname{\let\PYG@bf=\textbf\def\PYG@tc##1{\textcolor[rgb]{0.05,0.52,0.71}{##1}}}
\expandafter\def\csname PYG@tok@no\endcsname{\def\PYG@tc##1{\textcolor[rgb]{0.38,0.68,0.84}{##1}}}
\expandafter\def\csname PYG@tok@na\endcsname{\def\PYG@tc##1{\textcolor[rgb]{0.25,0.44,0.63}{##1}}}
\expandafter\def\csname PYG@tok@nb\endcsname{\def\PYG@tc##1{\textcolor[rgb]{0.00,0.44,0.13}{##1}}}
\expandafter\def\csname PYG@tok@nc\endcsname{\let\PYG@bf=\textbf\def\PYG@tc##1{\textcolor[rgb]{0.05,0.52,0.71}{##1}}}
\expandafter\def\csname PYG@tok@nd\endcsname{\let\PYG@bf=\textbf\def\PYG@tc##1{\textcolor[rgb]{0.33,0.33,0.33}{##1}}}
\expandafter\def\csname PYG@tok@ne\endcsname{\def\PYG@tc##1{\textcolor[rgb]{0.00,0.44,0.13}{##1}}}
\expandafter\def\csname PYG@tok@nf\endcsname{\def\PYG@tc##1{\textcolor[rgb]{0.02,0.16,0.49}{##1}}}
\expandafter\def\csname PYG@tok@si\endcsname{\let\PYG@it=\textit\def\PYG@tc##1{\textcolor[rgb]{0.44,0.63,0.82}{##1}}}
\expandafter\def\csname PYG@tok@s2\endcsname{\def\PYG@tc##1{\textcolor[rgb]{0.25,0.44,0.63}{##1}}}
\expandafter\def\csname PYG@tok@vi\endcsname{\def\PYG@tc##1{\textcolor[rgb]{0.73,0.38,0.84}{##1}}}
\expandafter\def\csname PYG@tok@nt\endcsname{\let\PYG@bf=\textbf\def\PYG@tc##1{\textcolor[rgb]{0.02,0.16,0.45}{##1}}}
\expandafter\def\csname PYG@tok@nv\endcsname{\def\PYG@tc##1{\textcolor[rgb]{0.73,0.38,0.84}{##1}}}
\expandafter\def\csname PYG@tok@s1\endcsname{\def\PYG@tc##1{\textcolor[rgb]{0.25,0.44,0.63}{##1}}}
\expandafter\def\csname PYG@tok@gp\endcsname{\let\PYG@bf=\textbf\def\PYG@tc##1{\textcolor[rgb]{0.78,0.36,0.04}{##1}}}
\expandafter\def\csname PYG@tok@sh\endcsname{\def\PYG@tc##1{\textcolor[rgb]{0.25,0.44,0.63}{##1}}}
\expandafter\def\csname PYG@tok@ow\endcsname{\let\PYG@bf=\textbf\def\PYG@tc##1{\textcolor[rgb]{0.00,0.44,0.13}{##1}}}
\expandafter\def\csname PYG@tok@sx\endcsname{\def\PYG@tc##1{\textcolor[rgb]{0.78,0.36,0.04}{##1}}}
\expandafter\def\csname PYG@tok@bp\endcsname{\def\PYG@tc##1{\textcolor[rgb]{0.00,0.44,0.13}{##1}}}
\expandafter\def\csname PYG@tok@c1\endcsname{\let\PYG@it=\textit\def\PYG@tc##1{\textcolor[rgb]{0.25,0.50,0.56}{##1}}}
\expandafter\def\csname PYG@tok@kc\endcsname{\let\PYG@bf=\textbf\def\PYG@tc##1{\textcolor[rgb]{0.00,0.44,0.13}{##1}}}
\expandafter\def\csname PYG@tok@c\endcsname{\let\PYG@it=\textit\def\PYG@tc##1{\textcolor[rgb]{0.25,0.50,0.56}{##1}}}
\expandafter\def\csname PYG@tok@mf\endcsname{\def\PYG@tc##1{\textcolor[rgb]{0.13,0.50,0.31}{##1}}}
\expandafter\def\csname PYG@tok@err\endcsname{\def\PYG@bc##1{\setlength{\fboxsep}{0pt}\fcolorbox[rgb]{1.00,0.00,0.00}{1,1,1}{\strut ##1}}}
\expandafter\def\csname PYG@tok@mb\endcsname{\def\PYG@tc##1{\textcolor[rgb]{0.13,0.50,0.31}{##1}}}
\expandafter\def\csname PYG@tok@ss\endcsname{\def\PYG@tc##1{\textcolor[rgb]{0.32,0.47,0.09}{##1}}}
\expandafter\def\csname PYG@tok@sr\endcsname{\def\PYG@tc##1{\textcolor[rgb]{0.14,0.33,0.53}{##1}}}
\expandafter\def\csname PYG@tok@mo\endcsname{\def\PYG@tc##1{\textcolor[rgb]{0.13,0.50,0.31}{##1}}}
\expandafter\def\csname PYG@tok@kd\endcsname{\let\PYG@bf=\textbf\def\PYG@tc##1{\textcolor[rgb]{0.00,0.44,0.13}{##1}}}
\expandafter\def\csname PYG@tok@mi\endcsname{\def\PYG@tc##1{\textcolor[rgb]{0.13,0.50,0.31}{##1}}}
\expandafter\def\csname PYG@tok@kn\endcsname{\let\PYG@bf=\textbf\def\PYG@tc##1{\textcolor[rgb]{0.00,0.44,0.13}{##1}}}
\expandafter\def\csname PYG@tok@o\endcsname{\def\PYG@tc##1{\textcolor[rgb]{0.40,0.40,0.40}{##1}}}
\expandafter\def\csname PYG@tok@kr\endcsname{\let\PYG@bf=\textbf\def\PYG@tc##1{\textcolor[rgb]{0.00,0.44,0.13}{##1}}}
\expandafter\def\csname PYG@tok@s\endcsname{\def\PYG@tc##1{\textcolor[rgb]{0.25,0.44,0.63}{##1}}}
\expandafter\def\csname PYG@tok@kp\endcsname{\def\PYG@tc##1{\textcolor[rgb]{0.00,0.44,0.13}{##1}}}
\expandafter\def\csname PYG@tok@w\endcsname{\def\PYG@tc##1{\textcolor[rgb]{0.73,0.73,0.73}{##1}}}
\expandafter\def\csname PYG@tok@kt\endcsname{\def\PYG@tc##1{\textcolor[rgb]{0.56,0.13,0.00}{##1}}}
\expandafter\def\csname PYG@tok@sc\endcsname{\def\PYG@tc##1{\textcolor[rgb]{0.25,0.44,0.63}{##1}}}
\expandafter\def\csname PYG@tok@sb\endcsname{\def\PYG@tc##1{\textcolor[rgb]{0.25,0.44,0.63}{##1}}}
\expandafter\def\csname PYG@tok@k\endcsname{\let\PYG@bf=\textbf\def\PYG@tc##1{\textcolor[rgb]{0.00,0.44,0.13}{##1}}}
\expandafter\def\csname PYG@tok@se\endcsname{\let\PYG@bf=\textbf\def\PYG@tc##1{\textcolor[rgb]{0.25,0.44,0.63}{##1}}}
\expandafter\def\csname PYG@tok@sd\endcsname{\let\PYG@it=\textit\def\PYG@tc##1{\textcolor[rgb]{0.25,0.44,0.63}{##1}}}

\def\PYGZbs{\char`\\}
\def\PYGZus{\char`\_}
\def\PYGZob{\char`\{}
\def\PYGZcb{\char`\}}
\def\PYGZca{\char`\^}
\def\PYGZam{\char`\&}
\def\PYGZlt{\char`\<}
\def\PYGZgt{\char`\>}
\def\PYGZsh{\char`\#}
\def\PYGZpc{\char`\%}
\def\PYGZdl{\char`\$}
\def\PYGZhy{\char`\-}
\def\PYGZsq{\char`\'}
\def\PYGZdq{\char`\"}
\def\PYGZti{\char`\~}
% for compatibility with earlier versions
\def\PYGZat{@}
\def\PYGZlb{[}
\def\PYGZrb{]}
\makeatother

\renewcommand\PYGZsq{\textquotesingle}

\begin{document}

\maketitle
\tableofcontents
\phantomsection\label{index::doc}



\chapter{Kid's activity}
\label{index:kid-s-activity}\label{index:tech-for-fun}

\section{Minecraft Pi Edition}
\label{kid/minecraft::doc}\label{kid/minecraft:minecraft-pi-edition}

\subsection{Basic commands}
\label{kid/minecraft:basic-commands}
\begin{tabulary}{\linewidth}{|L|L|}
\hline

\textbf{W}
 & 
move forward
\\
\hline
\textbf{S}
 & 
move backward
\\
\hline
\textbf{A}
 & 
move left
\\
\hline
\textbf{D}
 & 
move right
\\
\hline
\textbf{E}
 & 
show inventory of blocks
\\
\hline
\textbf{1}-\textbf{8}
 & 
select items in the quick bar
\\
\hline
\textbf{Space} / \textbf{Ctrl} + \textbf{Space}
 & 
jump (ascend in fly-mode)
\\
\hline
\textbf{Shift} / \textbf{Ctrl} + \textbf{Shift}
 & 
sneak (descend in fly-mode)
\\
\hline
\textbf{ESC}
 & 
pause / menu
\\
\hline
left mouse
 & 
destroy blocks
\\
\hline
right mouse
 & 
place blocks
\\
\hline
double \textbf{Space}
 & 
fly / fall
\\
\hline
\textbf{Tab}
 & 
release mouse
\\
\hline\end{tabulary}



\subsection{Available blocks in Minecraft Pi Edition}
\label{kid/minecraft:available-blocks-in-minecraft-pi-edition}
\begin{longtable}{|l|l|}
\hline
\endfirsthead

\multicolumn{2}{c}%
{{\textsf{\tablename\ \thetable{} -- continued from previous page}}} \\
\hline
\endhead

\hline \multicolumn{2}{|r|}{{\textsf{Continued on next page}}} \\ \hline
\endfoot

\endlastfoot


AIR
 & 
Block(0)
\\
\hline
STONE
 & 
Block(1)
\\
\hline
GRASS
 & 
Block(2)
\\
\hline
DIRT
 & 
Block(3)
\\
\hline
COBBLESTONE
 & 
Block(4)
\\
\hline
WOOD\_PLANKS
 & 
Block(5)
\\
\hline
SAPLING
 & 
Block(6)
\\
\hline
BEDROCK
 & 
Block(7)
\\
\hline
WATER\_FLOWING
 & 
Block(8)
\\
\hline
WATER
 & 
WATER\_FLOWING
\\
\hline
WATER\_STATIONARY
 & 
Block(9)
\\
\hline
LAVA\_FLOWING
 & 
Block(10)
\\
\hline
LAVA
 & 
LAVA\_FLOWING
\\
\hline
LAVA\_STATIONARY
 & 
Block(11)
\\
\hline
SAND
 & 
Block(12)
\\
\hline
GRAVEL
 & 
Block(13)
\\
\hline
GOLD\_ORE
 & 
Block(14)
\\
\hline
IRON\_ORE
 & 
Block(15)
\\
\hline
COAL\_ORE
 & 
Block(16)
\\
\hline
WOOD
 & 
Block(17)
\\
\hline
LEAVES
 & 
Block(18)
\\
\hline
GLASS
 & 
Block(20)
\\
\hline
LAPIS\_LAZULI\_ORE
 & 
Block(21)
\\
\hline
LAPIS\_LAZULI\_BLOCK
 & 
Block(22)
\\
\hline
SANDSTONE
 & 
Block(24)
\\
\hline
BED
 & 
Block(26)
\\
\hline
COBWEB
 & 
Block(30)
\\
\hline
GRASS\_TALL
 & 
Block(31)
\\
\hline
WOOL
 & 
Block(35)
\\
\hline
FLOWER\_YELLOW
 & 
Block(37)
\\
\hline
FLOWER\_CYAN
 & 
Block(38)
\\
\hline
MUSHROOM\_BROWN
 & 
Block(39)
\\
\hline
MUSHROOM\_RED
 & 
Block(40)
\\
\hline
GOLD\_BLOCK
 & 
Block(41)
\\
\hline
IRON\_BLOCK
 & 
Block(42)
\\
\hline
STONE\_SLAB\_DOUBLE
 & 
Block(43)
\\
\hline
STONE\_SLAB
 & 
Block(44)
\\
\hline
BRICK\_BLOCK
 & 
Block(45)
\\
\hline
TNT
 & 
Block(46)
\\
\hline
BOOKSHELF
 & 
Block(47)
\\
\hline
MOSS\_STONE
 & 
Block(48)
\\
\hline
OBSIDIAN
 & 
Block(49)
\\
\hline
TORCH
 & 
Block(50)
\\
\hline
FIRE
 & 
Block(51)
\\
\hline
STAIRS\_WOOD
 & 
Block(53)
\\
\hline
CHEST
 & 
Block(54)
\\
\hline
DIAMOND\_ORE
 & 
Block(56)
\\
\hline
DIAMOND\_BLOCK
 & 
Block(57)
\\
\hline
CRAFTING\_TABLE
 & 
Block(58)
\\
\hline
FARMLAND
 & 
Block(60)
\\
\hline
FURNACE\_INACTIVE
 & 
Block(61)
\\
\hline
FURNACE\_ACTIVE
 & 
Block(62)
\\
\hline
DOOR\_WOOD
 & 
Block(64)
\\
\hline
LADDER
 & 
Block(65)
\\
\hline
STAIRS\_COBBLESTONE
 & 
Block(67)
\\
\hline
DOOR\_IRON
 & 
Block(71)
\\
\hline
REDSTONE\_ORE
 & 
Block(73)
\\
\hline
SNOW
 & 
Block(78)
\\
\hline
ICE
 & 
Block(79)
\\
\hline
SNOW\_BLOCK
 & 
Block(80)
\\
\hline
CACTUS
 & 
Block(81)
\\
\hline
CLAY
 & 
Block(82)
\\
\hline
SUGAR\_CANE
 & 
Block(83)
\\
\hline
FENCE
 & 
Block(85)
\\
\hline
GLOWSTONE\_BLOCK
 & 
Block(89)
\\
\hline
BEDROCK\_INVISIBLE
 & 
Block(95)
\\
\hline
STONE\_BRICK
 & 
Block(98)
\\
\hline
GLASS\_PANE
 & 
Block(102)
\\
\hline
MELON
 & 
Block(103)
\\
\hline
FENCE\_GATE
 & 
Block(107)
\\
\hline
GLOWING\_OBSIDIAN
 & 
Block(246)
\\
\hline
NETHER\_REACTOR\_CORE
 & 
Block(247)
\\
\hline\end{longtable}



\subsection{List of python programs}
\label{kid/minecraft:list-of-python-programs}

\subsubsection{Short-cuts}
\label{kid/minecraft:short-cuts}
\begin{tabulary}{\linewidth}{|L|L|}
\hline

\textbf{Ctrl} + \textbf{S}
 & 
save
\\
\hline
\textbf{F5}
 & 
run
\\
\hline\end{tabulary}



\subsubsection{Display the player's position}
\label{kid/minecraft:display-the-player-s-position}
\begin{Verbatim}[commandchars=\\\{\},numbers=left,firstnumber=1,stepnumber=1]
\PYG{k+kn}{from} \PYG{n+nn}{mcpi} \PYG{k+kn}{import} \PYG{n}{minecraft}

\PYG{n}{mc} \PYG{o}{=} \PYG{n}{minecraft}\PYG{o}{.}\PYG{n}{Minecraft}\PYG{o}{.}\PYG{n}{create}\PYG{p}{(}\PYG{p}{)}

\PYG{n}{x}\PYG{p}{,}\PYG{n}{y}\PYG{p}{,}\PYG{n}{z} \PYG{o}{=} \PYG{n}{mc}\PYG{o}{.}\PYG{n}{player}\PYG{o}{.}\PYG{n}{getTilePos}\PYG{p}{(}\PYG{p}{)}
\PYG{n}{mc}\PYG{o}{.}\PYG{n}{postToChat}\PYG{p}{(}\PYG{l+s}{\PYGZdq{}}\PYG{l+s}{x=}\PYG{l+s}{\PYGZdq{}}\PYG{o}{+}\PYG{n+nb}{str}\PYG{p}{(}\PYG{n}{x}\PYG{p}{)}\PYG{o}{+}\PYG{l+s}{\PYGZdq{}}\PYG{l+s}{, y=}\PYG{l+s}{\PYGZdq{}}\PYG{o}{+}\PYG{n+nb}{str}\PYG{p}{(}\PYG{n}{y}\PYG{p}{)}\PYG{o}{+}\PYG{l+s}{\PYGZdq{}}\PYG{l+s}{, z=}\PYG{l+s}{\PYGZdq{}}\PYG{o}{+}\PYG{n+nb}{str}\PYG{p}{(}\PYG{n}{z}\PYG{p}{)}\PYG{p}{)}
\end{Verbatim}


\subsubsection{Teleport (change the player's position)}
\label{kid/minecraft:teleport-change-the-player-s-position}
In the following program, the player will be teleported 100 higher.

\begin{Verbatim}[commandchars=\\\{\},numbers=left,firstnumber=1,stepnumber=1]
\PYG{k+kn}{from} \PYG{n+nn}{mcpi} \PYG{k+kn}{import} \PYG{n}{minecraft}

\PYG{n}{mc} \PYG{o}{=} \PYG{n}{minecraft}\PYG{o}{.}\PYG{n}{Minecraft}\PYG{o}{.}\PYG{n}{create}\PYG{p}{(}\PYG{p}{)}

\PYG{n}{x}\PYG{p}{,}\PYG{n}{y}\PYG{p}{,}\PYG{n}{z} \PYG{o}{=} \PYG{n}{mc}\PYG{o}{.}\PYG{n}{player}\PYG{o}{.}\PYG{n}{getTilePos}\PYG{p}{(}\PYG{p}{)}
\PYG{n}{mc}\PYG{o}{.}\PYG{n}{player}\PYG{o}{.}\PYG{n}{setPos}\PYG{p}{(}\PYG{n}{x}\PYG{p}{,}\PYG{n}{y}\PYG{o}{+}\PYG{l+m+mi}{100}\PYG{p}{,}\PYG{n}{z}\PYG{p}{)}
\end{Verbatim}


\subsubsection{Build a huge block of activated TNTs}
\label{kid/minecraft:build-a-huge-block-of-activated-tnts}
When you click one TNT, there will be an explosion around that block of TNTs.

\begin{Verbatim}[commandchars=\\\{\},numbers=left,firstnumber=1,stepnumber=1]
\PYG{k+kn}{from} \PYG{n+nn}{mcpi} \PYG{k+kn}{import} \PYG{n}{minecraft}

\PYG{n}{mc} \PYG{o}{=} \PYG{n}{minecraft}\PYG{o}{.}\PYG{n}{Minecraft}\PYG{o}{.}\PYG{n}{create}\PYG{p}{(}\PYG{p}{)}

\PYG{n}{x}\PYG{p}{,}\PYG{n}{y}\PYG{p}{,}\PYG{n}{z} \PYG{o}{=} \PYG{n}{mc}\PYG{o}{.}\PYG{n}{player}\PYG{o}{.}\PYG{n}{getTilePos}\PYG{p}{(}\PYG{p}{)}

\PYG{n}{tnt} \PYG{o}{=} \PYG{l+m+mi}{46}
\PYG{n}{activated} \PYG{o}{=} \PYG{l+m+mi}{1}
\PYG{n}{mc}\PYG{o}{.}\PYG{n}{setBlocks}\PYG{p}{(}\PYG{n}{x}\PYG{o}{+}\PYG{l+m+mi}{1}\PYG{p}{,}\PYG{n}{y}\PYG{o}{+}\PYG{l+m+mi}{1}\PYG{p}{,}\PYG{n}{z}\PYG{o}{+}\PYG{l+m+mi}{1}\PYG{p}{,}\PYG{n}{x}\PYG{o}{+}\PYG{l+m+mi}{5}\PYG{p}{,}\PYG{n}{y}\PYG{o}{+}\PYG{l+m+mi}{5}\PYG{p}{,}\PYG{n}{z}\PYG{o}{+}\PYG{l+m+mi}{5}\PYG{p}{,}\PYG{n}{tnt}\PYG{p}{,}\PYG{n}{activated}\PYG{p}{)}
\end{Verbatim}


\subsubsection{Put a flower on the path}
\label{kid/minecraft:put-a-flower-on-the-path}
We will leave a flower when we are on a block of grass. Otherwise we will change the beneath block to a grass block.

\begin{Verbatim}[commandchars=\\\{\},numbers=left,firstnumber=1,stepnumber=1]
\PYG{k+kn}{from} \PYG{n+nn}{mcpi} \PYG{k+kn}{import} \PYG{n}{minecraft}
\PYG{k+kn}{from} \PYG{n+nn}{time} \PYG{k+kn}{import} \PYG{n}{sleep}

\PYG{n}{mc} \PYG{o}{=} \PYG{n}{minecraft}\PYG{o}{.}\PYG{n}{Minecraft}\PYG{o}{.}\PYG{n}{create}\PYG{p}{(}\PYG{p}{)}

\PYG{n}{grass} \PYG{o}{=} \PYG{l+m+mi}{2}
\PYG{n}{flower} \PYG{o}{=} \PYG{l+m+mi}{38}
\PYG{k}{while} \PYG{n+nb+bp}{True}\PYG{p}{:}
    \PYG{n}{x}\PYG{p}{,}\PYG{n}{y}\PYG{p}{,}\PYG{n}{z} \PYG{o}{=} \PYG{n}{mc}\PYG{o}{.}\PYG{n}{player}\PYG{o}{.}\PYG{n}{getTilePos}\PYG{p}{(}\PYG{p}{)}
    \PYG{n}{block\PYGZus{}beneath} \PYG{o}{=} \PYG{n}{mc}\PYG{o}{.}\PYG{n}{getBlock}\PYG{p}{(}\PYG{n}{x}\PYG{p}{,}\PYG{n}{y}\PYG{o}{\PYGZhy{}}\PYG{l+m+mi}{1}\PYG{p}{,}\PYG{n}{z}\PYG{p}{)}
    \PYG{k}{if} \PYG{n}{block\PYGZus{}beneath} \PYG{o}{==} \PYG{n}{grass}\PYG{p}{:}
        \PYG{n}{mc}\PYG{o}{.}\PYG{n}{setBlock}\PYG{p}{(}\PYG{n}{x}\PYG{p}{,}\PYG{n}{y}\PYG{p}{,}\PYG{n}{z}\PYG{p}{,}\PYG{n}{flower}\PYG{p}{)}
    \PYG{k}{else}\PYG{p}{:}
        \PYG{n}{mc}\PYG{o}{.}\PYG{n}{setBlock}\PYG{p}{(}\PYG{n}{x}\PYG{p}{,}\PYG{n}{y}\PYG{o}{\PYGZhy{}}\PYG{l+m+mi}{1}\PYG{p}{,}\PYG{n}{z}\PYG{p}{,}\PYG{n}{grass}\PYG{p}{)}
    \PYG{n}{sleep}\PYG{p}{(}\PYG{l+m+mf}{0.1}\PYG{p}{)}
\end{Verbatim}


\subsubsection{Clear space with input size}
\label{kid/minecraft:clear-space-with-input-size}
We will clear space for a given \textbf{size}. To do so, we will build a cube of \textbf{size} x \textbf{size} x \textbf{size} blocks, filled with the AIR block.

\begin{Verbatim}[commandchars=\\\{\},numbers=left,firstnumber=1,stepnumber=1]
\PYG{k+kn}{from} \PYG{n+nn}{mcpi} \PYG{k+kn}{import} \PYG{n}{minecraft}\PYG{p}{,} \PYG{n}{block}

\PYG{n}{mc} \PYG{o}{=} \PYG{n}{minecraft}\PYG{o}{.}\PYG{n}{Minecraft}\PYG{o}{.}\PYG{n}{create}\PYG{p}{(}\PYG{p}{)}

\PYG{n}{x}\PYG{p}{,}\PYG{n}{y}\PYG{p}{,}\PYG{n}{z} \PYG{o}{=} \PYG{n}{mc}\PYG{o}{.}\PYG{n}{player}\PYG{o}{.}\PYG{n}{getTilePos}\PYG{p}{(}\PYG{p}{)}
\PYG{n}{size} \PYG{o}{=} \PYG{n+nb}{int}\PYG{p}{(}\PYG{n+nb}{raw\PYGZus{}input}\PYG{p}{(}\PYG{l+s}{\PYGZdq{}}\PYG{l+s}{size of area to clear? }\PYG{l+s}{\PYGZdq{}}\PYG{p}{)}\PYG{p}{)}
\PYG{k}{if} \PYG{n}{size} \PYG{o}{\PYGZgt{}} \PYG{l+m+mi}{0}\PYG{p}{:}
    \PYG{n}{mc}\PYG{o}{.}\PYG{n}{setBlocks}\PYG{p}{(}\PYG{n}{x}\PYG{p}{,}\PYG{n}{y}\PYG{p}{,}\PYG{n}{z}\PYG{p}{,}\PYG{n}{x}\PYG{o}{+}\PYG{n}{size}\PYG{p}{,}\PYG{n}{y}\PYG{o}{+}\PYG{n}{size}\PYG{p}{,}\PYG{n}{z}\PYG{o}{+}\PYG{n}{size}\PYG{p}{,}\PYG{n}{block}\PYG{o}{.}\PYG{n}{AIR}\PYG{o}{.}\PYG{n}{id}\PYG{p}{)}
\end{Verbatim}
\setbox0\vbox{
\begin{minipage}{0.95\linewidth}
\textbf{Challenge}

\medskip


Change a little the above program so that the player is in the middle of the cleared space (and also dig down a few blocks).
\end{minipage}}
\begin{center}\setlength{\fboxsep}{5pt}\shadowbox{\box0}\end{center}


\subsubsection{Build a house}
\label{kid/minecraft:build-a-house}
\begin{Verbatim}[commandchars=\\\{\},numbers=left,firstnumber=1,stepnumber=1]
\PYG{k+kn}{from} \PYG{n+nn}{mcpi} \PYG{k+kn}{import} \PYG{n}{minecraft}\PYG{p}{,} \PYG{n}{block}

\PYG{n}{mc} \PYG{o}{=} \PYG{n}{minecraft}\PYG{o}{.}\PYG{n}{Minecraft}\PYG{o}{.}\PYG{n}{create}\PYG{p}{(}\PYG{p}{)}
\PYG{n}{SIZE} \PYG{o}{=} \PYG{l+m+mi}{20}

\PYG{k}{def} \PYG{n+nf}{house}\PYG{p}{(}\PYG{p}{)}\PYG{p}{:}
    \PYG{n}{midx} \PYG{o}{=} \PYG{n}{x} \PYG{o}{+} \PYG{n}{SIZE}\PYG{o}{/}\PYG{l+m+mi}{2}
    \PYG{n}{midy} \PYG{o}{=} \PYG{n}{y} \PYG{o}{+} \PYG{n}{SIZE}\PYG{o}{/}\PYG{l+m+mi}{2}
    \PYG{n}{mc}\PYG{o}{.}\PYG{n}{setBlocks}\PYG{p}{(}     \PYG{n}{x}\PYG{p}{,}       \PYG{n}{y}\PYG{p}{,}  \PYG{n}{z}\PYG{p}{,}  \PYG{n}{x}\PYG{o}{+}\PYG{n}{SIZE}\PYG{p}{,}  \PYG{n}{y}\PYG{o}{+}\PYG{n}{SIZE}\PYG{p}{,}  \PYG{n}{z}\PYG{o}{+}\PYG{n}{SIZE}\PYG{p}{,}\PYG{n}{block}\PYG{o}{.}\PYG{n}{COBBLESTONE}\PYG{o}{.}\PYG{n}{id}\PYG{p}{)}
    \PYG{n}{mc}\PYG{o}{.}\PYG{n}{setBlocks}\PYG{p}{(}   \PYG{n}{x}\PYG{o}{+}\PYG{l+m+mi}{1}\PYG{p}{,}     \PYG{n}{y}\PYG{o}{+}\PYG{l+m+mi}{1}\PYG{p}{,}\PYG{n}{z}\PYG{o}{+}\PYG{l+m+mi}{1}\PYG{p}{,}\PYG{n}{x}\PYG{o}{+}\PYG{n}{SIZE}\PYG{o}{\PYGZhy{}}\PYG{l+m+mi}{1}\PYG{p}{,}\PYG{n}{y}\PYG{o}{+}\PYG{n}{SIZE}\PYG{o}{\PYGZhy{}}\PYG{l+m+mi}{1}\PYG{p}{,}\PYG{n}{z}\PYG{o}{+}\PYG{n}{SIZE}\PYG{o}{\PYGZhy{}}\PYG{l+m+mi}{1}\PYG{p}{,}        \PYG{n}{block}\PYG{o}{.}\PYG{n}{AIR}\PYG{o}{.}\PYG{n}{id}\PYG{p}{)}
    \PYG{n}{mc}\PYG{o}{.}\PYG{n}{setBlocks}\PYG{p}{(}   \PYG{n}{x}\PYG{o}{+}\PYG{l+m+mi}{1}\PYG{p}{,}     \PYG{n}{y}\PYG{o}{+}\PYG{l+m+mi}{1}\PYG{p}{,}\PYG{n}{z}\PYG{o}{+}\PYG{l+m+mi}{1}\PYG{p}{,}\PYG{n}{x}\PYG{o}{+}\PYG{n}{SIZE}\PYG{o}{\PYGZhy{}}\PYG{l+m+mi}{1}\PYG{p}{,}     \PYG{n}{y}\PYG{o}{+}\PYG{l+m+mi}{1}\PYG{p}{,}\PYG{n}{z}\PYG{o}{+}\PYG{n}{SIZE}\PYG{o}{\PYGZhy{}}\PYG{l+m+mi}{1}\PYG{p}{,}       \PYG{n}{block}\PYG{o}{.}\PYG{n}{WOOL}\PYG{o}{.}\PYG{n}{id}\PYG{p}{,}\PYG{l+m+mi}{7}\PYG{p}{)}
    \PYG{c}{\PYGZsh{} left window}
    \PYG{n}{mc}\PYG{o}{.}\PYG{n}{setBlocks}\PYG{p}{(}   \PYG{n}{x}\PYG{o}{+}\PYG{l+m+mi}{3}\PYG{p}{,}\PYG{n}{y}\PYG{o}{+}\PYG{n}{SIZE}\PYG{o}{\PYGZhy{}}\PYG{l+m+mi}{3}\PYG{p}{,}  \PYG{n}{z}\PYG{p}{,}  \PYG{n}{midx}\PYG{o}{\PYGZhy{}}\PYG{l+m+mi}{3}\PYG{p}{,}  \PYG{n}{midy}\PYG{o}{+}\PYG{l+m+mi}{3}\PYG{p}{,}       \PYG{n}{z}\PYG{p}{,}      \PYG{n}{block}\PYG{o}{.}\PYG{n}{GLASS}\PYG{o}{.}\PYG{n}{id}\PYG{p}{)}
    \PYG{c}{\PYGZsh{} right window}
    \PYG{n}{mc}\PYG{o}{.}\PYG{n}{setBlocks}\PYG{p}{(}\PYG{n}{midx}\PYG{o}{+}\PYG{l+m+mi}{3}\PYG{p}{,}\PYG{n}{y}\PYG{o}{+}\PYG{n}{SIZE}\PYG{o}{\PYGZhy{}}\PYG{l+m+mi}{3}\PYG{p}{,}  \PYG{n}{z}\PYG{p}{,}\PYG{n}{x}\PYG{o}{+}\PYG{n}{SIZE}\PYG{o}{\PYGZhy{}}\PYG{l+m+mi}{3}\PYG{p}{,}  \PYG{n}{midy}\PYG{o}{+}\PYG{l+m+mi}{3}\PYG{p}{,}       \PYG{n}{z}\PYG{p}{,}      \PYG{n}{block}\PYG{o}{.}\PYG{n}{GLASS}\PYG{o}{.}\PYG{n}{id}\PYG{p}{)}
    \PYG{c}{\PYGZsh{} door}
    \PYG{n}{mc}\PYG{o}{.}\PYG{n}{setBlocks}\PYG{p}{(}\PYG{n}{midx}\PYG{o}{\PYGZhy{}}\PYG{l+m+mi}{3}\PYG{p}{,}       \PYG{n}{y}\PYG{p}{,}  \PYG{n}{z}\PYG{p}{,}  \PYG{n}{midx}\PYG{o}{+}\PYG{l+m+mi}{3}\PYG{p}{,}    \PYG{n}{midy}\PYG{p}{,}       \PYG{n}{z}\PYG{p}{,}        \PYG{n}{block}\PYG{o}{.}\PYG{n}{AIR}\PYG{o}{.}\PYG{n}{id}\PYG{p}{)}
    \PYG{c}{\PYGZsh{} roof}
    \PYG{n}{mc}\PYG{o}{.}\PYG{n}{setBlocks}\PYG{p}{(}     \PYG{n}{x}\PYG{p}{,}\PYG{n}{y}\PYG{o}{+}\PYG{n}{SIZE}\PYG{o}{+}\PYG{l+m+mi}{1}\PYG{p}{,}  \PYG{n}{z}\PYG{p}{,}  \PYG{n}{x}\PYG{o}{+}\PYG{n}{SIZE}\PYG{p}{,}\PYG{n}{y}\PYG{o}{+}\PYG{n}{SIZE}\PYG{o}{+}\PYG{l+m+mi}{1}\PYG{p}{,}  \PYG{n}{z}\PYG{o}{+}\PYG{n}{SIZE}\PYG{p}{,}       \PYG{n}{block}\PYG{o}{.}\PYG{n}{SNOW}\PYG{o}{.}\PYG{n}{id}\PYG{p}{)}

\PYG{n}{x}\PYG{p}{,}\PYG{n}{y}\PYG{p}{,}\PYG{n}{z} \PYG{o}{=} \PYG{n}{mc}\PYG{o}{.}\PYG{n}{player}\PYG{o}{.}\PYG{n}{getTilePos}\PYG{p}{(}\PYG{p}{)}

\PYG{c}{\PYGZsh{} build a house}
\PYG{n}{house}\PYG{p}{(}\PYG{p}{)}
\end{Verbatim}


\subsubsection{Build a street}
\label{kid/minecraft:build-a-street}
\begin{Verbatim}[commandchars=\\\{\},numbers=left,firstnumber=1,stepnumber=1]
\PYG{k+kn}{from} \PYG{n+nn}{mcpi} \PYG{k+kn}{import} \PYG{n}{minecraft}\PYG{p}{,} \PYG{n}{block}

\PYG{n}{mc} \PYG{o}{=} \PYG{n}{minecraft}\PYG{o}{.}\PYG{n}{Minecraft}\PYG{o}{.}\PYG{n}{create}\PYG{p}{(}\PYG{p}{)}
\PYG{n}{SIZE} \PYG{o}{=} \PYG{l+m+mi}{20}

\PYG{k}{def} \PYG{n+nf}{house}\PYG{p}{(}\PYG{p}{)}\PYG{p}{:}
    \PYG{n}{midx} \PYG{o}{=} \PYG{n}{x} \PYG{o}{+} \PYG{n}{SIZE}\PYG{o}{/}\PYG{l+m+mi}{2}
    \PYG{n}{midy} \PYG{o}{=} \PYG{n}{y} \PYG{o}{+} \PYG{n}{SIZE}\PYG{o}{/}\PYG{l+m+mi}{2}
    \PYG{n}{mc}\PYG{o}{.}\PYG{n}{setBlocks}\PYG{p}{(}     \PYG{n}{x}\PYG{p}{,}       \PYG{n}{y}\PYG{p}{,}  \PYG{n}{z}\PYG{p}{,}  \PYG{n}{x}\PYG{o}{+}\PYG{n}{SIZE}\PYG{p}{,}  \PYG{n}{y}\PYG{o}{+}\PYG{n}{SIZE}\PYG{p}{,}  \PYG{n}{z}\PYG{o}{+}\PYG{n}{SIZE}\PYG{p}{,}\PYG{n}{block}\PYG{o}{.}\PYG{n}{COBBLESTONE}\PYG{o}{.}\PYG{n}{id}\PYG{p}{)}
    \PYG{n}{mc}\PYG{o}{.}\PYG{n}{setBlocks}\PYG{p}{(}   \PYG{n}{x}\PYG{o}{+}\PYG{l+m+mi}{1}\PYG{p}{,}     \PYG{n}{y}\PYG{o}{+}\PYG{l+m+mi}{1}\PYG{p}{,}\PYG{n}{z}\PYG{o}{+}\PYG{l+m+mi}{1}\PYG{p}{,}\PYG{n}{x}\PYG{o}{+}\PYG{n}{SIZE}\PYG{o}{\PYGZhy{}}\PYG{l+m+mi}{1}\PYG{p}{,}\PYG{n}{y}\PYG{o}{+}\PYG{n}{SIZE}\PYG{o}{\PYGZhy{}}\PYG{l+m+mi}{1}\PYG{p}{,}\PYG{n}{z}\PYG{o}{+}\PYG{n}{SIZE}\PYG{o}{\PYGZhy{}}\PYG{l+m+mi}{1}\PYG{p}{,}        \PYG{n}{block}\PYG{o}{.}\PYG{n}{AIR}\PYG{o}{.}\PYG{n}{id}\PYG{p}{)}
    \PYG{n}{mc}\PYG{o}{.}\PYG{n}{setBlocks}\PYG{p}{(}   \PYG{n}{x}\PYG{o}{+}\PYG{l+m+mi}{1}\PYG{p}{,}     \PYG{n}{y}\PYG{o}{+}\PYG{l+m+mi}{1}\PYG{p}{,}\PYG{n}{z}\PYG{o}{+}\PYG{l+m+mi}{1}\PYG{p}{,}\PYG{n}{x}\PYG{o}{+}\PYG{n}{SIZE}\PYG{o}{\PYGZhy{}}\PYG{l+m+mi}{1}\PYG{p}{,}     \PYG{n}{y}\PYG{o}{+}\PYG{l+m+mi}{1}\PYG{p}{,}\PYG{n}{z}\PYG{o}{+}\PYG{n}{SIZE}\PYG{o}{\PYGZhy{}}\PYG{l+m+mi}{1}\PYG{p}{,}       \PYG{n}{block}\PYG{o}{.}\PYG{n}{WOOL}\PYG{o}{.}\PYG{n}{id}\PYG{p}{,}\PYG{l+m+mi}{7}\PYG{p}{)}
    \PYG{c}{\PYGZsh{} left window}
    \PYG{n}{mc}\PYG{o}{.}\PYG{n}{setBlocks}\PYG{p}{(}   \PYG{n}{x}\PYG{o}{+}\PYG{l+m+mi}{3}\PYG{p}{,}\PYG{n}{y}\PYG{o}{+}\PYG{n}{SIZE}\PYG{o}{\PYGZhy{}}\PYG{l+m+mi}{3}\PYG{p}{,}  \PYG{n}{z}\PYG{p}{,}  \PYG{n}{midx}\PYG{o}{\PYGZhy{}}\PYG{l+m+mi}{3}\PYG{p}{,}  \PYG{n}{midy}\PYG{o}{+}\PYG{l+m+mi}{3}\PYG{p}{,}       \PYG{n}{z}\PYG{p}{,}      \PYG{n}{block}\PYG{o}{.}\PYG{n}{GLASS}\PYG{o}{.}\PYG{n}{id}\PYG{p}{)}
    \PYG{c}{\PYGZsh{} right window}
    \PYG{n}{mc}\PYG{o}{.}\PYG{n}{setBlocks}\PYG{p}{(}\PYG{n}{midx}\PYG{o}{+}\PYG{l+m+mi}{3}\PYG{p}{,}\PYG{n}{y}\PYG{o}{+}\PYG{n}{SIZE}\PYG{o}{\PYGZhy{}}\PYG{l+m+mi}{3}\PYG{p}{,}  \PYG{n}{z}\PYG{p}{,}\PYG{n}{x}\PYG{o}{+}\PYG{n}{SIZE}\PYG{o}{\PYGZhy{}}\PYG{l+m+mi}{3}\PYG{p}{,}  \PYG{n}{midy}\PYG{o}{+}\PYG{l+m+mi}{3}\PYG{p}{,}       \PYG{n}{z}\PYG{p}{,}      \PYG{n}{block}\PYG{o}{.}\PYG{n}{GLASS}\PYG{o}{.}\PYG{n}{id}\PYG{p}{)}
    \PYG{c}{\PYGZsh{} door}
    \PYG{n}{mc}\PYG{o}{.}\PYG{n}{setBlocks}\PYG{p}{(}\PYG{n}{midx}\PYG{o}{\PYGZhy{}}\PYG{l+m+mi}{3}\PYG{p}{,}       \PYG{n}{y}\PYG{p}{,}  \PYG{n}{z}\PYG{p}{,}  \PYG{n}{midx}\PYG{o}{+}\PYG{l+m+mi}{3}\PYG{p}{,}    \PYG{n}{midy}\PYG{p}{,}       \PYG{n}{z}\PYG{p}{,}        \PYG{n}{block}\PYG{o}{.}\PYG{n}{AIR}\PYG{o}{.}\PYG{n}{id}\PYG{p}{)}
    \PYG{c}{\PYGZsh{} roof}
    \PYG{n}{mc}\PYG{o}{.}\PYG{n}{setBlocks}\PYG{p}{(}     \PYG{n}{x}\PYG{p}{,}\PYG{n}{y}\PYG{o}{+}\PYG{n}{SIZE}\PYG{o}{+}\PYG{l+m+mi}{1}\PYG{p}{,}  \PYG{n}{z}\PYG{p}{,}  \PYG{n}{x}\PYG{o}{+}\PYG{n}{SIZE}\PYG{p}{,}\PYG{n}{y}\PYG{o}{+}\PYG{n}{SIZE}\PYG{o}{+}\PYG{l+m+mi}{1}\PYG{p}{,}  \PYG{n}{z}\PYG{o}{+}\PYG{n}{SIZE}\PYG{p}{,}       \PYG{n}{block}\PYG{o}{.}\PYG{n}{SNOW}\PYG{o}{.}\PYG{n}{id}\PYG{p}{)}

\PYG{n}{x}\PYG{p}{,}\PYG{n}{y}\PYG{p}{,}\PYG{n}{z} \PYG{o}{=} \PYG{n}{mc}\PYG{o}{.}\PYG{n}{player}\PYG{o}{.}\PYG{n}{getTilePos}\PYG{p}{(}\PYG{p}{)}

\PYG{c}{\PYGZsh{} build a street}
\PYG{k}{for} \PYG{n}{h} \PYG{o+ow}{in} \PYG{n+nb}{range}\PYG{p}{(}\PYG{l+m+mi}{5}\PYG{p}{)}\PYG{p}{:}
    \PYG{n}{house}\PYG{p}{(}\PYG{p}{)}
    \PYG{n}{x} \PYG{o}{+}\PYG{o}{=} \PYG{n}{SIZE}
\end{Verbatim}
\setbox0\vbox{
\begin{minipage}{0.95\linewidth}
\textbf{\texttt{range(5) = range(0,5,1)}}

\medskip


\code{range(5)} means the list of the first five integers starting from 0, i.e. 0, 1, 2, 3, 4.

\code{range(5)} is indeed a shortcut of \code{range(0,5,1)} which means the list of integers starting from 0 and less than 5, increased 1 per step.
\end{minipage}}
\begin{center}\setlength{\fboxsep}{5pt}\shadowbox{\box0}\end{center}


\subsubsection{Magic bridge}
\label{kid/minecraft:magic-bridge}
Put a glass bridge under the feet whenever you walk, making sure that the player will never falls into the sea or falls out of the sky.

\begin{Verbatim}[commandchars=\\\{\},numbers=left,firstnumber=1,stepnumber=1]
\PYG{k+kn}{from} \PYG{n+nn}{mcpi} \PYG{k+kn}{import} \PYG{n}{minecraft}\PYG{p}{,} \PYG{n}{block}
\PYG{k+kn}{import} \PYG{n+nn}{time}

\PYG{n}{mc} \PYG{o}{=} \PYG{n}{minecraft}\PYG{o}{.}\PYG{n}{Minecraft}\PYG{o}{.}\PYG{n}{create}\PYG{p}{(}\PYG{p}{)}

\PYG{k}{def} \PYG{n+nf}{buildBridge}\PYG{p}{(}\PYG{p}{)}\PYG{p}{:}
    \PYG{n}{x}\PYG{p}{,}\PYG{n}{y}\PYG{p}{,}\PYG{n}{z} \PYG{o}{=} \PYG{n}{mc}\PYG{o}{.}\PYG{n}{player}\PYG{o}{.}\PYG{n}{getTilePos}\PYG{p}{(}\PYG{p}{)}
    \PYG{n}{b} \PYG{o}{=} \PYG{n}{mc}\PYG{o}{.}\PYG{n}{getBlock}\PYG{p}{(}\PYG{n}{x}\PYG{p}{,}\PYG{n}{y}\PYG{o}{\PYGZhy{}}\PYG{l+m+mi}{1}\PYG{p}{,}\PYG{n}{z}\PYG{p}{)} \PYG{c}{\PYGZsh{} y\PYGZhy{}1 means under the feet}
    \PYG{k}{if} \PYG{p}{(}\PYG{n}{b} \PYG{o}{==} \PYG{n}{block}\PYG{o}{.}\PYG{n}{AIR}\PYG{o}{.}\PYG{n}{id}\PYG{p}{)} \PYG{o+ow}{or}
       \PYG{p}{(}\PYG{n}{b} \PYG{o}{==} \PYG{n}{block}\PYG{o}{.}\PYG{n}{WATER\PYGZus{}STATIONARY}\PYG{o}{.}\PYG{n}{id}\PYG{p}{)} \PYG{o+ow}{or}
       \PYG{p}{(}\PYG{n}{b} \PYG{o}{==} \PYG{n}{block}\PYG{o}{.}\PYG{n}{WATER\PYGZus{}FLOWING}\PYG{o}{.}\PYG{n}{id}\PYG{p}{)}\PYG{p}{:}
        \PYG{n}{mc}\PYG{o}{.}\PYG{n}{setBlock}\PYG{p}{(}\PYG{n}{x}\PYG{p}{,}\PYG{n}{y}\PYG{o}{\PYGZhy{}}\PYG{l+m+mi}{1}\PYG{p}{,}\PYG{n}{z}\PYG{p}{,}\PYG{n}{block}\PYG{o}{.}\PYG{n}{GLASS}\PYG{o}{.}\PYG{n}{id}\PYG{p}{)}

\PYG{k}{while} \PYG{n+nb+bp}{True}\PYG{p}{:}
    \PYG{n}{time}\PYG{o}{.}\PYG{n}{sleep}\PYG{p}{(}\PYG{l+m+mf}{0.01}\PYG{p}{)}
    \PYG{n}{buildBridge}\PYG{p}{(}\PYG{p}{)}
\end{Verbatim}


\subsubsection{Vanishing bridge}
\label{kid/minecraft:vanishing-bridge}
Based on the previous magic bridge. Here we will add some more magic: whenever we are landed on a non-glass block, we will make disappear one glass block of our magic bridge.

To do this, we will build a list of all created glass blocks in order. When it will be the case, we will remove the oldest block in the list.

\begin{Verbatim}[commandchars=\\\{\},numbers=left,firstnumber=1,stepnumber=1]
\PYG{k+kn}{from} \PYG{n+nn}{mcpi} \PYG{k+kn}{import} \PYG{n}{minecraft}\PYG{p}{,} \PYG{n}{block}
\PYG{k+kn}{import} \PYG{n+nn}{time}

\PYG{n}{mc} \PYG{o}{=} \PYG{n}{minecraft}\PYG{o}{.}\PYG{n}{Minecraft}\PYG{o}{.}\PYG{n}{create}\PYG{p}{(}\PYG{p}{)}
\PYG{n}{bridge} \PYG{o}{=} \PYG{p}{[}\PYG{p}{]}

\PYG{k}{def} \PYG{n+nf}{buildBridge}\PYG{p}{(}\PYG{p}{)}\PYG{p}{:}
    \PYG{n}{x}\PYG{p}{,}\PYG{n}{y}\PYG{p}{,}\PYG{n}{z} \PYG{o}{=} \PYG{n}{mc}\PYG{o}{.}\PYG{n}{player}\PYG{o}{.}\PYG{n}{getTilePos}\PYG{p}{(}\PYG{p}{)}
    \PYG{n}{b} \PYG{o}{=} \PYG{n}{mc}\PYG{o}{.}\PYG{n}{getBlock}\PYG{p}{(}\PYG{n}{x}\PYG{p}{,}\PYG{n}{y}\PYG{o}{\PYGZhy{}}\PYG{l+m+mi}{1}\PYG{p}{,}\PYG{n}{z}\PYG{p}{)} \PYG{c}{\PYGZsh{} y\PYGZhy{}1 means under the feet}
    \PYG{k}{if} \PYG{p}{(}\PYG{n}{b} \PYG{o}{==} \PYG{n}{block}\PYG{o}{.}\PYG{n}{AIR}\PYG{o}{.}\PYG{n}{id}\PYG{p}{)} \PYG{o+ow}{or}
       \PYG{p}{(}\PYG{n}{b} \PYG{o}{==} \PYG{n}{block}\PYG{o}{.}\PYG{n}{WATER\PYGZus{}STATIONARY}\PYG{o}{.}\PYG{n}{id}\PYG{p}{)} \PYG{o+ow}{or}
       \PYG{p}{(}\PYG{n}{b} \PYG{o}{==} \PYG{n}{block}\PYG{o}{.}\PYG{n}{WATER\PYGZus{}FLOWING}\PYG{o}{.}\PYG{n}{id}\PYG{p}{)}\PYG{p}{:}
        \PYG{n}{mc}\PYG{o}{.}\PYG{n}{setBlock}\PYG{p}{(}\PYG{n}{x}\PYG{p}{,}\PYG{n}{y}\PYG{o}{\PYGZhy{}}\PYG{l+m+mi}{1}\PYG{p}{,}\PYG{n}{z}\PYG{p}{,}\PYG{n}{block}\PYG{o}{.}\PYG{n}{GLASS}\PYG{o}{.}\PYG{n}{id}\PYG{p}{)}
        \PYG{n}{coordinate} \PYG{o}{=} \PYG{p}{[}\PYG{n}{x}\PYG{p}{,}\PYG{n}{y}\PYG{o}{\PYGZhy{}}\PYG{l+m+mi}{1}\PYG{p}{,}\PYG{n}{z}\PYG{p}{]}
        \PYG{n}{bridge}\PYG{o}{.}\PYG{n}{append}\PYG{p}{(}\PYG{n}{coordinate}\PYG{p}{)}
    \PYG{k}{elif} \PYG{n}{b} \PYG{o}{!=} \PYG{n}{block}\PYG{o}{.}\PYG{n}{GLASS}\PYG{o}{.}\PYG{n}{id}\PYG{p}{:}
        \PYG{k}{if} \PYG{n+nb}{len}\PYG{p}{(}\PYG{n}{bridge}\PYG{p}{)} \PYG{o}{\PYGZgt{}} \PYG{l+m+mi}{0}\PYG{p}{:}
            \PYG{n}{coordinate} \PYG{o}{=} \PYG{n}{bridge}\PYG{o}{.}\PYG{n}{pop}\PYG{p}{(}\PYG{p}{)}
            \PYG{n}{a}\PYG{p}{,}\PYG{n}{b}\PYG{p}{,}\PYG{n}{c} \PYG{o}{=} \PYG{n}{coordinate}
            \PYG{n}{mc}\PYG{o}{.}\PYG{n}{setBlock}\PYG{p}{(}\PYG{n}{a}\PYG{p}{,}\PYG{n}{b}\PYG{p}{,}\PYG{n}{c}\PYG{p}{,}\PYG{n}{block}\PYG{o}{.}\PYG{n}{AIR}\PYG{o}{.}\PYG{n}{id}\PYG{p}{)}
            \PYG{n}{time}\PYG{o}{.}\PYG{n}{sleep}\PYG{p}{(}\PYG{l+m+mf}{0.01}\PYG{p}{)}

\PYG{k}{while} \PYG{n+nb+bp}{True}\PYG{p}{:}
    \PYG{n}{time}\PYG{o}{.}\PYG{n}{sleep}\PYG{p}{(}\PYG{l+m+mf}{0.01}\PYG{p}{)}
    \PYG{n}{buildBridge}\PYG{p}{(}\PYG{p}{)}
\end{Verbatim}


\subsubsection{Vanishing bridge (Improved Version)}
\label{kid/minecraft:vanishing-bridge-improved-version}
One question raised by Tom on the previous version of vanishing bridge: can we remove also the bridge's one glass block when we are on a glass block which is not part of the bridge?

For this to be done, we need to check when we have a glass block, whether its coordinate is in the list of the bridge's glass blocks' coordinates.

\begin{Verbatim}[commandchars=\\\{\},numbers=left,firstnumber=1,stepnumber=1]
\PYG{k+kn}{from} \PYG{n+nn}{mcpi} \PYG{k+kn}{import} \PYG{n}{minecraft}\PYG{p}{,} \PYG{n}{block}
\PYG{k+kn}{import} \PYG{n+nn}{time}

\PYG{n}{mc} \PYG{o}{=} \PYG{n}{minecraft}\PYG{o}{.}\PYG{n}{Minecraft}\PYG{o}{.}\PYG{n}{create}\PYG{p}{(}\PYG{p}{)}
\PYG{n}{bridge} \PYG{o}{=} \PYG{p}{[}\PYG{p}{]}

\PYG{k}{def} \PYG{n+nf}{buildBridge}\PYG{p}{(}\PYG{p}{)}\PYG{p}{:}
    \PYG{n}{x}\PYG{p}{,}\PYG{n}{y}\PYG{p}{,}\PYG{n}{z} \PYG{o}{=} \PYG{n}{mc}\PYG{o}{.}\PYG{n}{player}\PYG{o}{.}\PYG{n}{getTilePos}\PYG{p}{(}\PYG{p}{)}
    \PYG{n}{b} \PYG{o}{=} \PYG{n}{mc}\PYG{o}{.}\PYG{n}{getBlock}\PYG{p}{(}\PYG{n}{x}\PYG{p}{,}\PYG{n}{y}\PYG{o}{\PYGZhy{}}\PYG{l+m+mi}{1}\PYG{p}{,}\PYG{n}{z}\PYG{p}{)} \PYG{c}{\PYGZsh{} y\PYGZhy{}1 means under the feet}
    \PYG{k}{if} \PYG{p}{(}\PYG{n}{b} \PYG{o}{==} \PYG{n}{block}\PYG{o}{.}\PYG{n}{AIR}\PYG{o}{.}\PYG{n}{id}\PYG{p}{)} \PYG{o+ow}{or}
       \PYG{p}{(}\PYG{n}{b} \PYG{o}{==} \PYG{n}{block}\PYG{o}{.}\PYG{n}{WATER\PYGZus{}STATIONARY}\PYG{o}{.}\PYG{n}{id}\PYG{p}{)} \PYG{o+ow}{or}
       \PYG{p}{(}\PYG{n}{b} \PYG{o}{==} \PYG{n}{block}\PYG{o}{.}\PYG{n}{WATER\PYGZus{}FLOWING}\PYG{o}{.}\PYG{n}{id}\PYG{p}{)}\PYG{p}{:}
        \PYG{n}{mc}\PYG{o}{.}\PYG{n}{setBlock}\PYG{p}{(}\PYG{n}{x}\PYG{p}{,}\PYG{n}{y}\PYG{o}{\PYGZhy{}}\PYG{l+m+mi}{1}\PYG{p}{,}\PYG{n}{z}\PYG{p}{,}\PYG{n}{block}\PYG{o}{.}\PYG{n}{GLASS}\PYG{o}{.}\PYG{n}{id}\PYG{p}{)}
        \PYG{n}{coordinate} \PYG{o}{=} \PYG{p}{[}\PYG{n}{x}\PYG{p}{,}\PYG{n}{y}\PYG{o}{\PYGZhy{}}\PYG{l+m+mi}{1}\PYG{p}{,}\PYG{n}{z}\PYG{p}{]}
        \PYG{n}{bridge}\PYG{o}{.}\PYG{n}{append}\PYG{p}{(}\PYG{n}{coordinate}\PYG{p}{)}
    \PYG{k}{elif} \PYG{n}{b} \PYG{o}{!=} \PYG{n}{block}\PYG{o}{.}\PYG{n}{GLASS}\PYG{o}{.}\PYG{n}{id}\PYG{p}{:}
        \PYG{k}{if} \PYG{n+nb}{len}\PYG{p}{(}\PYG{n}{bridge}\PYG{p}{)} \PYG{o}{\PYGZgt{}} \PYG{l+m+mi}{0}\PYG{p}{:}
            \PYG{n}{coordinate} \PYG{o}{=} \PYG{n}{bridge}\PYG{o}{.}\PYG{n}{pop}\PYG{p}{(}\PYG{p}{)}
            \PYG{n}{a}\PYG{p}{,}\PYG{n}{b}\PYG{p}{,}\PYG{n}{c} \PYG{o}{=} \PYG{n}{coordinate}
            \PYG{n}{mc}\PYG{o}{.}\PYG{n}{setBlock}\PYG{p}{(}\PYG{n}{a}\PYG{p}{,}\PYG{n}{b}\PYG{p}{,}\PYG{n}{c}\PYG{p}{,}\PYG{n}{block}\PYG{o}{.}\PYG{n}{AIR}\PYG{o}{.}\PYG{n}{id}\PYG{p}{)}
            \PYG{n}{time}\PYG{o}{.}\PYG{n}{sleep}\PYG{p}{(}\PYG{l+m+mf}{0.01}\PYG{p}{)}
    \PYG{k}{else}\PYG{p}{:} \PYG{c}{\PYGZsh{} b == block.GLASS.id}
        \PYG{k}{if} \PYG{p}{[}\PYG{n}{x}\PYG{p}{,}\PYG{n}{y}\PYG{o}{\PYGZhy{}}\PYG{l+m+mi}{1}\PYG{p}{,}\PYG{n}{z}\PYG{p}{]} \PYG{o+ow}{not} \PYG{o+ow}{in} \PYG{n}{bridge}\PYG{p}{:}
            \PYG{k}{if} \PYG{n+nb}{len}\PYG{p}{(}\PYG{n}{bridge}\PYG{p}{)} \PYG{o}{\PYGZgt{}} \PYG{l+m+mi}{0}\PYG{p}{:}
                \PYG{n}{coordinate} \PYG{o}{=} \PYG{n}{bridge}\PYG{o}{.}\PYG{n}{pop}\PYG{p}{(}\PYG{p}{)}
                \PYG{n}{a}\PYG{p}{,}\PYG{n}{b}\PYG{p}{,}\PYG{n}{c} \PYG{o}{=} \PYG{n}{coordinate}
                \PYG{n}{mc}\PYG{o}{.}\PYG{n}{setBlock}\PYG{p}{(}\PYG{n}{a}\PYG{p}{,}\PYG{n}{b}\PYG{p}{,}\PYG{n}{c}\PYG{p}{,}\PYG{n}{block}\PYG{o}{.}\PYG{n}{AIR}\PYG{o}{.}\PYG{n}{id}\PYG{p}{)}
                \PYG{n}{time}\PYG{o}{.}\PYG{n}{sleep}\PYG{p}{(}\PYG{l+m+mf}{0.01}\PYG{p}{)}

\PYG{k}{while} \PYG{n+nb+bp}{True}\PYG{p}{:}
    \PYG{n}{time}\PYG{o}{.}\PYG{n}{sleep}\PYG{p}{(}\PYG{l+m+mf}{0.01}\PYG{p}{)}
    \PYG{n}{buildBridge}\PYG{p}{(}\PYG{p}{)}
\end{Verbatim}


\subsubsection{Vanishing bridge (Simplified Version)}
\label{kid/minecraft:vanishing-bridge-simplified-version}
There are two repeated blocks in the above code: it's our chance to create a new function!

\begin{Verbatim}[commandchars=\\\{\},numbers=left,firstnumber=1,stepnumber=1]
\PYG{k+kn}{from} \PYG{n+nn}{mcpi} \PYG{k+kn}{import} \PYG{n}{minecraft}\PYG{p}{,} \PYG{n}{block}
\PYG{k+kn}{import} \PYG{n+nn}{time}

\PYG{n}{mc} \PYG{o}{=} \PYG{n}{minecraft}\PYG{o}{.}\PYG{n}{Minecraft}\PYG{o}{.}\PYG{n}{create}\PYG{p}{(}\PYG{p}{)}
\PYG{n}{bridge} \PYG{o}{=} \PYG{p}{[}\PYG{p}{]}

\PYG{k}{def} \PYG{n+nf}{popBridge}\PYG{p}{(}\PYG{p}{)}\PYG{p}{:}
    \PYG{k}{if} \PYG{n+nb}{len}\PYG{p}{(}\PYG{n}{bridge}\PYG{p}{)} \PYG{o}{\PYGZgt{}} \PYG{l+m+mi}{0}\PYG{p}{:}
        \PYG{n}{coordinate} \PYG{o}{=} \PYG{n}{bridge}\PYG{o}{.}\PYG{n}{pop}\PYG{p}{(}\PYG{p}{)}
        \PYG{n}{a}\PYG{p}{,}\PYG{n}{b}\PYG{p}{,}\PYG{n}{c} \PYG{o}{=} \PYG{n}{coordinate}
        \PYG{n}{mc}\PYG{o}{.}\PYG{n}{setBlock}\PYG{p}{(}\PYG{n}{a}\PYG{p}{,}\PYG{n}{b}\PYG{p}{,}\PYG{n}{c}\PYG{p}{,}\PYG{n}{block}\PYG{o}{.}\PYG{n}{AIR}\PYG{o}{.}\PYG{n}{id}\PYG{p}{)}
        \PYG{n}{time}\PYG{o}{.}\PYG{n}{sleep}\PYG{p}{(}\PYG{l+m+mf}{0.01}\PYG{p}{)}

\PYG{k}{def} \PYG{n+nf}{buildBridge}\PYG{p}{(}\PYG{p}{)}\PYG{p}{:}
    \PYG{n}{x}\PYG{p}{,}\PYG{n}{y}\PYG{p}{,}\PYG{n}{z} \PYG{o}{=} \PYG{n}{mc}\PYG{o}{.}\PYG{n}{player}\PYG{o}{.}\PYG{n}{getTilePos}\PYG{p}{(}\PYG{p}{)}
    \PYG{n}{b} \PYG{o}{=} \PYG{n}{mc}\PYG{o}{.}\PYG{n}{getBlock}\PYG{p}{(}\PYG{n}{x}\PYG{p}{,}\PYG{n}{y}\PYG{o}{\PYGZhy{}}\PYG{l+m+mi}{1}\PYG{p}{,}\PYG{n}{z}\PYG{p}{)} \PYG{c}{\PYGZsh{} y\PYGZhy{}1 means under the feet}
    \PYG{k}{if} \PYG{p}{(}\PYG{n}{b} \PYG{o}{==} \PYG{n}{block}\PYG{o}{.}\PYG{n}{AIR}\PYG{o}{.}\PYG{n}{id}\PYG{p}{)} \PYG{o+ow}{or}
       \PYG{p}{(}\PYG{n}{b} \PYG{o}{==} \PYG{n}{block}\PYG{o}{.}\PYG{n}{WATER\PYGZus{}STATIONARY}\PYG{o}{.}\PYG{n}{id}\PYG{p}{)} \PYG{o+ow}{or}
       \PYG{p}{(}\PYG{n}{b} \PYG{o}{==} \PYG{n}{block}\PYG{o}{.}\PYG{n}{WATER\PYGZus{}FLOWING}\PYG{o}{.}\PYG{n}{id}\PYG{p}{)}\PYG{p}{:}
        \PYG{n}{mc}\PYG{o}{.}\PYG{n}{setBlock}\PYG{p}{(}\PYG{n}{x}\PYG{p}{,}\PYG{n}{y}\PYG{o}{\PYGZhy{}}\PYG{l+m+mi}{1}\PYG{p}{,}\PYG{n}{z}\PYG{p}{,}\PYG{n}{block}\PYG{o}{.}\PYG{n}{GLASS}\PYG{o}{.}\PYG{n}{id}\PYG{p}{)}
        \PYG{n}{coordinate} \PYG{o}{=} \PYG{p}{[}\PYG{n}{x}\PYG{p}{,}\PYG{n}{y}\PYG{o}{\PYGZhy{}}\PYG{l+m+mi}{1}\PYG{p}{,}\PYG{n}{z}\PYG{p}{]}
        \PYG{n}{bridge}\PYG{o}{.}\PYG{n}{append}\PYG{p}{(}\PYG{n}{coordinate}\PYG{p}{)}
    \PYG{k}{elif} \PYG{n}{b} \PYG{o}{!=} \PYG{n}{block}\PYG{o}{.}\PYG{n}{GLASS}\PYG{o}{.}\PYG{n}{id}\PYG{p}{:}
        \PYG{n}{popBridge}\PYG{p}{(}\PYG{p}{)}
    \PYG{k}{else}\PYG{p}{:} \PYG{c}{\PYGZsh{} b == block.GLASS.id}
        \PYG{k}{if} \PYG{p}{[}\PYG{n}{x}\PYG{p}{,}\PYG{n}{y}\PYG{o}{\PYGZhy{}}\PYG{l+m+mi}{1}\PYG{p}{,}\PYG{n}{z}\PYG{p}{]} \PYG{o+ow}{not} \PYG{o+ow}{in} \PYG{n}{bridge}\PYG{p}{:}
            \PYG{n}{popBridge}\PYG{p}{(}\PYG{p}{)}

\PYG{k}{while} \PYG{n+nb+bp}{True}\PYG{p}{:}
    \PYG{n}{time}\PYG{o}{.}\PYG{n}{sleep}\PYG{p}{(}\PYG{l+m+mf}{0.01}\PYG{p}{)}
    \PYG{n}{buildBridge}\PYG{p}{(}\PYG{p}{)}
\end{Verbatim}


\subsubsection{Treasure Hunt}
\label{kid/minecraft:treasure-hunt}
\begin{Verbatim}[commandchars=\\\{\},numbers=left,firstnumber=1,stepnumber=1]
\PYG{k+kn}{from} \PYG{n+nn}{mcpi} \PYG{k+kn}{import} \PYG{n}{minecraft}\PYG{p}{,} \PYG{n}{block}
\PYG{k+kn}{import} \PYG{n+nn}{time}\PYG{o}{,} \PYG{n+nn}{random}

\PYG{n}{mc} \PYG{o}{=} \PYG{n}{minecraft}\PYG{o}{.}\PYG{n}{Minecraft}\PYG{o}{.}\PYG{n}{create}\PYG{p}{(}\PYG{p}{)}

\PYG{n}{score} \PYG{o}{=} \PYG{l+m+mi}{0}
\PYG{n}{RANGE} \PYG{o}{=} \PYG{l+m+mi}{5} \PYG{c}{\PYGZsh{} increase this number to make a more difficult game!}
\PYG{n}{TIMEOUT} \PYG{o}{=} \PYG{l+m+mi}{10}
\PYG{n}{timer} \PYG{o}{=} \PYG{n}{TIMEOUT}

\PYG{n}{treasurex} \PYG{o}{=} \PYG{n+nb+bp}{None}
\PYG{n}{treasurey} \PYG{o}{=} \PYG{n+nb+bp}{None}
\PYG{n}{treasurez} \PYG{o}{=} \PYG{n+nb+bp}{None}

\PYG{k}{def} \PYG{n+nf}{placeTreasure}\PYG{p}{(}\PYG{p}{)}\PYG{p}{:}
    \PYG{k}{global} \PYG{n}{treasurex}\PYG{p}{,} \PYG{n}{treasurey}\PYG{p}{,} \PYG{n}{treasurez}
    \PYG{n}{x}\PYG{p}{,}\PYG{n}{y}\PYG{p}{,}\PYG{n}{z} \PYG{o}{=} \PYG{n}{mc}\PYG{o}{.}\PYG{n}{player}\PYG{o}{.}\PYG{n}{getTilePos}\PYG{p}{(}\PYG{p}{)}
    \PYG{n}{treasurex} \PYG{o}{=} \PYG{n}{random}\PYG{o}{.}\PYG{n}{randint}\PYG{p}{(}\PYG{n}{x}\PYG{p}{,} \PYG{n}{x}\PYG{o}{+}\PYG{n}{RANGE}\PYG{p}{)}
    \PYG{n}{treasurey} \PYG{o}{=} \PYG{n}{random}\PYG{o}{.}\PYG{n}{randint}\PYG{p}{(}\PYG{n}{y}\PYG{o}{+}\PYG{l+m+mi}{2}\PYG{p}{,} \PYG{n}{y}\PYG{o}{+}\PYG{n}{RANGE}\PYG{o}{+}\PYG{l+m+mi}{2}\PYG{p}{)}
    \PYG{n}{treasurez} \PYG{o}{=} \PYG{n}{random}\PYG{o}{.}\PYG{n}{randint}\PYG{p}{(}\PYG{n}{z}\PYG{p}{,} \PYG{n}{z}\PYG{o}{+}\PYG{n}{RANGE}\PYG{p}{)}
    \PYG{n}{mc}\PYG{o}{.}\PYG{n}{setBlock}\PYG{p}{(}\PYG{n}{treasurex}\PYG{p}{,} \PYG{n}{treasurey}\PYG{p}{,} \PYG{n}{treasurez}\PYG{p}{,} \PYG{n}{block}\PYG{o}{.}\PYG{n}{DIAMOND\PYGZus{}BLOCK}\PYG{o}{.}\PYG{n}{id}\PYG{p}{)}

\PYG{k}{def} \PYG{n+nf}{checkHit}\PYG{p}{(}\PYG{p}{)}\PYG{p}{:}
    \PYG{k}{global} \PYG{n}{score}\PYG{p}{,} \PYG{n}{treasurex}
    \PYG{n}{events} \PYG{o}{=} \PYG{n}{mc}\PYG{o}{.}\PYG{n}{events}\PYG{o}{.}\PYG{n}{pollBlockHits}\PYG{p}{(}\PYG{p}{)}
    \PYG{k}{for} \PYG{n}{e} \PYG{o+ow}{in} \PYG{n}{events}\PYG{p}{:}
        \PYG{n}{x}\PYG{p}{,}\PYG{n}{y}\PYG{p}{,}\PYG{n}{z} \PYG{o}{=} \PYG{n}{e}\PYG{o}{.}\PYG{n}{pos}
        \PYG{k}{if} \PYG{n}{x} \PYG{o}{==} \PYG{n}{treasurex} \PYG{o+ow}{and} \PYG{n}{y} \PYG{o}{==} \PYG{n}{treasurey} \PYG{o+ow}{and} \PYG{n}{z} \PYG{o}{==} \PYG{n}{treasurez}\PYG{p}{:}
            \PYG{n}{mc}\PYG{o}{.}\PYG{n}{postToChat}\PYG{p}{(}\PYG{l+s}{\PYGZdq{}}\PYG{l+s}{HIT!}\PYG{l+s}{\PYGZdq{}}\PYG{p}{)}
            \PYG{n}{score} \PYG{o}{+}\PYG{o}{=} \PYG{l+m+mi}{20}
            \PYG{n}{mc}\PYG{o}{.}\PYG{n}{setBlock}\PYG{p}{(}\PYG{n}{treasurex}\PYG{p}{,} \PYG{n}{treasurey}\PYG{p}{,} \PYG{n}{treasurez}\PYG{p}{,} \PYG{n}{block}\PYG{o}{.}\PYG{n}{AIR}\PYG{o}{.}\PYG{n}{id}\PYG{p}{)}
            \PYG{n}{treasurex} \PYG{o}{=} \PYG{n+nb+bp}{None}

\PYG{k}{def} \PYG{n+nf}{homingBeacon}\PYG{p}{(}\PYG{p}{)}\PYG{p}{:}
    \PYG{k}{global} \PYG{n}{timer}
    \PYG{k}{if} \PYG{n}{treasurex} \PYG{o}{!=} \PYG{n+nb+bp}{None}\PYG{p}{:}
        \PYG{n}{timer} \PYG{o}{\PYGZhy{}}\PYG{o}{=} \PYG{l+m+mi}{1}
        \PYG{k}{if} \PYG{n}{timer} \PYG{o}{==} \PYG{l+m+mi}{0}\PYG{p}{:}
            \PYG{n}{timer} \PYG{o}{=} \PYG{n}{TIMEOUT}
            \PYG{n}{x}\PYG{p}{,}\PYG{n}{y}\PYG{p}{,}\PYG{n}{z} \PYG{o}{=} \PYG{n}{mc}\PYG{o}{.}\PYG{n}{player}\PYG{o}{.}\PYG{n}{getTilePos}\PYG{p}{(}\PYG{p}{)}
            \PYG{n}{diffx} \PYG{o}{=} \PYG{n+nb}{abs}\PYG{p}{(}\PYG{n}{x} \PYG{o}{\PYGZhy{}} \PYG{n}{treasurex}\PYG{p}{)}
            \PYG{n}{diffy} \PYG{o}{=} \PYG{n+nb}{abs}\PYG{p}{(}\PYG{n}{y} \PYG{o}{\PYGZhy{}} \PYG{n}{treasurey}\PYG{p}{)}
            \PYG{n}{diffz} \PYG{o}{=} \PYG{n+nb}{abs}\PYG{p}{(}\PYG{n}{z} \PYG{o}{\PYGZhy{}} \PYG{n}{treasurez}\PYG{p}{)}
            \PYG{n}{diff} \PYG{o}{=} \PYG{n}{diffx} \PYG{o}{+} \PYG{n}{diffy} \PYG{o}{+} \PYG{n}{diffz}
            \PYG{n}{mc}\PYG{o}{.}\PYG{n}{postToChat}\PYG{p}{(}\PYG{l+s}{\PYGZdq{}}\PYG{l+s}{score:}\PYG{l+s}{\PYGZdq{}} \PYG{o}{+} \PYG{n+nb}{str}\PYG{p}{(}\PYG{n}{score}\PYG{p}{)} \PYG{o}{+} \PYG{l+s}{\PYGZdq{}}\PYG{l+s}{ treasure:}\PYG{l+s}{\PYGZdq{}} \PYG{o}{+} \PYG{n+nb}{str}\PYG{p}{(}\PYG{n}{diff}\PYG{p}{)}\PYG{p}{)}

\PYG{n}{bridge} \PYG{o}{=} \PYG{p}{[}\PYG{p}{]}

\PYG{k}{def} \PYG{n+nf}{buildBridge}\PYG{p}{(}\PYG{p}{)}\PYG{p}{:}
    \PYG{k}{global} \PYG{n}{score}
    \PYG{n}{x}\PYG{p}{,}\PYG{n}{y}\PYG{p}{,}\PYG{n}{z} \PYG{o}{=} \PYG{n}{mc}\PYG{o}{.}\PYG{n}{player}\PYG{o}{.}\PYG{n}{getTilePos}\PYG{p}{(}\PYG{p}{)}
    \PYG{n}{b} \PYG{o}{=} \PYG{n}{mc}\PYG{o}{.}\PYG{n}{getBlock}\PYG{p}{(}\PYG{n}{x}\PYG{p}{,} \PYG{n}{y}\PYG{o}{\PYGZhy{}}\PYG{l+m+mi}{1}\PYG{p}{,} \PYG{n}{z}\PYG{p}{)}
    \PYG{k}{if} \PYG{n}{treasurex} \PYG{o}{==} \PYG{n+nb+bp}{None}\PYG{p}{:}
        \PYG{k}{if} \PYG{n+nb}{len}\PYG{p}{(}\PYG{n}{bridge}\PYG{p}{)} \PYG{o}{\PYGZgt{}} \PYG{l+m+mi}{0}\PYG{p}{:}
            \PYG{n}{coordinate} \PYG{o}{=} \PYG{n}{bridge}\PYG{o}{.}\PYG{n}{pop}\PYG{p}{(}\PYG{p}{)}
            \PYG{n}{a}\PYG{p}{,}\PYG{n}{b}\PYG{p}{,}\PYG{n}{c} \PYG{o}{=} \PYG{n}{coordinate}
            \PYG{n}{mc}\PYG{o}{.}\PYG{n}{setBlock}\PYG{p}{(}\PYG{n}{a}\PYG{p}{,} \PYG{n}{b}\PYG{p}{,} \PYG{n}{c}\PYG{p}{,} \PYG{n}{block}\PYG{o}{.}\PYG{n}{AIR}\PYG{o}{.}\PYG{n}{id}\PYG{p}{)}
            \PYG{n}{mc}\PYG{o}{.}\PYG{n}{postToChat}\PYG{p}{(}\PYG{l+s}{\PYGZdq{}}\PYG{l+s}{bridge:}\PYG{l+s}{\PYGZdq{}} \PYG{o}{+} \PYG{n+nb}{str}\PYG{p}{(}\PYG{n+nb}{len}\PYG{p}{(}\PYG{n}{bridge}\PYG{p}{)}\PYG{p}{)}\PYG{p}{)}
            \PYG{n}{time}\PYG{o}{.}\PYG{n}{sleep}\PYG{p}{(}\PYG{l+m+mf}{0.01}\PYG{p}{)}
    \PYG{k}{elif} \PYG{n}{b} \PYG{o}{!=} \PYG{n}{block}\PYG{o}{.}\PYG{n}{GOLD\PYGZus{}BLOCK}\PYG{o}{.}\PYG{n}{id}\PYG{p}{:}
        \PYG{n}{mc}\PYG{o}{.}\PYG{n}{setBlock}\PYG{p}{(}\PYG{n}{x}\PYG{p}{,} \PYG{n}{y}\PYG{o}{\PYGZhy{}}\PYG{l+m+mi}{1}\PYG{p}{,} \PYG{n}{z}\PYG{p}{,} \PYG{n}{block}\PYG{o}{.}\PYG{n}{GOLD\PYGZus{}BLOCK}\PYG{o}{.}\PYG{n}{id}\PYG{p}{)}
        \PYG{n}{coordinate} \PYG{o}{=} \PYG{p}{[}\PYG{n}{x}\PYG{p}{,} \PYG{n}{y}\PYG{o}{\PYGZhy{}}\PYG{l+m+mi}{1}\PYG{p}{,} \PYG{n}{z}\PYG{p}{]}
        \PYG{n}{bridge}\PYG{o}{.}\PYG{n}{append}\PYG{p}{(}\PYG{n}{coordinate}\PYG{p}{)}
        \PYG{n}{score} \PYG{o}{\PYGZhy{}}\PYG{o}{=} \PYG{l+m+mi}{1}

\PYG{k}{while} \PYG{n+nb+bp}{True}\PYG{p}{:}
    \PYG{n}{time}\PYG{o}{.}\PYG{n}{sleep}\PYG{p}{(}\PYG{l+m+mf}{0.01}\PYG{p}{)}

    \PYG{k}{if} \PYG{n}{treasurex} \PYG{o}{==} \PYG{n+nb+bp}{None} \PYG{o+ow}{and} \PYG{n+nb}{len}\PYG{p}{(}\PYG{n}{bridge}\PYG{p}{)} \PYG{o}{==} \PYG{l+m+mi}{0}\PYG{p}{:}
        \PYG{n}{placeTreasure}\PYG{p}{(}\PYG{p}{)}

    \PYG{n}{checkHit}\PYG{p}{(}\PYG{p}{)}
    \PYG{n}{homingBeacon}\PYG{p}{(}\PYG{p}{)}
    \PYG{n}{buildBridge}\PYG{p}{(}\PYG{p}{)}
\end{Verbatim}


\subsubsection{Build a Maze}
\label{kid/minecraft:build-a-maze}
\begin{Verbatim}[commandchars=\\\{\},numbers=left,firstnumber=1,stepnumber=1]
\PYG{k+kn}{from} \PYG{n+nn}{mcpi} \PYG{k+kn}{import} \PYG{n}{minecraft}\PYG{p}{,} \PYG{n}{block}

\PYG{n}{mc} \PYG{o}{=} \PYG{n}{minecraft}\PYG{o}{.}\PYG{n}{Minecraft}\PYG{o}{.}\PYG{n}{create}\PYG{p}{(}\PYG{p}{)}

\PYG{n}{GAP} \PYG{o}{=} \PYG{n}{block}\PYG{o}{.}\PYG{n}{AIR}\PYG{o}{.}\PYG{n}{id}
\PYG{n}{WALL} \PYG{o}{=} \PYG{n}{block}\PYG{o}{.}\PYG{n}{GOLD\PYGZus{}BLOCK}\PYG{o}{.}\PYG{n}{id}
\PYG{n}{FLOOR} \PYG{o}{=} \PYG{n}{block}\PYG{o}{.}\PYG{n}{GRASS}\PYG{o}{.}\PYG{n}{id}

\PYG{n}{FILENAME} \PYG{o}{=} \PYG{l+s}{\PYGZdq{}}\PYG{l+s}{maze.csv}\PYG{l+s}{\PYGZdq{}}
\PYG{n}{f} \PYG{o}{=} \PYG{n+nb}{open}\PYG{p}{(}\PYG{n}{FILENAME}\PYG{p}{,} \PYG{l+s}{\PYGZdq{}}\PYG{l+s}{r}\PYG{l+s}{\PYGZdq{}}\PYG{p}{)}

\PYG{n}{x}\PYG{p}{,}\PYG{n}{y}\PYG{p}{,}\PYG{n}{z} \PYG{o}{=} \PYG{n}{mc}\PYG{o}{.}\PYG{n}{player}\PYG{o}{.}\PYG{n}{getTilePos}\PYG{p}{(}\PYG{p}{)}
\PYG{n}{ORIGINX} \PYG{o}{=} \PYG{n}{x}\PYG{o}{+}\PYG{l+m+mi}{1}
\PYG{n}{ORIGINY} \PYG{o}{=} \PYG{n}{y}
\PYG{n}{ORIGINZ} \PYG{o}{=} \PYG{n}{z}\PYG{o}{+}\PYG{l+m+mi}{1}

\PYG{n}{z} \PYG{o}{=} \PYG{n}{ORIGINZ}
\PYG{k}{for} \PYG{n}{line} \PYG{o+ow}{in} \PYG{n}{f}\PYG{o}{.}\PYG{n}{readlines}\PYG{p}{(}\PYG{p}{)}\PYG{p}{:}
    \PYG{n}{data} \PYG{o}{=} \PYG{n}{line}\PYG{o}{.}\PYG{n}{split}\PYG{p}{(}\PYG{l+s}{\PYGZdq{}}\PYG{l+s}{,}\PYG{l+s}{\PYGZdq{}}\PYG{p}{)}
    \PYG{n}{x} \PYG{o}{=} \PYG{n}{ORIGINX}
    \PYG{k}{for} \PYG{n}{cell} \PYG{o+ow}{in} \PYG{n}{data}\PYG{p}{:}
        \PYG{k}{if} \PYG{n}{cell} \PYG{o}{==} \PYG{l+s}{\PYGZdq{}}\PYG{l+s}{0}\PYG{l+s}{\PYGZdq{}}\PYG{p}{:}
            \PYG{n}{b} \PYG{o}{=} \PYG{n}{GAP}
        \PYG{k}{else}\PYG{p}{:}
            \PYG{n}{b} \PYG{o}{=} \PYG{n}{WALL}
        \PYG{n}{mc}\PYG{o}{.}\PYG{n}{setBlock}\PYG{p}{(}\PYG{n}{x}\PYG{p}{,} \PYG{n}{ORIGINY}\PYG{p}{,} \PYG{n}{z}\PYG{p}{,} \PYG{n}{b}\PYG{p}{)}
        \PYG{n}{mc}\PYG{o}{.}\PYG{n}{setBlock}\PYG{p}{(}\PYG{n}{x}\PYG{p}{,} \PYG{n}{ORIGINY}\PYG{o}{+}\PYG{l+m+mi}{1}\PYG{p}{,} \PYG{n}{z}\PYG{p}{,} \PYG{n}{b}\PYG{p}{)}
        \PYG{n}{mc}\PYG{o}{.}\PYG{n}{setBlock}\PYG{p}{(}\PYG{n}{x}\PYG{p}{,} \PYG{n}{ORIGINY}\PYG{o}{\PYGZhy{}}\PYG{l+m+mi}{1}\PYG{p}{,} \PYG{n}{z}\PYG{p}{,} \PYG{n}{FLOOR}\PYG{p}{)}
        \PYG{n}{x} \PYG{o}{+}\PYG{o}{=} \PYG{l+m+mi}{1}
    \PYG{n}{z} \PYG{o}{+}\PYG{o}{=} \PYG{l+m+mi}{1}
\end{Verbatim}
\setbox0\vbox{
\begin{minipage}{0.95\linewidth}
\textbf{maze.csv sample}

\medskip


\begin{Verbatim}[commandchars=\\\{\}]
\PYG{l+m+mi}{1}\PYG{p}{,}\PYG{l+m+mi}{1}\PYG{p}{,}\PYG{l+m+mi}{1}\PYG{p}{,}\PYG{l+m+mi}{1}\PYG{p}{,}\PYG{l+m+mi}{1}\PYG{p}{,}\PYG{l+m+mi}{1}\PYG{p}{,}\PYG{l+m+mi}{1}\PYG{p}{,}\PYG{l+m+mi}{1}\PYG{p}{,}\PYG{l+m+mi}{1}\PYG{p}{,}\PYG{l+m+mi}{1}\PYG{p}{,}\PYG{l+m+mi}{1}\PYG{p}{,}\PYG{l+m+mi}{1}\PYG{p}{,}\PYG{l+m+mi}{1}\PYG{p}{,}\PYG{l+m+mi}{1}\PYG{p}{,}\PYG{l+m+mi}{1}\PYG{p}{,}\PYG{l+m+mi}{1}
\PYG{l+m+mi}{0}\PYG{p}{,}\PYG{l+m+mi}{0}\PYG{p}{,}\PYG{l+m+mi}{0}\PYG{p}{,}\PYG{l+m+mi}{0}\PYG{p}{,}\PYG{l+m+mi}{0}\PYG{p}{,}\PYG{l+m+mi}{0}\PYG{p}{,}\PYG{l+m+mi}{0}\PYG{p}{,}\PYG{l+m+mi}{0}\PYG{p}{,}\PYG{l+m+mi}{0}\PYG{p}{,}\PYG{l+m+mi}{0}\PYG{p}{,}\PYG{l+m+mi}{0}\PYG{p}{,}\PYG{l+m+mi}{0}\PYG{p}{,}\PYG{l+m+mi}{0}\PYG{p}{,}\PYG{l+m+mi}{0}\PYG{p}{,}\PYG{l+m+mi}{0}\PYG{p}{,}\PYG{l+m+mi}{1}
\PYG{l+m+mi}{1}\PYG{p}{,}\PYG{l+m+mi}{1}\PYG{p}{,}\PYG{l+m+mi}{1}\PYG{p}{,}\PYG{l+m+mi}{1}\PYG{p}{,}\PYG{l+m+mi}{1}\PYG{p}{,}\PYG{l+m+mi}{1}\PYG{p}{,}\PYG{l+m+mi}{1}\PYG{p}{,}\PYG{l+m+mi}{1}\PYG{p}{,}\PYG{l+m+mi}{1}\PYG{p}{,}\PYG{l+m+mi}{0}\PYG{p}{,}\PYG{l+m+mi}{1}\PYG{p}{,}\PYG{l+m+mi}{0}\PYG{p}{,}\PYG{l+m+mi}{1}\PYG{p}{,}\PYG{l+m+mi}{1}\PYG{p}{,}\PYG{l+m+mi}{0}\PYG{p}{,}\PYG{l+m+mi}{1}
\PYG{l+m+mi}{1}\PYG{p}{,}\PYG{l+m+mi}{0}\PYG{p}{,}\PYG{l+m+mi}{0}\PYG{p}{,}\PYG{l+m+mi}{1}\PYG{p}{,}\PYG{l+m+mi}{0}\PYG{p}{,}\PYG{l+m+mi}{0}\PYG{p}{,}\PYG{l+m+mi}{0}\PYG{p}{,}\PYG{l+m+mi}{0}\PYG{p}{,}\PYG{l+m+mi}{1}\PYG{p}{,}\PYG{l+m+mi}{0}\PYG{p}{,}\PYG{l+m+mi}{1}\PYG{p}{,}\PYG{l+m+mi}{0}\PYG{p}{,}\PYG{l+m+mi}{1}\PYG{p}{,}\PYG{l+m+mi}{0}\PYG{p}{,}\PYG{l+m+mi}{0}\PYG{p}{,}\PYG{l+m+mi}{1}
\PYG{l+m+mi}{1}\PYG{p}{,}\PYG{l+m+mi}{1}\PYG{p}{,}\PYG{l+m+mi}{0}\PYG{p}{,}\PYG{l+m+mi}{1}\PYG{p}{,}\PYG{l+m+mi}{0}\PYG{p}{,}\PYG{l+m+mi}{1}\PYG{p}{,}\PYG{l+m+mi}{1}\PYG{p}{,}\PYG{l+m+mi}{0}\PYG{p}{,}\PYG{l+m+mi}{0}\PYG{p}{,}\PYG{l+m+mi}{0}\PYG{p}{,}\PYG{l+m+mi}{0}\PYG{p}{,}\PYG{l+m+mi}{0}\PYG{p}{,}\PYG{l+m+mi}{1}\PYG{p}{,}\PYG{l+m+mi}{0}\PYG{p}{,}\PYG{l+m+mi}{1}\PYG{p}{,}\PYG{l+m+mi}{1}
\PYG{l+m+mi}{1}\PYG{p}{,}\PYG{l+m+mi}{1}\PYG{p}{,}\PYG{l+m+mi}{0}\PYG{p}{,}\PYG{l+m+mi}{1}\PYG{p}{,}\PYG{l+m+mi}{0}\PYG{p}{,}\PYG{l+m+mi}{1}\PYG{p}{,}\PYG{l+m+mi}{1}\PYG{p}{,}\PYG{l+m+mi}{1}\PYG{p}{,}\PYG{l+m+mi}{1}\PYG{p}{,}\PYG{l+m+mi}{1}\PYG{p}{,}\PYG{l+m+mi}{1}\PYG{p}{,}\PYG{l+m+mi}{1}\PYG{p}{,}\PYG{l+m+mi}{1}\PYG{p}{,}\PYG{l+m+mi}{0}\PYG{p}{,}\PYG{l+m+mi}{1}\PYG{p}{,}\PYG{l+m+mi}{1}
\PYG{l+m+mi}{1}\PYG{p}{,}\PYG{l+m+mi}{1}\PYG{p}{,}\PYG{l+m+mi}{0}\PYG{p}{,}\PYG{l+m+mi}{0}\PYG{p}{,}\PYG{l+m+mi}{0}\PYG{p}{,}\PYG{l+m+mi}{1}\PYG{p}{,}\PYG{l+m+mi}{1}\PYG{p}{,}\PYG{l+m+mi}{1}\PYG{p}{,}\PYG{l+m+mi}{1}\PYG{p}{,}\PYG{l+m+mi}{1}\PYG{p}{,}\PYG{l+m+mi}{0}\PYG{p}{,}\PYG{l+m+mi}{0}\PYG{p}{,}\PYG{l+m+mi}{0}\PYG{p}{,}\PYG{l+m+mi}{0}\PYG{p}{,}\PYG{l+m+mi}{1}\PYG{p}{,}\PYG{l+m+mi}{1}
\PYG{l+m+mi}{1}\PYG{p}{,}\PYG{l+m+mi}{1}\PYG{p}{,}\PYG{l+m+mi}{1}\PYG{p}{,}\PYG{l+m+mi}{1}\PYG{p}{,}\PYG{l+m+mi}{1}\PYG{p}{,}\PYG{l+m+mi}{1}\PYG{p}{,}\PYG{l+m+mi}{0}\PYG{p}{,}\PYG{l+m+mi}{0}\PYG{p}{,}\PYG{l+m+mi}{0}\PYG{p}{,}\PYG{l+m+mi}{0}\PYG{p}{,}\PYG{l+m+mi}{0}\PYG{p}{,}\PYG{l+m+mi}{1}\PYG{p}{,}\PYG{l+m+mi}{1}\PYG{p}{,}\PYG{l+m+mi}{1}\PYG{p}{,}\PYG{l+m+mi}{1}\PYG{p}{,}\PYG{l+m+mi}{1}
\PYG{l+m+mi}{1}\PYG{p}{,}\PYG{l+m+mi}{0}\PYG{p}{,}\PYG{l+m+mi}{0}\PYG{p}{,}\PYG{l+m+mi}{0}\PYG{p}{,}\PYG{l+m+mi}{0}\PYG{p}{,}\PYG{l+m+mi}{1}\PYG{p}{,}\PYG{l+m+mi}{0}\PYG{p}{,}\PYG{l+m+mi}{0}\PYG{p}{,}\PYG{l+m+mi}{0}\PYG{p}{,}\PYG{l+m+mi}{0}\PYG{p}{,}\PYG{l+m+mi}{0}\PYG{p}{,}\PYG{l+m+mi}{0}\PYG{p}{,}\PYG{l+m+mi}{0}\PYG{p}{,}\PYG{l+m+mi}{0}\PYG{p}{,}\PYG{l+m+mi}{0}\PYG{p}{,}\PYG{l+m+mi}{1}
\PYG{l+m+mi}{1}\PYG{p}{,}\PYG{l+m+mi}{0}\PYG{p}{,}\PYG{l+m+mi}{1}\PYG{p}{,}\PYG{l+m+mi}{1}\PYG{p}{,}\PYG{l+m+mi}{1}\PYG{p}{,}\PYG{l+m+mi}{1}\PYG{p}{,}\PYG{l+m+mi}{0}\PYG{p}{,}\PYG{l+m+mi}{0}\PYG{p}{,}\PYG{l+m+mi}{0}\PYG{p}{,}\PYG{l+m+mi}{0}\PYG{p}{,}\PYG{l+m+mi}{0}\PYG{p}{,}\PYG{l+m+mi}{1}\PYG{p}{,}\PYG{l+m+mi}{1}\PYG{p}{,}\PYG{l+m+mi}{1}\PYG{p}{,}\PYG{l+m+mi}{1}\PYG{p}{,}\PYG{l+m+mi}{1}
\PYG{l+m+mi}{1}\PYG{p}{,}\PYG{l+m+mi}{0}\PYG{p}{,}\PYG{l+m+mi}{0}\PYG{p}{,}\PYG{l+m+mi}{0}\PYG{p}{,}\PYG{l+m+mi}{0}\PYG{p}{,}\PYG{l+m+mi}{0}\PYG{p}{,}\PYG{l+m+mi}{0}\PYG{p}{,}\PYG{l+m+mi}{0}\PYG{p}{,}\PYG{l+m+mi}{0}\PYG{p}{,}\PYG{l+m+mi}{0}\PYG{p}{,}\PYG{l+m+mi}{0}\PYG{p}{,}\PYG{l+m+mi}{0}\PYG{p}{,}\PYG{l+m+mi}{0}\PYG{p}{,}\PYG{l+m+mi}{0}\PYG{p}{,}\PYG{l+m+mi}{0}\PYG{p}{,}\PYG{l+m+mi}{1}
\PYG{l+m+mi}{1}\PYG{p}{,}\PYG{l+m+mi}{0}\PYG{p}{,}\PYG{l+m+mi}{1}\PYG{p}{,}\PYG{l+m+mi}{1}\PYG{p}{,}\PYG{l+m+mi}{1}\PYG{p}{,}\PYG{l+m+mi}{1}\PYG{p}{,}\PYG{l+m+mi}{1}\PYG{p}{,}\PYG{l+m+mi}{1}\PYG{p}{,}\PYG{l+m+mi}{1}\PYG{p}{,}\PYG{l+m+mi}{1}\PYG{p}{,}\PYG{l+m+mi}{0}\PYG{p}{,}\PYG{l+m+mi}{1}\PYG{p}{,}\PYG{l+m+mi}{1}\PYG{p}{,}\PYG{l+m+mi}{1}\PYG{p}{,}\PYG{l+m+mi}{1}\PYG{p}{,}\PYG{l+m+mi}{1}
\PYG{l+m+mi}{1}\PYG{p}{,}\PYG{l+m+mi}{0}\PYG{p}{,}\PYG{l+m+mi}{1}\PYG{p}{,}\PYG{l+m+mi}{0}\PYG{p}{,}\PYG{l+m+mi}{0}\PYG{p}{,}\PYG{l+m+mi}{0}\PYG{p}{,}\PYG{l+m+mi}{0}\PYG{p}{,}\PYG{l+m+mi}{0}\PYG{p}{,}\PYG{l+m+mi}{0}\PYG{p}{,}\PYG{l+m+mi}{1}\PYG{p}{,}\PYG{l+m+mi}{0}\PYG{p}{,}\PYG{l+m+mi}{0}\PYG{p}{,}\PYG{l+m+mi}{0}\PYG{p}{,}\PYG{l+m+mi}{0}\PYG{p}{,}\PYG{l+m+mi}{0}\PYG{p}{,}\PYG{l+m+mi}{1}
\PYG{l+m+mi}{1}\PYG{p}{,}\PYG{l+m+mi}{0}\PYG{p}{,}\PYG{l+m+mi}{1}\PYG{p}{,}\PYG{l+m+mi}{0}\PYG{p}{,}\PYG{l+m+mi}{1}\PYG{p}{,}\PYG{l+m+mi}{1}\PYG{p}{,}\PYG{l+m+mi}{1}\PYG{p}{,}\PYG{l+m+mi}{1}\PYG{p}{,}\PYG{l+m+mi}{0}\PYG{p}{,}\PYG{l+m+mi}{1}\PYG{p}{,}\PYG{l+m+mi}{1}\PYG{p}{,}\PYG{l+m+mi}{1}\PYG{p}{,}\PYG{l+m+mi}{1}\PYG{p}{,}\PYG{l+m+mi}{1}\PYG{p}{,}\PYG{l+m+mi}{0}\PYG{p}{,}\PYG{l+m+mi}{1}
\PYG{l+m+mi}{1}\PYG{p}{,}\PYG{l+m+mi}{0}\PYG{p}{,}\PYG{l+m+mi}{0}\PYG{p}{,}\PYG{l+m+mi}{0}\PYG{p}{,}\PYG{l+m+mi}{0}\PYG{p}{,}\PYG{l+m+mi}{0}\PYG{p}{,}\PYG{l+m+mi}{0}\PYG{p}{,}\PYG{l+m+mi}{0}\PYG{p}{,}\PYG{l+m+mi}{0}\PYG{p}{,}\PYG{l+m+mi}{1}\PYG{p}{,}\PYG{l+m+mi}{0}\PYG{p}{,}\PYG{l+m+mi}{0}\PYG{p}{,}\PYG{l+m+mi}{0}\PYG{p}{,}\PYG{l+m+mi}{0}\PYG{p}{,}\PYG{l+m+mi}{0}\PYG{p}{,}\PYG{l+m+mi}{1}
\PYG{l+m+mi}{1}\PYG{p}{,}\PYG{l+m+mi}{1}\PYG{p}{,}\PYG{l+m+mi}{1}\PYG{p}{,}\PYG{l+m+mi}{1}\PYG{p}{,}\PYG{l+m+mi}{1}\PYG{p}{,}\PYG{l+m+mi}{1}\PYG{p}{,}\PYG{l+m+mi}{1}\PYG{p}{,}\PYG{l+m+mi}{1}\PYG{p}{,}\PYG{l+m+mi}{1}\PYG{p}{,}\PYG{l+m+mi}{1}\PYG{p}{,}\PYG{l+m+mi}{0}\PYG{p}{,}\PYG{l+m+mi}{1}\PYG{p}{,}\PYG{l+m+mi}{1}\PYG{p}{,}\PYG{l+m+mi}{1}\PYG{p}{,}\PYG{l+m+mi}{1}\PYG{p}{,}\PYG{l+m+mi}{1}
\end{Verbatim}
\end{minipage}}
\begin{center}\setlength{\fboxsep}{5pt}\shadowbox{\box0}\end{center}
\setbox0\vbox{
\begin{minipage}{0.95\linewidth}
\textbf{Challenge}

\medskip

\begin{itemize}
\item {} 
make your own CSV datasheet

\item {} 
plant some random treasure

\end{itemize}
\end{minipage}}
\begin{center}\setlength{\fboxsep}{5pt}\shadowbox{\box0}\end{center}


\section{Pygame}
\label{kid/pygame::doc}\label{kid/pygame:pygame}

\subsection{List of pygame programs}
\label{kid/pygame:list-of-pygame-programs}

\subsubsection{Draw a circle}
\label{kid/pygame:draw-a-circle}
\begin{Verbatim}[commandchars=\\\{\},numbers=left,firstnumber=1,stepnumber=1]
\PYG{k+kn}{import} \PYG{n+nn}{pygame}

\PYG{n}{width}\PYG{p}{,}\PYG{n}{height} \PYG{o}{=} \PYG{l+m+mi}{640}\PYG{p}{,}\PYG{l+m+mi}{480}
\PYG{n}{radius} \PYG{o}{=} \PYG{l+m+mi}{100}
\PYG{n}{fill} \PYG{o}{=} \PYG{l+m+mi}{1}

\PYG{n}{pygame}\PYG{o}{.}\PYG{n}{init}\PYG{p}{(}\PYG{p}{)}
\PYG{n}{window} \PYG{o}{=} \PYG{n}{pygame}\PYG{o}{.}\PYG{n}{display}\PYG{o}{.}\PYG{n}{set\PYGZus{}mode}\PYG{p}{(}\PYG{p}{(}\PYG{n}{width}\PYG{p}{,}\PYG{n}{height}\PYG{p}{)}\PYG{p}{)}
\PYG{n}{window}\PYG{o}{.}\PYG{n}{fill}\PYG{p}{(}\PYG{n}{pygame}\PYG{o}{.}\PYG{n}{Color}\PYG{p}{(}\PYG{l+m+mi}{255}\PYG{p}{,}\PYG{l+m+mi}{255}\PYG{p}{,}\PYG{l+m+mi}{255}\PYG{p}{)}\PYG{p}{)} \PYG{c}{\PYGZsh{} white}

\PYG{k}{while} \PYG{n+nb+bp}{True}\PYG{p}{:}
    \PYG{n}{pygame}\PYG{o}{.}\PYG{n}{draw}\PYG{o}{.}\PYG{n}{circle}\PYG{p}{(}\PYG{n}{window}\PYG{p}{,}
                       \PYG{n}{pygame}\PYG{o}{.}\PYG{n}{Color}\PYG{p}{(}\PYG{l+m+mi}{255}\PYG{p}{,}\PYG{l+m+mi}{0}\PYG{p}{,}\PYG{l+m+mi}{0}\PYG{p}{)}\PYG{p}{,} \PYG{c}{\PYGZsh{} red}
                       \PYG{p}{(}\PYG{n}{width}\PYG{o}{/}\PYG{l+m+mi}{2}\PYG{p}{,}\PYG{n}{height}\PYG{o}{/}\PYG{l+m+mi}{2}\PYG{p}{)}\PYG{p}{,}
                       \PYG{n}{radius}\PYG{p}{,}
                       \PYG{n}{fill}\PYG{p}{)}
    \PYG{n}{pygame}\PYG{o}{.}\PYG{n}{display}\PYG{o}{.}\PYG{n}{update}\PYG{p}{(}\PYG{p}{)}
    \PYG{k}{if} \PYG{n}{pygame}\PYG{o}{.}\PYG{n}{QUIT} \PYG{o+ow}{in} \PYG{p}{[}\PYG{n}{e}\PYG{o}{.}\PYG{n}{type} \PYG{k}{for} \PYG{n}{e} \PYG{o+ow}{in} \PYG{n}{pygame}\PYG{o}{.}\PYG{n}{event}\PYG{o}{.}\PYG{n}{get}\PYG{p}{(}\PYG{p}{)}\PYG{p}{]}\PYG{p}{:}
        \PYG{k}{break}
\end{Verbatim}


\subsubsection{Draw circles based on mouse move / position}
\label{kid/pygame:draw-circles-based-on-mouse-move-position}
\begin{Verbatim}[commandchars=\\\{\},numbers=left,firstnumber=1,stepnumber=1]
\PYG{k+kn}{import} \PYG{n+nn}{pygame}
\PYG{k+kn}{from} \PYG{n+nn}{pygame.locals} \PYG{k+kn}{import} \PYG{o}{*}

\PYG{n}{width}\PYG{p}{,}\PYG{n}{height} \PYG{o}{=} \PYG{l+m+mi}{640}\PYG{p}{,}\PYG{l+m+mi}{640}
\PYG{n}{radius} \PYG{o}{=} \PYG{l+m+mi}{0}
\PYG{n}{fill} \PYG{o}{=} \PYG{l+m+mi}{1}
\PYG{n}{mouseX}\PYG{p}{,}\PYG{n}{mouseY} \PYG{o}{=} \PYG{l+m+mi}{0}\PYG{p}{,}\PYG{l+m+mi}{0}

\PYG{n}{pygame}\PYG{o}{.}\PYG{n}{init}\PYG{p}{(}\PYG{p}{)}
\PYG{n}{window} \PYG{o}{=} \PYG{n}{pygame}\PYG{o}{.}\PYG{n}{display}\PYG{o}{.}\PYG{n}{set\PYGZus{}mode}\PYG{p}{(}\PYG{p}{(}\PYG{n}{width}\PYG{p}{,}\PYG{n}{height}\PYG{p}{)}\PYG{p}{)}
\PYG{n}{window}\PYG{o}{.}\PYG{n}{fill}\PYG{p}{(}\PYG{n}{pygame}\PYG{o}{.}\PYG{n}{Color}\PYG{p}{(}\PYG{l+m+mi}{255}\PYG{p}{,}\PYG{l+m+mi}{255}\PYG{p}{,}\PYG{l+m+mi}{255}\PYG{p}{)}\PYG{p}{)} \PYG{c}{\PYGZsh{} white}
\PYG{n}{fps} \PYG{o}{=} \PYG{n}{pygame}\PYG{o}{.}\PYG{n}{time}\PYG{o}{.}\PYG{n}{Clock}\PYG{p}{(}\PYG{p}{)} \PYG{c}{\PYGZsh{} FPS = Frame Per Second}

\PYG{k}{while} \PYG{n+nb+bp}{True}\PYG{p}{:}  \PYG{c}{\PYGZsh{} one frame per loop}
    \PYG{k}{for} \PYG{n}{event} \PYG{o+ow}{in} \PYG{n}{pygame}\PYG{o}{.}\PYG{n}{event}\PYG{o}{.}\PYG{n}{get}\PYG{p}{(}\PYG{p}{)}\PYG{p}{:}
        \PYG{k}{if} \PYG{n}{event}\PYG{o}{.}\PYG{n}{type} \PYG{o}{==} \PYG{n}{MOUSEMOTION}\PYG{p}{:}
            \PYG{n}{mouseX}\PYG{p}{,}\PYG{n}{mouseY} \PYG{o}{=} \PYG{n}{event}\PYG{o}{.}\PYG{n}{pos}
        \PYG{k}{if} \PYG{n}{event}\PYG{o}{.}\PYG{n}{type} \PYG{o}{==} \PYG{n}{MOUSEBUTTONDOWN}\PYG{p}{:} \PYG{c}{\PYGZsh{} mouse click}
            \PYG{n}{window}\PYG{o}{.}\PYG{n}{fill}\PYG{p}{(}\PYG{n}{pygame}\PYG{o}{.}\PYG{n}{Color}\PYG{p}{(}\PYG{l+m+mi}{255}\PYG{p}{,}\PYG{l+m+mi}{255}\PYG{p}{,}\PYG{l+m+mi}{255}\PYG{p}{)}\PYG{p}{)} \PYG{c}{\PYGZsh{} clear screen}
        \PYG{n}{radius} \PYG{o}{=} \PYG{p}{(}\PYG{n+nb}{abs}\PYG{p}{(}\PYG{n}{width}\PYG{o}{/}\PYG{l+m+mi}{2} \PYG{o}{\PYGZhy{}} \PYG{n}{mouseX}\PYG{p}{)} \PYG{o}{+} \PYG{n+nb}{abs}\PYG{p}{(}\PYG{n}{height}\PYG{o}{/}\PYG{l+m+mi}{2} \PYG{o}{\PYGZhy{}} \PYG{n}{mouseY}\PYG{p}{)}\PYG{p}{)}\PYG{o}{/}\PYG{l+m+mi}{2} \PYG{o}{+} \PYG{l+m+mi}{1}
        \PYG{n}{pygame}\PYG{o}{.}\PYG{n}{draw}\PYG{o}{.}\PYG{n}{circle}\PYG{p}{(}\PYG{n}{window}\PYG{p}{,}
                           \PYG{n}{pygame}\PYG{o}{.}\PYG{n}{Color}\PYG{p}{(}\PYG{l+m+mi}{255}\PYG{p}{,}\PYG{l+m+mi}{0}\PYG{p}{,}\PYG{l+m+mi}{0}\PYG{p}{)}\PYG{p}{,} \PYG{c}{\PYGZsh{} red}
                           \PYG{p}{(}\PYG{n}{mouseX}\PYG{p}{,}\PYG{n}{mouseY}\PYG{p}{)}\PYG{p}{,}
                           \PYG{n}{radius}\PYG{p}{,}
                           \PYG{n}{fill}\PYG{p}{)}
    \PYG{n}{pygame}\PYG{o}{.}\PYG{n}{display}\PYG{o}{.}\PYG{n}{update}\PYG{p}{(}\PYG{p}{)}
    \PYG{k}{if} \PYG{n}{pygame}\PYG{o}{.}\PYG{n}{QUIT} \PYG{o+ow}{in} \PYG{p}{[}\PYG{n}{e}\PYG{o}{.}\PYG{n}{type} \PYG{k}{for} \PYG{n}{e} \PYG{o+ow}{in} \PYG{n}{pygame}\PYG{o}{.}\PYG{n}{event}\PYG{o}{.}\PYG{n}{get}\PYG{p}{(}\PYG{p}{)}\PYG{p}{]}\PYG{p}{:}
        \PYG{k}{break}
    \PYG{n}{fps}\PYG{o}{.}\PYG{n}{tick}\PYG{p}{(}\PYG{l+m+mi}{30}\PYG{p}{)} \PYG{c}{\PYGZsh{} wait so that frame rate is 30 fps}
\end{Verbatim}


\section{Scratch}
\label{kid/scratch:scratch}\label{kid/scratch::doc}

\chapter{Hardware}
\label{index:hardware}

\section{Raspberry Pi}
\label{hardware/raspberrypi::doc}\label{hardware/raspberrypi:raspberry-pi}

\subsection{Default settings}
\label{hardware/raspberrypi:default-settings}
\begin{tabulary}{\linewidth}{|L|L|}
\hline

login
 & 
\textbf{pi}
\\
\hline
password
 & 
\textbf{raspberry}
\\
\hline
hostname
 & 
\textbf{raspberrypi}
\\
\hline
keyboard
 & 
UK
\\
\hline\end{tabulary}



\subsection{Basic commands}
\label{hardware/raspberrypi:basic-commands}

\subsubsection{Config}
\label{hardware/raspberrypi:config}
\begin{Verbatim}[commandchars=\\\{\}]
\PYG{n+nv}{\PYGZdl{} }sudo raspi\PYGZhy{}config
\end{Verbatim}


\subsubsection{Start X server}
\label{hardware/raspberrypi:start-x-server}
\begin{Verbatim}[commandchars=\\\{\}]
\PYG{n+nv}{\PYGZdl{} }startx
\end{Verbatim}


\subsubsection{Reboot}
\label{hardware/raspberrypi:reboot}
\begin{Verbatim}[commandchars=\\\{\}]
\PYG{n+nv}{\PYGZdl{} }sudo reboot
\end{Verbatim}


\subsubsection{Shutdown}
\label{hardware/raspberrypi:shutdown}
\begin{Verbatim}[commandchars=\\\{\}]
\PYG{n+nv}{\PYGZdl{} }sudo shutdown \PYGZhy{}h now
\end{Verbatim}


\subsubsection{Change datetime}
\label{hardware/raspberrypi:change-datetime}
\begin{Verbatim}[commandchars=\\\{\}]
\PYG{n+nv}{\PYGZdl{} }sudo date \PYGZhy{}\PYGZhy{}set\PYG{o}{=}\PYG{l+s+s2}{\PYGZdq{}Sun Nov 18 1:55:16 EDT 2012\PYGZdq{}}
\end{Verbatim}


\subsubsection{Update}
\label{hardware/raspberrypi:update}
\begin{Verbatim}[commandchars=\\\{\}]
\PYG{n+nv}{\PYGZdl{} }sudo apt\PYGZhy{}get update
\PYG{n+nv}{\PYGZdl{} }sudo apt\PYGZhy{}get upgrade
\end{Verbatim}


\subsection{Information}
\label{hardware/raspberrypi:information}

\subsubsection{Check OS version}
\label{hardware/raspberrypi:check-os-version}
\begin{Verbatim}[commandchars=\\\{\}]
\PYG{n+nv}{\PYGZdl{} }cat /proc/version
\end{Verbatim}


\subsubsection{Check board version}
\label{hardware/raspberrypi:check-board-version}
\begin{Verbatim}[commandchars=\\\{\}]
\PYG{n+nv}{\PYGZdl{} }cat /proc/cpuinfo
\end{Verbatim}


\subsubsection{Display network interface and associated IP addresses}
\label{hardware/raspberrypi:display-network-interface-and-associated-ip-addresses}
\begin{Verbatim}[commandchars=\\\{\}]
\PYG{n+nv}{\PYGZdl{} }ifconfig
\end{Verbatim}


\subsection{Short-cuts}
\label{hardware/raspberrypi:short-cuts}
\begin{tabulary}{\linewidth}{|L|L|}
\hline

\textbf{Ctrl} + \textbf{C}
 & 
kill currently running program
\\
\hline
\textbf{Ctrl} + \textbf{D}
 & 
exit shell
\\
\hline
\textbf{Ctrl} + \textbf{A}
 & 
move cursor to the beginning of the line
\\
\hline
\textbf{Ctrl} + \textbf{E}
 & 
move cursor to the end of the line
\\
\hline
\textbf{Ctrl} + \textbf{Alt} + \textbf{Backspace}
 & 
{[}optional{]} terminate the X server
\\
\hline\end{tabulary}



\subsection{Setup Keyboard}
\label{hardware/raspberrypi:setup-keyboard}
The default keyboard is UK. Let's change it to AU keyboard.

The trick is that Australia is not listed in the country list for the keyboard, we need to setup a US keyboard instead.


\subsubsection{Change the keyboard config}
\label{hardware/raspberrypi:change-the-keyboard-config}
\begin{Verbatim}[commandchars=\\\{\}]
\PYG{n+nv}{\PYGZdl{} }sudo vi /etc/default/keyboard
\end{Verbatim}

\begin{Verbatim}[commandchars=\\\{\}]
XKBMODEL =\PYGZdq{}pc105\PYGZdq{}
XKBLAYOUT=\PYGZdq{}us\PYGZdq{}
XKBVARIANT=\PYGZdq{}\PYGZdq{}
XKBOPTIONS=\PYGZdq{}\PYGZdq{}

BACKSPACE=\PYGZdq{}guess\PYGZdq{}
\end{Verbatim}


\subsubsection{Then run the following commands and reboot}
\label{hardware/raspberrypi:then-run-the-following-commands-and-reboot}
\begin{Verbatim}[commandchars=\\\{\}]
\PYG{n+nv}{\PYGZdl{} }sudo setxkbmap \PYGZhy{}layout us
\PYG{n+nv}{\PYGZdl{} }sudo udevadm trigger \PYGZhy{}\PYGZhy{}subsysstem\PYGZhy{}match\PYG{o}{=}input \PYGZhy{}\PYGZhy{}action\PYG{o}{=}change
\end{Verbatim}


\subsection{Utilities / Softwares}
\label{hardware/raspberrypi:utilities-softwares}

\subsubsection{raspi-config tool}
\label{hardware/raspberrypi:raspi-config-tool}
\begin{Verbatim}[commandchars=\\\{\}]
\PYG{n+nv}{\PYGZdl{} }sudo apt\PYGZhy{}get install raspi\PYGZhy{}config
\end{Verbatim}


\subsubsection{Minecraft}
\label{hardware/raspberrypi:minecraft}
\begin{Verbatim}[commandchars=\\\{\}]
\PYG{n+nv}{\PYGZdl{} }sudo apt\PYGZhy{}get install minecraft\PYGZhy{}pi
\end{Verbatim}


\subsubsection{Screenshot : scrot}
\label{hardware/raspberrypi:screenshot-scrot}
\begin{Verbatim}[commandchars=\\\{\}]
\PYG{n+nv}{\PYGZdl{} }sudo apt\PYGZhy{}get install scrot
\end{Verbatim}


\subsubsection{Mercurial}
\label{hardware/raspberrypi:mercurial}
\begin{Verbatim}[commandchars=\\\{\}]
\PYG{n+nv}{\PYGZdl{} }sudo apt\PYGZhy{}get install mercurial
\end{Verbatim}


\section{Arduino}
\label{hardware/arduino::doc}\label{hardware/arduino:arduino}

\chapter{Programming language}
\label{index:programming-language}

\section{Shell}
\label{language/shell:shell}\label{language/shell::doc}

\section{Python}
\label{language/python:python}\label{language/python::doc}

\chapter{Editor}
\label{index:editor}

\section{Vi}
\label{editor/vi:vi}\label{editor/vi::doc}

\subsection{Cursor Movement Commands}
\label{editor/vi:cursor-movement-commands}
\code{(n)} indicates a number, and is optional

\begin{tabulary}{\linewidth}{|L|L|}
\hline

(n)h
 & 
left (n) space(s)
\\
\hline
(n)j
 & 
down (n) space(s)
\\
\hline
(n)k
 & 
up (n) space(s)
\\
\hline
(n)l
 & 
right (n) space(s)
\\
\hline\end{tabulary}


(The arrow keys usually work also)

\begin{tabulary}{\linewidth}{|L|L|}
\hline

CTRL + F
 & 
forward one screen
\\
\hline
CTRL + B
 & 
back one screen
\\
\hline
CTRL + D
 & 
down half screen
\\
\hline
CTRL + U
 & 
up half screen
\\
\hline
H
 & 
beginning of top line of screen
\\
\hline
M
 & 
beginning of middle line of screen
\\
\hline
L
 & 
beginning of last line of screen
\\
\hline
G
 & 
beginning of last line of file
\\
\hline
(n)G
 & 
move to beginning of line (n)
\\
\hline
0
 & 
(zero) beginning of line
\\
\hline
\$
 & 
end of line
\\
\hline
(n)w
 & 
forward (n) word(s)
\\
\hline
(n)b
 & 
back (n) word(s)
\\
\hline
e
 & 
end of word
\\
\hline\end{tabulary}



\subsection{Inserting Text}
\label{editor/vi:inserting-text}
\begin{tabulary}{\linewidth}{|L|L|}
\hline

i
 & 
insert text before the cursor
\\
\hline
a
 & 
append text after the cursor (does not overwrite other text)
\\
\hline
I
 & 
insert text at the beginning of the line
\\
\hline
A
 & 
append text to the end of the line
\\
\hline
r
 & 
replace the character under the cursor with the next character typed
\\
\hline
R
 & 
Overwrite characters until the end of the line (or until escape is pressed to change command)
\\
\hline
o
 & 
(alpha o) open new line after the current line to type text
\\
\hline
O
 & 
(alpha O) open new line before the current line to type text
\\
\hline\end{tabulary}



\subsection{Deleting Text}
\label{editor/vi:deleting-text}
\begin{tabulary}{\linewidth}{|L|L|}
\hline

dd
 & 
deletes current line
\\
\hline
(n)dd
 & 
deletes (n) line(s)
\\
\hline
(n)dw
 & 
deletes (n) word(s)
\\
\hline
D
 & 
deletes from cursor to end of line
\\
\hline
x
 & 
deletes current character
\\
\hline
(n)x
 & 
deletes (n) character(s)
\\
\hline
X
 & 
deletes previous character
\\
\hline\end{tabulary}



\subsection{Change Commands}
\label{editor/vi:change-commands}
\begin{tabulary}{\linewidth}{|L|L|}
\hline

(n)cc
 & 
changes (n) characters on line(s) until end of the line (or until escape is pressed)
\\
\hline
cw
 & 
changes characters of word until end of the word (or until escape is pressed)
\\
\hline
(n)cw
 & 
changes characters of the next (n) words
\\
\hline
c\$
 & 
changes text to the end of the line
\\
\hline
ct(x)
 & 
changes text to the letter (x)
\\
\hline
C
 & 
changes remaining text on the current line (until stopped by escape key)
\\
\hline
\textasciitilde{}
 & 
changes the case of the current character
\\
\hline
J
 & 
joins the current line and the next line
\\
\hline
u
 & 
undo the last command just done on this line
\\
\hline
.
 & 
repeats last change
\\
\hline
s
 & 
substitutes text for current character
\\
\hline
S
 & 
substitutes text for current line
\\
\hline
:s
 & 
substitutes new word(s) for old
\code{:\textless{}line nos effected\textgreater{} s/old/new/g}
\\
\hline
\&
 & 
repeats last substitution (:s) command
\\
\hline
(n)yy
 & 
yanks (n) lines to buffer
\\
\hline
y(n)w
 & 
yanks (n) words to buffer
\\
\hline
p
 & 
puts yanked or deleted text after cursor
\\
\hline
P
 & 
puts yanked or deleted text before cursor
\\
\hline\end{tabulary}



\subsection{File Manipulation}
\label{editor/vi:file-manipulation}
\begin{tabulary}{\linewidth}{|L|L|}
\hline

:w (file)
 & 
writes changes to file (default is current file)
\\
\hline
:wq
 & 
writes changes to current file and quits edit session
\\
\hline
:w! (file)
 & 
overwrites file (default is current file)
\\
\hline
:q
 & 
quits edit session w/no changes made
\\
\hline
:q!
 & 
quits edit session and discards changes
\\
\hline
:n
 & 
edits next file in argument list
\\
\hline
:f (name)
 & 
changes name of current file to (name)
\\
\hline
:r (file)
 & 
reads contents of file into current edit at the current cursor position (insert a file)
\\
\hline
:!(command)
 & 
shell escape
\\
\hline
:r!(command)
 & 
inserts result of shell command at cursor position
\\
\hline
ZZ
 & 
write changes to current file and exit
\\
\hline\end{tabulary}



\section{VIM (Vi IMproved)}
\label{editor/vim:vim-vi-improved}\label{editor/vim::doc}

\subsection{Basic commands}
\label{editor/vim:basic-commands}

\subsubsection{Read only (use \textbf{:wq!} to force the modification)}
\label{editor/vim:read-only-use-wq-to-force-the-modification}
\begin{Verbatim}[commandchars=\\\{\}]
\PYG{n+nv}{\PYGZdl{} }vim \PYGZhy{}R file
\end{Verbatim}


\subsubsection{Running shell commands}
\label{editor/vim:running-shell-commands}
\begin{Verbatim}[commandchars=\\\{\}]
!command
\end{Verbatim}

e.g. \textbf{!ls} will launch \textbf{ls}

if you wants to go directly to shell without quitting from VI editor you can go by executing \textbf{!sh} / \textbf{!bash} / \textbf{!ksh} from VI and then come back to VI editor by just executing command \textbf{exit} from shell.
for Cygwin, \textbf{!bash} and \textbf{exit} seems to be the best choice


\subsubsection{Launch VIM from command line}
\label{editor/vim:launch-vim-from-command-line}
\begin{Verbatim}[commandchars=\\\{\}]
\PYG{n+nv}{\PYGZdl{} }vi file.txt                        open and edit file file.txt
\PYG{n+nv}{\PYGZdl{} }vi file1.txt file2.txt file3.txt   open several files
\PYG{n+nv}{\PYGZdl{} }vi +25 file.txt                    edit from the 25th line
\PYG{n+nv}{\PYGZdl{} }vi + file.txt                      edit at the end of file
\PYG{n+nv}{\PYGZdl{} }vi +/text file.txt                 edit from the first line containing the word \PYG{n+nb}{test}
\PYG{n+nv}{\PYGZdl{} }vi \PYGZhy{}r file.txt                     restore a crashed file
\PYG{n+nv}{\PYGZdl{} }view file.txt                      vi in \PYG{n+nb}{read}\PYGZhy{}only mode
\PYG{n+nv}{\PYGZdl{} }vimtutor                           VIM tutorial
\end{Verbatim}


\subsubsection{Saving and quiting commands}
\label{editor/vim:saving-and-quiting-commands}
\begin{Verbatim}[commandchars=\\\{\}]
:w            save the current file (before quit)
:w file.txt   save the modified file with another file name
              (even if the file was opened in read\PYGZhy{}only mode)
:wq           save and quit
ZZ            save and quit
:q!           quit without saving
:wq!          save change in the current file opened in read\PYGZhy{}only mode, and then quit
:w!           save change in the current file opened in read\PYGZhy{}only mode
\end{Verbatim}


\subsubsection{Checking history and help}
\label{editor/vim:checking-history-and-help}
\begin{Verbatim}[commandchars=\\\{\}]
:history        vim commands history
:help           all helps
:help command   help on one command
\end{Verbatim}


\subsubsection{Recording and replaying commands}
\label{editor/vim:recording-and-replaying-commands}
Recoding in vim or VI editor can be done by using \textbf{q} and the executing recorded comment by using \textbf{q@1}


\subsection{Options}
\label{editor/vim:options}
Here are the major VIM editor options

\begin{Verbatim}[commandchars=\\\{\}]
:set nu           This will display line number in front of each line quite useful if
                  you want line by line information. You can turn it off by executing
                  \PYGZdq{}set nonu\PYGZdq{}. Remember for turning it off put \PYGZdq{}no\PYGZdq{} in front of option,
                  like here option is \PYGZdq{}nu\PYGZdq{} so for turning it off use \PYGZdq{}nonu\PYGZdq{}.
:set nonu         removing line number display
:set hlsearch     This will highlight the matching word when we do search in VI editor,
                  quite useful but if you find it annoying or not able to see sometime
                  due to your color scheme you can turn it off by executing \PYGZdq{}set
                  nohlsearch\PYGZdq{}.
:set wrap         If your file has contains some long lines and you want them to wrap
                  use this option, if its already on and you just don\PYGZsq{}t want them to
                  wrap use \PYGZdq{}set nowrap\PYGZdq{}.
:colorscheme      color scheme is used to change color of VIM editor, my favorite color
                  scheme is murphy so if you want to change color scheme of VI editor
                  you can do by executing \PYGZdq{}colorscheme murphy\PYGZdq{}.
:syntax on        syntax can be turn on and off based on your need, if it\PYGZsq{}s on it will
                  display color syntax for .xml, .html and .perl files.
:set ignorecase   This VI editor option allows you do case insensitive search because
                  if it\PYGZsq{}s set VI will not distinguish between two words which are just
                  differ in case.
:set smartcase    Another VI editor option which allows case\PYGZhy{}sensitive search if the
                  word you are searching contains an uppercase character.
\end{Verbatim}


\subsection{Navigation}
\label{editor/vim:navigation}
Here are some navigating commands

\begin{tabulary}{\linewidth}{|L|L|}
\hline

gg
 & 
goes to start of file
\\
\hline
SHIFT g
 & 
goes to end of file
\\
\hline
0
 & 
goes to beginning of the line
\\
\hline
\$
 & 
goes to end of the line
\\
\hline
nG
 & 
goes to nth line
\\
\hline
:n
 & 
another way of going to nth line
\\
\hline\end{tabulary}



\subsection{Editing}
\label{editor/vim:editing}

\subsubsection{Editing commands}
\label{editor/vim:editing-commands}
\begin{tabulary}{\linewidth}{|L|L|}
\hline

yy
 & 
equivalent to cut also called yank
\\
\hline
p
 & 
paste below line
\\
\hline
SHIFT p
 & 
paste above line
\\
\hline
dd
 & 
deletes the current line
\\
\hline
5dd
 & 
deletes 5 lines
\\
\hline
u
 & 
undo last change
\\
\hline
CTRL + R
 & 
Re do last change
\\
\hline\end{tabulary}



\subsubsection{Copy (or cut) / paste (without strange indent)}
\label{editor/vim:copy-or-cut-paste-without-strange-indent}\begin{enumerate}
\item {} 
move the mouse pointer to the beginning of your desired copy text

\item {} 
type `v' (visual) for Visual mode, then using mouse pointer move to the end of selected text

\item {} 
type `y' (yank) for Copy or `d' (delete) for Cut

\item {} 
move to your paste location, then type `p' (paste)

\end{enumerate}


\subsubsection{Tabulation}
\label{editor/vim:tabulation}\begin{enumerate}
\item {} 
define TAB as 2 spaces

\end{enumerate}

\begin{Verbatim}[commandchars=\\\{\}]
:set tabstop=2 shiftwidth=2 expandtab
\end{Verbatim}
\begin{enumerate}
\setcounter{enumi}{1}
\item {} 
replace TAB by 4 spaces

\end{enumerate}

\begin{Verbatim}[commandchars=\\\{\}]
:\PYGZpc{}s/\PYGZbs{}t/    /g
\end{Verbatim}


\subsection{Multi-files, multi-windows}
\label{editor/vim:multi-files-multi-windows}

\subsubsection{Opening multi-files / another file}
\label{editor/vim:opening-multi-files-another-file}
\begin{Verbatim}[commandchars=\\\{\}]
\PYG{n+nv}{\PYGZdl{} }vim file1 file2 file3 ...
\end{Verbatim}

\begin{Verbatim}[commandchars=\\\{\}]
:n        edit next file among multi\PYGZhy{}files (with respect to the order given in the
          command line)
:wn       save the modification and edit the next file
:n!       edit the next file without saving the ongoing modification
:e        reload the current file
:e file   load file in the current window
\end{Verbatim}


\subsubsection{Multi-windows}
\label{editor/vim:multi-windows}
\begin{Verbatim}[commandchars=\\\{\}]
:sp(lit) file   split horizontally the window and load file in the splitted window
:vsplit file    split vertically the window and load file in the splitted window
:vs             vertically split window
CTRL + w + w    switch among all (sub\PYGZhy{})windows
:q              close the current (sub\PYGZhy{})window
\end{Verbatim}


\subsection{Search and Replace}
\label{editor/vim:search-and-replace}

\subsubsection{Searching commands}
\label{editor/vim:searching-commands}
\begin{Verbatim}[commandchars=\\\{\}]
/Exception   will search for word \PYGZdq{}Exception\PYGZdq{} from top to bottom and stop when it got
             first match, to go to next match type \PYGZdq{}n\PYGZdq{} and for coming back to previous
             match press \PYGZdq{}Shift + N\PYGZdq{}
?Exception   will search for word \PYGZdq{}Exception\PYGZdq{} from bottom to top and stop when it got
             first match, to go to next match type \PYGZdq{}n\PYGZdq{} and for coming back to previous
             match press \PYGZdq{}Shift + N\PYGZdq{}, remember for next match it will go towards top
             of file.
\end{Verbatim}


\subsubsection{Find and replace}
\label{editor/vim:find-and-replace}
\begin{Verbatim}[commandchars=\\\{\}]
:\PYGZpc{}s/Old/New/g     This is an example of global search it will replace all occurrence of
                  word \PYGZdq{}Old\PYGZdq{} in file by \PYGZdq{}New\PYGZdq{}. It\PYGZsq{}s also equivalent to following
                  command \PYGZdq{}: 0,\PYGZdl{} s/Old/New/g\PYGZdq{} which actually tells that search from
                  first to last line.
:\PYGZpc{}s/Old/New/gc    This is similar to first command but with the introduction of \PYGZdq{}c\PYGZdq{}; it
                  will ask for confirmation
:\PYGZpc{}s/Old/New/gci   This is command is global, case insensitive and ask for confirmation;
                  to make it case Sensitive use \PYGZdq{}I\PYGZdq{}
\end{Verbatim}


\subsubsection{Substitution}
\label{editor/vim:substitution}
Substitution is very useful when working with text. Below you have some example. For more information, you could check the link : \href{http://vim.wikia.com/wiki/Search\_and\_replace}{http://vim.wikia.com/wiki/Search\_and\_replace}

\begin{Verbatim}[commandchars=\\\{\}]
:s/abc/def/           change the first \PYGZsq{}abc\PYGZsq{} of the line to \PYGZsq{}def\PYGZsq{}
:s/abc/def/g          change all \PYGZsq{}abc\PYGZsq{} of the line to \PYGZsq{}def\PYGZsq{}
:\PYGZpc{}s/abc/def/g         change all \PYGZsq{}abc\PYGZsq{} of all lines to \PYGZsq{}def\PYGZsq{}
:\PYGZpc{}s/\PYGZbs{}\PYGZlt{}abc\PYGZbs{}\PYGZgt{}/def/g     change all words \PYGZsq{}abc\PYGZsq{} of all lines to \PYGZsq{}def\PYGZsq{}
:\PYGZpc{}s/\PYGZbs{}\PYGZlt{}abc\PYGZbs{}\PYGZgt{}/def/gI    change all words \PYGZsq{}abc\PYGZsq{} (case sensitive) of all lines to \PYGZsq{}def\PYGZsq{}
:\PYGZpc{}s/\PYGZbs{}\PYGZlt{}abc\PYGZbs{}\PYGZgt{}/def/gci   change all words \PYGZsq{}abc\PYGZsq{} (case insensitive) of all lines to \PYGZsq{}def\PYGZsq{},
                      ask for confirmation
:5,10s/abc/def/g      change all \PYGZsq{}abc\PYGZsq{} to \PYGZsq{}def\PYGZsq{}, from line 5 to line 10 inclusive
:.,+5s/abc/def/g      change all \PYGZsq{}abc\PYGZsq{} to \PYGZsq{}def\PYGZsq{}, for the current line and the 5 next
                      lines
:.,\PYGZdl{}s/abc/def/g       change all \PYGZsq{}abc\PYGZsq{} to \PYGZsq{}def\PYGZsq{}, from the current line to the last line
:g/\PYGZca{}a/s/abc/def/g     change all \PYGZsq{}abc\PYGZsq{} to \PYGZsq{}def\PYGZsq{}, for each line starting with \PYGZsq{}a\PYGZsq{}
\end{Verbatim}


\section{JOE (Joe’s Own Editor)}
\label{editor/joe:joe-joes-own-editor}\label{editor/joe::doc}

\subsection{Basic commands}
\label{editor/joe:basic-commands}

\subsubsection{Launch JOE from command line}
\label{editor/joe:launch-joe-from-command-line}
\begin{Verbatim}[commandchars=\\\{\}]
\PYG{n+nv}{\PYGZdl{} }joe file.txt                     open and edit file.txt
\PYG{n+nv}{\PYGZdl{} }joe \PYGZhy{}wordwrap file.txt           option wordwrap
\PYG{n+nv}{\PYGZdl{} }joe \PYGZhy{}lmargin \PYG{l+m}{5} \PYGZhy{}tab \PYG{l+m}{5} file.txt   left \PYG{n+nv}{margin} \PYG{o}{=} \PYG{l+m}{5} chars and \PYG{n+nv}{TAB} \PYG{o}{=} \PYG{l+m}{5} chars
\PYG{n+nv}{\PYGZdl{} }joe +25 file.txt                 edit from 25th line
\PYG{n+nv}{\PYGZdl{} }jmacs file.txt                   variant : simulate GNU\PYGZhy{}EMACS
\PYG{n+nv}{\PYGZdl{} }jstar file.txt                   variant : simulate WordStar
\PYG{n+nv}{\PYGZdl{} }jpico file.txt                   variant : simulate the Pine mailer editor PICO
\PYG{n+nv}{\PYGZdl{} }rjoe file.txt                    variant : restraint the edit to the file file.txt
                                   only
\end{Verbatim}


\subsubsection{Saving and quiting commands}
\label{editor/joe:saving-and-quiting-commands}
\begin{tabulary}{\linewidth}{|L|L|}
\hline

CTRL + k + d
 & 
save the file
\\
\hline
CTRL + k + x
 & 
save and exit
\\
\hline
CTRL + c
 & 
exit without save
\\
\hline
CTRL + k + z
 & 
exit and leave JOE in background (fg to go back)
\\
\hline\end{tabulary}



\subsubsection{Orthographe}
\label{editor/joe:orthographe}
\begin{tabulary}{\linewidth}{|L|L|}
\hline

CTRL + {[} + n
 & 
check one word
\\
\hline
CTRL + {[} + l
 & 
check one file
\\
\hline\end{tabulary}



\subsubsection{Misc}
\label{editor/joe:misc}
\begin{tabulary}{\linewidth}{|L|L|}
\hline

CTRL + k + a
 & 
move to the middle
\\
\hline
CTRL + t
 & 
display and choose the options
\\
\hline
CTRL + r
 & 
refresh the display
\\
\hline
CTRL + k + h
 & 
display or close the online help
\\
\hline\end{tabulary}



\subsection{Navigation}
\label{editor/joe:navigation}

\subsubsection{Cursor / Move}
\label{editor/joe:cursor-move}
\begin{tabulary}{\linewidth}{|L|L|}
\hline

CTRL + b
 & 
move to left
\\
\hline
CTRL + p
 & 
move to top
\\
\hline
CTRL + f
 & 
move to right
\\
\hline
CTRL + n
 & 
move to down
\\
\hline
CTRL + z
 & 
move to the previous word
\\
\hline
CTRL + x
 & 
move to the next word
\\
\hline\end{tabulary}



\subsubsection{Navigation}
\label{editor/joe:id1}
\begin{tabulary}{\linewidth}{|L|L|}
\hline

CTRL + u
 & 
previous screen
\\
\hline
CTRL + v
 & 
next screen
\\
\hline
CTRL + a
 & 
beginning of the line
\\
\hline
CTRL + e
 & 
end of the line
\\
\hline
CTRL + k + u
 & 
beginning of the file
\\
\hline
CTRL + k + v
 & 
end of the file
\\
\hline
CTRL + k + l
 & 
go to line n
\\
\hline\end{tabulary}



\subsection{Editing}
\label{editor/joe:editing}

\subsubsection{Blocs operations}
\label{editor/joe:blocs-operations}
\begin{tabulary}{\linewidth}{|L|L|}
\hline

CTRL + k + b
 & 
beginning of the bloc
\\
\hline
CTRL + k + k
 & 
end of the bloc
\\
\hline
CTRL + k + m
 & 
move of the bloc
\\
\hline
CTRL + k + c
 & 
copy the bloc
\\
\hline
CTRL + k + w
 & 
write the bloc in a file
\\
\hline
CTRL + k + y
 & 
delete the bloc
\\
\hline
CTRL + k + /
 & 
filter the bloc
\\
\hline\end{tabulary}



\subsubsection{Deletion}
\label{editor/joe:deletion}
\begin{tabulary}{\linewidth}{|L|L|}
\hline

CTRL + d
 & 
delete one character
\\
\hline
CTRL + y
 & 
delete one line
\\
\hline
CTRL + w
 & 
delete one word on the right of the cursor
\\
\hline
CTRL + o
 & 
delete one word on the left of the cursor
\\
\hline
CTRL + j
 & 
delete the rest of the line (i.e. the right side of the cursor)
\\
\hline
CTRL + \_
 & 
cancel the operation
\\
\hline
CTRL + 6
 & 
redo the cancelled operation
\\
\hline\end{tabulary}



\subsubsection{Files}
\label{editor/joe:files}
\begin{tabulary}{\linewidth}{|L|L|}
\hline

CTRL + k + e
 & 
open / edit a new file
\\
\hline
CTRL + k + r
 & 
insert one file at the cursor position
\\
\hline\end{tabulary}



\subsection{Search}
\label{editor/joe:search}
\begin{tabulary}{\linewidth}{|L|L|}
\hline

CTRL + k + f
 & 
search one text
\\
\hline
CTRL + l
 & 
search the next
\\
\hline\end{tabulary}



\section{NANO (Nano’s ANOther editor)}
\label{editor/nano::doc}\label{editor/nano:nano-nanos-another-editor}

\subsection{Basic commands}
\label{editor/nano:basic-commands}
NANO is the open source clone of the editor PICO, distributed as part of the mail client Pine.


\subsubsection{Launch NANO from command line}
\label{editor/nano:launch-nano-from-command-line}
\begin{Verbatim}[commandchars=\\\{\}]
\PYG{n+nv}{\PYGZdl{} }nano file.txt       open and edit file.txt
\PYG{n+nv}{\PYGZdl{} }nano \PYGZhy{}B file.txt    save the original file as file.txt\PYGZti{} or \PYGZti{}.file
\PYG{n+nv}{\PYGZdl{} }nano \PYGZhy{}m file.txt    activate the mouse cursor \PYG{o}{(}\PYG{k}{if} supported\PYG{o}{)}
\PYG{n+nv}{\PYGZdl{} }nano +25 file.txt   edit from the 25th line
\PYG{n+nv}{\PYGZdl{} }jpico file.txt      simulator JOE of PICO
\end{Verbatim}


\subsubsection{Short-cuts Fn}
\label{editor/nano:short-cuts-fn}
\begin{tabulary}{\linewidth}{|L|L|L|}
\hline

F1
 & 
CTRL + g
 & 
display online help (CTRL + x to quit)
\\
\hline
F2
 & 
CTRL + x
 & 
quit NANO (or close ongoing buffer)
\\
\hline
F3
 & 
CTRL + o
 & 
save ongoing file
\\
\hline
F4
 & 
CTRL + j
 & 
reformat the text of paragraph
\\
\hline
F5
 & 
CTRL + r
 & 
insert one file
\\
\hline
F6
 & 
CTRL + w
 & 
search one text
\\
\hline
F7
 & 
CTRL + y
 & 
previous screen
\\
\hline
F8
 & 
CTRL + v
 & 
next screen
\\
\hline
F9
 & 
CTRL + k
 & 
cut (and copy) the line (or the chosen text)
\\
\hline
F10
 & 
CTRL + u
 & 
paste the cut text
\\
\hline
F11
 & 
CTRL + c
 & 
display the cursor position
\\
\hline
F12
 & 
CTRL + t
 & 
start the orthograph verification
\\
\hline\end{tabulary}



\subsubsection{Misc}
\label{editor/nano:misc}
\begin{tabulary}{\linewidth}{|L|L|}
\hline

CTRL + 6
 & 
choose one text from the cursor (CTRL + 6 to cancel the action)
\\
\hline\end{tabulary}



\subsection{Navigation}
\label{editor/nano:navigation}
\begin{tabulary}{\linewidth}{|L|L|}
\hline

CTRL + \_
 & 
go to line n and column m
\\
\hline
CTRL + f
 & 
move to left
\\
\hline
CTRL + b
 & 
move to right
\\
\hline
CTRL + SPACE
 & 
move to the previous word
\\
\hline
CTRL + p
 & 
previous line
\\
\hline
CTRL + n
 & 
next line
\\
\hline
CTRL + a
 & 
beginning of the line
\\
\hline
CTRL + e
 & 
end of the line
\\
\hline\end{tabulary}


\code{META} does not exist on most of the recent keyboards. We list the commands below just for reference.

Sometimes you could mimic a command \code{META} + \code{s} (toggle smooth scrolling mode on and off) as follows:
\begin{itemize}
\item {} 
press \code{Esc} key

\item {} 
release \code{Esc} key

\item {} 
press \code{s} key

\end{itemize}

\begin{tabulary}{\linewidth}{|L|L|}
\hline

META + SPACE
 & 
move to the next word
\\
\hline
META + (
 & 
beginning of the paragraph
\\
\hline
META + )
 & 
end of the paragraph
\\
\hline
META + n
 & 
beginning of the file
\\
\hline
META + /
 & 
end of the file
\\
\hline
META + {]}
 & 
move to the opening curly bracket \{ which corresponds to the closing curly bracket \}
\\
\hline
META + =
 & 
move screen one line down
\\
\hline
META + \_
 & 
move screen one line up
\\
\hline\end{tabulary}



\subsection{Search}
\label{editor/nano:search}
\begin{tabulary}{\linewidth}{|L|L|}
\hline

CTRL + n
 & 
search and replace one text
\\
\hline\end{tabulary}



\chapter{System}
\label{index:system}

\section{Archive and Compression}
\label{system/linux/archive:archive-and-compression}\label{system/linux/archive::doc}

\subsection{tar (Tape ARchiver)}
\label{system/linux/archive:tar-tape-archiver}
tar is very useful to backup / group your files. You could group your files / folders with

\begin{Verbatim}[commandchars=\\\{\}]
\PYG{n+nv}{\PYGZdl{} }tar cvf backup.tar file1 file2 ... folder1 folder2 ...
\end{Verbatim}

To extract content of a tar file, just do the following

\begin{Verbatim}[commandchars=\\\{\}]
\PYG{n+nv}{\PYGZdl{} }tar xvf backup.tar
\end{Verbatim}

N.B. tar is probably one of the first commands existing in the Unix / Linux world. You could invoke its options with a dash or not, this will not change the result.


\subsubsection{tar options}
\label{system/linux/archive:tar-options}
Here is a list of some useful options :

\begin{tabulary}{\linewidth}{|L|L|}
\hline

c
 & 
create, for creating tar file
\\
\hline
v
 & 
verbose, display name of files including,excluding from tar command
\\
\hline \multirow{2}{*}{
f
} & 
following, used to point name of tar file to be created.
\\
\cline{2-2} & 
it actually tells tar command that name of the file is ``next'' letter just after options.
\\
\hline
x
 & 
extract, for extracting files from tar file.
\\
\hline
t
 & 
for viewing content of tar file
\\
\hline
z
 & 
zip, tells tar command that create tar file using gzip.
\\
\hline
j
 & 
another compressing option tells tar command to use bzip2 for compression
\\
\hline
r
 & 
update or add file or directory in already existed .tar file
\\
\hline
wildcards
 & 
to specify patters in unix tar command
\\
\hline\end{tabulary}



\subsubsection{tarball}
\label{system/linux/archive:tarball}
When combining with gzip, tar becomes an extremely recommanded tool to archieve your documents. The resulted file is often called tarball, and is ended with either the extension .tgz or .tar.gz. For example, to archieve files / folders into a tarball:

\begin{Verbatim}[commandchars=\\\{\}]
\PYG{n+nv}{\PYGZdl{} }tar zcvf backup.tgz file1 file2 ... folder1 folder2 ...
\end{Verbatim}

and to extract a tarball:

\begin{Verbatim}[commandchars=\\\{\}]
\PYG{n+nv}{\PYGZdl{} }tar zxvf backup.tgz
\end{Verbatim}


\subsubsection{creation}
\label{system/linux/archive:creation}
Create tar archive/file

\begin{Verbatim}[commandchars=\\\{\}]
\PYG{n+nv}{\PYGZdl{} }tar \PYGZhy{}cvf tarball.tar *
\PYG{n+nv}{\PYGZdl{} }tar \PYGZhy{}cvf tarball.tar file1 folder1 file2
\end{Verbatim}

Create compressed tar file (gzip, bzip2)

\begin{Verbatim}[commandchars=\\\{\}]
\PYG{n+nv}{\PYGZdl{} }tar \PYGZhy{}zcvf tarball.tgz *
\PYG{n+nv}{\PYGZdl{} }tar \PYGZhy{}zcvf tarball.tar.gz *
\PYG{n+nv}{\PYGZdl{} }tar \PYGZhy{}jcvf tarball.tar.bz2 *
\end{Verbatim}


\subsubsection{extraction}
\label{system/linux/archive:extraction}
Extract contents from tar file

\begin{Verbatim}[commandchars=\\\{\}]
\PYG{n+nv}{\PYGZdl{} }tar \PYGZhy{}xvf tarball.tar
\end{Verbatim}

Extract contents from compressed tar file

\begin{Verbatim}[commandchars=\\\{\}]
\PYG{n+nv}{\PYGZdl{} }tar \PYGZhy{}zxvf tarball.tgz
\PYG{n+nv}{\PYGZdl{} }tar \PYGZhy{}zxvf tarball.tar.gz
\PYG{n+nv}{\PYGZdl{} }tar \PYGZhy{}jxvf tarball.tar.bz2
\end{Verbatim}

Extract a particular file from tar file

\begin{Verbatim}[commandchars=\\\{\}]
\PYG{n+nv}{\PYGZdl{} }tar \PYGZhy{}xvf tarball.tar file
\PYG{n+nv}{\PYGZdl{} }tar \PYGZhy{}zxvf tarball.tgz file
\PYG{n+nv}{\PYGZdl{} }tar \PYGZhy{}zxvf tarball.tar.gz file
\PYG{n+nv}{\PYGZdl{} }tar \PYGZhy{}jxvf tarball.tar.bz2 file
\end{Verbatim}

Extract a group of files from tar file

\begin{Verbatim}[commandchars=\\\{\}]
\PYG{n+nv}{\PYGZdl{} }tar \PYGZhy{}xvf tarball.tar \PYGZhy{}\PYGZhy{}wildcards \PYG{l+s+s2}{\PYGZdq{}s*\PYGZdq{}}
\PYG{n+nv}{\PYGZdl{} }tar \PYGZhy{}zxvf tarball.tgz \PYGZhy{}\PYGZhy{}wildcards \PYG{l+s+s2}{\PYGZdq{}s*\PYGZdq{}}
\PYG{n+nv}{\PYGZdl{} }tar \PYGZhy{}zxvf tarball.tar.gz \PYGZhy{}\PYGZhy{}wildcards \PYG{l+s+s2}{\PYGZdq{}s*\PYGZdq{}}
\PYG{n+nv}{\PYGZdl{} }tar \PYGZhy{}jxvf tarball.tar.bz2 \PYGZhy{}\PYGZhy{}wildcards \PYG{l+s+s2}{\PYGZdq{}s*\PYGZdq{}}
\end{Verbatim}


\subsubsection{other operations}
\label{system/linux/archive:other-operations}
View contents of tar file

\begin{Verbatim}[commandchars=\\\{\}]
\PYG{n+nv}{\PYGZdl{} }tar \PYGZhy{}tvf tarball.tar
\end{Verbatim}

View contents of compressed tar file

\begin{Verbatim}[commandchars=\\\{\}]
\PYG{n+nv}{\PYGZdl{} }tar \PYGZhy{}ztvf tarball.tgz
\PYG{n+nv}{\PYGZdl{} }tar \PYGZhy{}ztvf tarball.tar.gz
\end{Verbatim}

Update existing tar file (!not compressed tar file)

\begin{Verbatim}[commandchars=\\\{\}]
\PYG{n+nv}{\PYGZdl{} }tar \PYGZhy{}cvf tarball.tar file folder1
\PYG{n+nv}{\PYGZdl{} }tar \PYGZhy{}rvf tarball.tar folder2
\end{Verbatim}

Calculate the size (in KB) of (compressed) tar file

\begin{Verbatim}[commandchars=\\\{\}]
\PYG{n+nv}{\PYGZdl{} }tar \PYGZhy{}cf \PYGZhy{} * \PYG{p}{\textbar{}} wc \PYGZhy{}c
\PYG{n+nv}{\PYGZdl{} }tar \PYGZhy{}zcf \PYGZhy{} * \PYG{p}{\textbar{}} wc \PYGZhy{}c
\PYG{n+nv}{\PYGZdl{} }tar \PYGZhy{}jcf \PYGZhy{} * \PYG{p}{\textbar{}} wc \PYGZhy{}c
\end{Verbatim}

Delete items (files/folders) from tar file (!not compressed)

\begin{Verbatim}[commandchars=\\\{\}]
\PYG{n+nv}{\PYGZdl{} }gunzip tarball.tar.gz
\PYG{n+nv}{\PYGZdl{} }tar \PYGZhy{}\PYGZhy{}list \PYGZhy{}\PYGZhy{}file tarball.tar
\PYG{n+nv}{\PYGZdl{} }tar \PYGZhy{}\PYGZhy{}file tarball.tar \PYGZhy{}\PYGZhy{}delete file1 folder1 folder2/file2
\PYG{n+nv}{\PYGZdl{} }tar \PYGZhy{}\PYGZhy{}list \PYGZhy{}\PYGZhy{}file tarball.tar
\PYG{n+nv}{\PYGZdl{} }gzip tarball.tar
\end{Verbatim}


\subsection{parallel gzip}
\label{system/linux/archive:parallel-gzip}
For modern multi-processor, multi-core machines, a parallel implementation of gzip exists, called pigz. The official link is \href{http://zlib.net/pigz/}{http://zlib.net/pigz/} and the manual could be downloaded here : \href{http://zlib.net/pigz/pigz.pdf}{http://zlib.net/pigz/pigz.pdf}

To compress while keeping the original file, using

\begin{Verbatim}[commandchars=\\\{\}]
pigz \PYGZhy{}\PYGZhy{}best \PYGZhy{}k file
\end{Verbatim}

and to decompress the .gz file

\begin{Verbatim}[commandchars=\\\{\}]
pigz \PYGZhy{}d file.gz
\end{Verbatim}


\subsection{bz2}
\label{system/linux/archive:bz2}

\subsubsection{Compression to bz2}
\label{system/linux/archive:compression-to-bz2}
\begin{Verbatim}[commandchars=\\\{\}]
\PYG{n+nv}{\PYGZdl{} }bzip2 file
\PYG{n+nv}{\PYGZdl{} }bzip2 \PYGZhy{}v file
\end{Verbatim}


\subsubsection{Decompression from bz2}
\label{system/linux/archive:decompression-from-bz2}
\begin{Verbatim}[commandchars=\\\{\}]
\PYG{n+nv}{\PYGZdl{} }bunzip2 file.bz2
\PYG{n+nv}{\PYGZdl{} }bzip2 \PYGZhy{}d file.bz2
\PYG{n+nv}{\PYGZdl{} }bunzip2 \PYGZhy{}v file.bz2
\end{Verbatim}


\subsubsection{Archive to tarball .tar.bz2}
\label{system/linux/archive:archive-to-tarball-tar-bz2}
\begin{Verbatim}[commandchars=\\\{\}]
\PYG{n+nv}{\PYGZdl{} }tar cjvf files.tar.bz2 *.txt
\PYG{n+nv}{\PYGZdl{} }tar \PYGZhy{}cjvf archive.tar.bz2 path/to/folder
\PYG{n+nv}{\PYGZdl{} }tar \PYGZhy{}\PYGZhy{}bzip2 \PYGZhy{}xf path/to/file.tar.bz2
\PYG{n+nv}{\PYGZdl{} }tar \PYGZhy{}cvf \PYGZhy{} path/to/folder \PYG{p}{\textbar{}} bzip2 \PYGZgt{} archive.tar.bz2
\end{Verbatim}


\subsubsection{Restore from tarball .tar.bz2}
\label{system/linux/archive:restore-from-tarball-tar-bz2}
\begin{Verbatim}[commandchars=\\\{\}]
\PYG{n+nv}{\PYGZdl{} }tar xjvf files.tar.bz2
\PYG{n+nv}{\PYGZdl{} }tar \PYGZhy{}xjvf path/to/file.tar.bz2
\PYG{n+nv}{\PYGZdl{} }bzcat path/to/file.tar.bz2 \PYG{p}{\textbar{}} tar \PYGZhy{}xvf \PYGZhy{}
\end{Verbatim}


\subsection{zip}
\label{system/linux/archive:zip}
\begin{Verbatim}[commandchars=\\\{\}]
\PYG{n+nv}{\PYGZdl{} }zip archive.zip file1 file2 file3
\PYG{n+nv}{\PYGZdl{} }unzip archive.zip
\end{Verbatim}


\subsection{lzma}
\label{system/linux/archive:lzma}

\subsubsection{Restore from lzma tarball}
\label{system/linux/archive:restore-from-lzma-tarball}
\begin{Verbatim}[commandchars=\\\{\}]
\PYG{n+nv}{\PYGZdl{} }tar \PYGZhy{}xYvf archive.tar.lzma
\end{Verbatim}


\subsection{lzo}
\label{system/linux/archive:lzo}
The difference between the format .lzo and others (.gz or .bz2) are :
\begin{itemize}
\item {} 
.lzo is not installed by default on your machine;

\item {} 
.lzo keeps the original file, unless if you use the option -U;

\item {} 
.lzo runs fast, but the compression ratio is relatively low.

\end{itemize}

When we try to pass a list of files and folders to lzop, only the files will be compressed and the folders will be skipped.


\subsubsection{Compression to lzo}
\label{system/linux/archive:compression-to-lzo}
\begin{Verbatim}[commandchars=\\\{\}]
\PYG{n+nv}{\PYGZdl{} }lzop \PYGZhy{}v file
\PYG{n+nv}{\PYGZdl{} }cat file \PYG{p}{\textbar{}} lzop \PYGZgt{} file.lzo
\end{Verbatim}

Delete the original file

\begin{Verbatim}[commandchars=\\\{\}]
\PYG{n+nv}{\PYGZdl{} }lzop \PYGZhy{}U file
\end{Verbatim}

Test the result's integrality

\begin{Verbatim}[commandchars=\\\{\}]
\PYG{n+nv}{\PYGZdl{} }lzop \PYGZhy{}t file.lzo
\end{Verbatim}

Show file headers

\begin{Verbatim}[commandchars=\\\{\}]
\PYG{n+nv}{\PYGZdl{} }lzop \PYGZhy{}\PYGZhy{}info file.lzo       afficher les en\PYGZhy{}têtes du ficher
\end{Verbatim}

Show compression information

\begin{Verbatim}[commandchars=\\\{\}]
\PYG{n+nv}{\PYGZdl{} }lzop \PYGZhy{}l file.lzo
\end{Verbatim}

Show content of a compressed lzo file

\begin{Verbatim}[commandchars=\\\{\}]
\PYG{n+nv}{\PYGZdl{} }lzop \PYGZhy{}\PYGZhy{}ls file.lzo
\end{Verbatim}


\subsubsection{Decompression from lzo}
\label{system/linux/archive:decompression-from-lzo}
The lzo file is kept by default.

\begin{Verbatim}[commandchars=\\\{\}]
\PYG{n+nv}{\PYGZdl{} }lzop \PYGZhy{}dv file.lzo
\end{Verbatim}


\subsubsection{Archive all text files to a lzo tarball}
\label{system/linux/archive:archive-all-text-files-to-a-lzo-tarball}
\begin{Verbatim}[commandchars=\\\{\}]
\PYG{n+nv}{\PYGZdl{} }tar \PYGZhy{}\PYGZhy{}use\PYGZhy{}compress\PYGZhy{}program\PYG{o}{=}lzop \PYGZhy{}cf files.tar.lzo *.txt
\end{Verbatim}


\subsubsection{Restore from lzo tarball}
\label{system/linux/archive:restore-from-lzo-tarball}
\begin{Verbatim}[commandchars=\\\{\}]
\PYG{n+nv}{\PYGZdl{} }tar \PYGZhy{}\PYGZhy{}use\PYGZhy{}compress\PYGZhy{}program\PYG{o}{=}lzop \PYGZhy{}xf files.tar.lzo
\end{Verbatim}


\section{Daily Tools}
\label{system/linux/tools::doc}\label{system/linux/tools:daily-tools}

\subsection{find}
\label{system/linux/tools:find}
\begin{Verbatim}[commandchars=\\\{\}]
find \PYGZhy{}print is the same as find, \PYGZhy{}print option is a default option
find \PYGZhy{}print0 ... \textbar{} xargs \PYGZhy{}0 ... can avoid whitespace problem
find \PYGZhy{}delete can be used for \PYGZhy{}exec rm \PYGZob{}\PYGZcb{} \PYGZbs{};
\end{Verbatim}

Find out shell scripts without execution right

\begin{Verbatim}[commandchars=\\\{\}]
\PYG{n+nv}{\PYGZdl{} }find . \PYGZhy{}iname \PYG{l+s+s2}{\PYGZdq{}*.sh\PYGZdq{}} \PYGZhy{}perm 644
\end{Verbatim}

Add execution right to shell scripts

\begin{Verbatim}[commandchars=\\\{\}]
\PYG{n+nv}{\PYGZdl{} }\PYG{k}{for} i in \PYG{l+s+sb}{{}`}find . \PYGZhy{}iname \PYG{l+s+s2}{\PYGZdq{}*.sh\PYGZdq{}} \PYGZhy{}perm 644\PYG{l+s+sb}{{}`}\PYG{p}{;} \PYG{k}{do} \PYG{n+nb}{echo} \PYG{n+nv}{\PYGZdl{}i}\PYG{p}{;} chmod a+x \PYG{n+nv}{\PYGZdl{}i}\PYG{p}{;} \PYG{k}{done}\PYG{p}{;}
\end{Verbatim}

Find out all files modified since less than 1 day, excatly 1 day or more than 1 day

\begin{Verbatim}[commandchars=\\\{\}]
\PYG{n+nv}{\PYGZdl{} }find . \PYGZhy{}mtime \PYGZhy{}1
\PYG{n+nv}{\PYGZdl{} }find . \PYGZhy{}mtime 1
\PYG{n+nv}{\PYGZdl{} }find . \PYGZhy{}mtime +1
\end{Verbatim}

Delete found files

\begin{Verbatim}[commandchars=\\\{\}]
\PYG{n+nv}{\PYGZdl{} }find . \PYGZhy{}name \PYG{l+s+s2}{\PYGZdq{}*.tmp\PYGZdq{}} \PYGZhy{}delete
\PYG{n+nv}{\PYGZdl{} }find . \PYGZhy{}name \PYG{l+s+s2}{\PYGZdq{}*.tmp\PYGZdq{}} \PYGZhy{}print \PYG{p}{\textbar{}} xargs rm \PYGZhy{}f
\end{Verbatim}

Grep found files using -print0 and xargs -0 to avoid whitespace problem

\begin{Verbatim}[commandchars=\\\{\}]
\PYG{n+nv}{\PYGZdl{} }find . \PYGZhy{}name \PYG{l+s+s2}{\PYGZdq{}*.txt\PYGZdq{}} \PYGZhy{}print \PYG{p}{\textbar{}} xargs grep \PYG{l+s+s2}{\PYGZdq{}Exception\PYGZdq{}}
\PYG{n+nv}{\PYGZdl{} }find . \PYGZhy{}name \PYG{l+s+s2}{\PYGZdq{}*.txt\PYGZdq{}} \PYGZhy{}print0 \PYG{p}{\textbar{}} xargs \PYGZhy{}0 grep \PYG{l+s+s2}{\PYGZdq{}Exception\PYGZdq{}}
\end{Verbatim}

Find in the current folder, type file (not link, directory) and newer than first\_file

\begin{Verbatim}[commandchars=\\\{\}]
\PYG{n+nv}{\PYGZdl{} }find . \PYGZhy{}maxdepth \PYG{l+m}{1} \PYGZhy{}type f \PYGZhy{}newer first\PYGZus{}file
\end{Verbatim}

Find in the current folder, type file and modified since more than 15 mins

\begin{Verbatim}[commandchars=\\\{\}]
\PYG{n+nv}{\PYGZdl{} }find . \PYGZhy{}type f \PYGZhy{}cmin \PYG{l+m}{15} \PYGZhy{}prune
\end{Verbatim}

Find out and list files more than 1000 bytes

\begin{Verbatim}[commandchars=\\\{\}]
\PYG{n+nv}{\PYGZdl{} }find . \PYGZhy{}size +1000c \PYGZhy{}exec ls \PYGZhy{}l \PYG{o}{\PYGZob{}}\PYG{o}{\PYGZcb{}} \PYG{l+s+se}{\PYGZbs{};}
\end{Verbatim}

Find out files with size more than 10000 bytes and less than 50000 bytes

\begin{Verbatim}[commandchars=\\\{\}]
\PYG{n+nv}{\PYGZdl{} }find . \PYGZhy{}size +10000c \PYGZhy{}size \PYGZhy{}50000c \PYGZhy{}print
\end{Verbatim}

Find out and list files modified 10 days ago and more than 50000 bytes

\begin{Verbatim}[commandchars=\\\{\}]
\PYG{n+nv}{\PYGZdl{} }find . \PYGZhy{}mtime +10 \PYGZhy{}size +50000c \PYGZhy{}exec ls \PYGZhy{}l \PYG{o}{\PYGZob{}}\PYG{o}{\PYGZcb{}} \PYG{l+s+se}{\PYGZbs{};}
\end{Verbatim}

Find out and list all symbolic links

\begin{Verbatim}[commandchars=\\\{\}]
\PYG{n+nv}{\PYGZdl{} }find . \PYGZhy{}type l \PYGZhy{}print \PYG{p}{\textbar{}} xargs ls \PYGZhy{}ld \PYG{p}{\textbar{}} awk \PYG{l+s+s1}{\PYGZsq{}\PYGZob{}print \PYGZdl{}9 \PYGZdq{} \PYGZdq{} \PYGZdl{}10 \PYGZdq{} \PYGZdq{} \PYGZdl{}11\PYGZcb{}\PYGZsq{}}
\end{Verbatim}

Find all the files without permission 777

\begin{Verbatim}[commandchars=\\\{\}]
\PYG{n+nv}{\PYGZdl{} }find / \PYGZhy{}type f ! \PYGZhy{}perm 777
\end{Verbatim}

Find all the SGID bit files whose permissions set to 644

\begin{Verbatim}[commandchars=\\\{\}]
\PYG{n+nv}{\PYGZdl{} }find / \PYGZhy{}perm 2644
\end{Verbatim}

Find all the Sticky Bit set files whose permission are 551

\begin{Verbatim}[commandchars=\\\{\}]
\PYG{n+nv}{\PYGZdl{} }find / \PYGZhy{}perm 1551
\end{Verbatim}

Find all SUID set files.

\begin{Verbatim}[commandchars=\\\{\}]
\PYG{n+nv}{\PYGZdl{} }find / \PYGZhy{}perm /u\PYG{o}{=}s
\end{Verbatim}

Find all SGID set files.

\begin{Verbatim}[commandchars=\\\{\}]
\PYG{n+nv}{\PYGZdl{} }find / \PYGZhy{}perm /g+s
\end{Verbatim}

Find all empty files under certain path

\begin{Verbatim}[commandchars=\\\{\}]
\PYG{n+nv}{\PYGZdl{} }find /tmp \PYGZhy{}type f \PYGZhy{}empty
\end{Verbatim}

Find all all empty directories under certain path

\begin{Verbatim}[commandchars=\\\{\}]
\PYG{n+nv}{\PYGZdl{} }find /tmp \PYGZhy{}type d \PYGZhy{}empty
\end{Verbatim}


\subsection{grep}
\label{system/linux/tools:grep}
There are several variants of grep:
\begin{itemize}
\item {} 
grep

\item {} 
egrep : extended grep

\item {} 
fgrep : fixed grep

\item {} 
zgrep

\end{itemize}

Grep A, but excluding B

\begin{Verbatim}[commandchars=\\\{\}]
\PYG{n+nv}{\PYGZdl{} }grep A file \PYG{p}{\textbar{}} grep \PYGZhy{}v B
\end{Verbatim}

Count the occurence of a word A

\begin{Verbatim}[commandchars=\\\{\}]
\PYG{n+nv}{\PYGZdl{} }grep \PYGZhy{}c A file
\end{Verbatim}

Grep A and show n lines prior to the A position and n lines after the A position

\begin{Verbatim}[commandchars=\\\{\}]
\PYG{n+nv}{\PYGZdl{} }grep \PYGZhy{}\PYGZhy{}context\PYG{o}{=}n A file
\PYG{n+nv}{\PYGZdl{} }grep \PYGZhy{}C n A file
\end{Verbatim}

Extended grep with more regular expression

\begin{Verbatim}[commandchars=\\\{\}]
\PYG{n+nv}{\PYGZdl{} }egrep \PYG{l+s+s2}{\PYGZdq{}A\textbar{}B\PYGZdq{}} file
\end{Verbatim}

Case insensitive grep

\begin{Verbatim}[commandchars=\\\{\}]
\PYG{n+nv}{\PYGZdl{} }grep \PYGZhy{}i A file
\end{Verbatim}

Grep .gz file

\begin{Verbatim}[commandchars=\\\{\}]
\PYG{n+nv}{\PYGZdl{} }zgrep A file.gz
\end{Verbatim}

Grep whole word

\begin{Verbatim}[commandchars=\\\{\}]
\PYG{n+nv}{\PYGZdl{} }grep \PYGZhy{}w WORD file
\end{Verbatim}

Grep words starting with WORD

\begin{Verbatim}[commandchars=\\\{\}]
\PYG{n+nv}{\PYGZdl{} }grep \PYG{l+s+s1}{\PYGZsq{}\PYGZlt{}WORD\PYGZsq{}} file
\end{Verbatim}

Grep words ending in WORD

\begin{Verbatim}[commandchars=\\\{\}]
\PYG{n+nv}{\PYGZdl{} }grep \PYG{l+s+s1}{\PYGZsq{}WORD\PYGZgt{}\PYGZsq{}} file
\end{Verbatim}

Grep with matching pattern in color

\begin{Verbatim}[commandchars=\\\{\}]
\PYG{n+nv}{\PYGZdl{} }grep A file \PYGZhy{}\PYGZhy{}color
\end{Verbatim}

Grep among files matching the given pattern

\begin{Verbatim}[commandchars=\\\{\}]
\PYG{n+nv}{\PYGZdl{} }grep \PYGZhy{}l A *.log
\end{Verbatim}

Grep showing line number

\begin{Verbatim}[commandchars=\\\{\}]
\PYG{n+nv}{\PYGZdl{} }grep \PYGZhy{}n A file
\end{Verbatim}

Recursive grep

\begin{Verbatim}[commandchars=\\\{\}]
\PYG{n+nv}{\PYGZdl{} }grep \PYGZhy{}R A folder
\end{Verbatim}


\subsection{sort}
\label{system/linux/tools:sort}
Sort based on the numerical order of the 2nd column

\begin{Verbatim}[commandchars=\\\{\}]
\PYG{n+nv}{\PYGZdl{} }ps \PYGZhy{}ef \PYG{p}{\textbar{}} sort \PYGZhy{}nk2
\end{Verbatim}

Sort based on the reverse numerical order of the 3rd column

\begin{Verbatim}[commandchars=\\\{\}]
\PYG{n+nv}{\PYGZdl{} }ps \PYGZhy{}ef \PYG{p}{\textbar{}} sort \PYGZhy{}rnk3
\end{Verbatim}

Sort based on the alphabetic order of the 4th column

\begin{Verbatim}[commandchars=\\\{\}]
\PYG{n+nv}{\PYGZdl{} }ps \PYGZhy{}ef \PYG{p}{\textbar{}} sort \PYGZhy{}nk4
\end{Verbatim}

Sort based on the alphabetic order

\begin{Verbatim}[commandchars=\\\{\}]
\PYG{n+nv}{\PYGZdl{} }cat file \PYG{p}{\textbar{}} sort
\PYG{n+nv}{\PYGZdl{} }sort file
\end{Verbatim}

Sort based on the alphabetic order, and remove duplicates

\begin{Verbatim}[commandchars=\\\{\}]
\PYG{n+nv}{\PYGZdl{} }cat file \PYG{p}{\textbar{}} sort \PYG{p}{\textbar{}} uniq
\PYG{n+nv}{\PYGZdl{} }cat file \PYG{p}{\textbar{}} sort \PYGZhy{}u
\PYG{n+nv}{\PYGZdl{} }sort \PYGZhy{}u file
\end{Verbatim}

Sort file, case insensitive

\begin{Verbatim}[commandchars=\\\{\}]
\PYG{n+nv}{\PYGZdl{} }sort \PYGZhy{}f file
\end{Verbatim}


\subsection{sed}
\label{system/linux/tools:sed}
Replace word A by B in a file

\begin{Verbatim}[commandchars=\\\{\}]
\PYG{n+nv}{\PYGZdl{} }sed s/A/B/g file
\end{Verbatim}


\subsection{cat}
\label{system/linux/tools:cat}
Create a new file and fill it

\begin{Verbatim}[commandchars=\\\{\}]
\PYG{n+nv}{\PYGZdl{} }cat \PYGZgt{} file.txt
fill the file
...
CTRL+D
\end{Verbatim}

Append new input to a file (so, keep its current content)

\begin{Verbatim}[commandchars=\\\{\}]
\PYG{n+nv}{\PYGZdl{} }cat \PYGZgt{}\PYGZgt{} file.txt
append the file
...
CTRL+D
\end{Verbatim}

Displaying with line numbers (option -b will number only non-empty lines)

\begin{Verbatim}[commandchars=\\\{\}]
\PYG{n+nv}{\PYGZdl{} }cat \PYGZhy{}n file.txt
\PYG{n+nv}{\PYGZdl{} }cat \PYGZhy{}b file.txt
\end{Verbatim}

Copy files

\begin{Verbatim}[commandchars=\\\{\}]
\PYG{n+nv}{\PYGZdl{} }cat file1 \PYGZgt{} file2
\end{Verbatim}

Concatenating files

\begin{Verbatim}[commandchars=\\\{\}]
\PYG{n+nv}{\PYGZdl{} }cat file \PYG{l+m}{1} file2 \PYGZgt{} file
\end{Verbatim}

Squeezing repeating blank lines into one

\begin{Verbatim}[commandchars=\\\{\}]
\PYG{n+nv}{\PYGZdl{} }cat \PYGZhy{}s file
\end{Verbatim}

Show non-printing chars (\textasciicircum{}V, \textasciicircum{}M, etc.)

\begin{Verbatim}[commandchars=\\\{\}]
\PYG{n+nv}{\PYGZdl{} }cat \PYGZhy{}v file
\end{Verbatim}


\subsection{top / htop}
\label{system/linux/tools:top-htop}
By default, the display of top is ordered by CPU usage

\begin{Verbatim}[commandchars=\\\{\}]
\PYG{n+nv}{\PYGZdl{} }top
\end{Verbatim}

In the top interface, you have the following choices

\begin{tabulary}{\linewidth}{|L|L|}
\hline

M
 & 
ordered by Memory usage
\\
\hline
1
 & 
display all CPUs
\\
\hline
c
 & 
display path of commands
\\
\hline
k
 & 
kill a process via its PID
\\
\hline
z
 & 
highlight running processes
\\
\hline
s
 & 
change refresh delay
\\
\hline
h or ?
 & 
online help
\\
\hline
q
 & 
quit
\\
\hline
W
 & 
save changes (into \textasciitilde{}/.toprc)
\\
\hline\end{tabulary}


To work on one particular user

\begin{Verbatim}[commandchars=\\\{\}]
\PYG{n+nv}{\PYGZdl{} }top \PYGZhy{}u user
\end{Verbatim}


\subsection{lsof}
\label{system/linux/tools:lsof}
list all open files

\begin{Verbatim}[commandchars=\\\{\}]
\PYG{n+nv}{\PYGZdl{} }lsof
\end{Verbatim}

\code{FD} stands for \textbf{File descriptor} and may seen some of the values as:

\begin{tabulary}{\linewidth}{|L|L|}
\hline

cwd
 & 
current working directory
\\
\hline
rtd
 & 
root directory
\\
\hline
txt
 & 
program text (code and data)
\\
\hline
mem
 & 
memory-mapped file
\\
\hline\end{tabulary}


Also in \code{FD} column numbers like \code{1u} is actual file descriptor and followed by u,r,w of it’s mode as:

\begin{tabulary}{\linewidth}{|L|L|}
\hline

r
 & 
for read access
\\
\hline
w
 & 
for write access
\\
\hline
u
 & 
for read and write access
\\
\hline\end{tabulary}


\code{TYPE} of files and it’s identification:

\begin{tabulary}{\linewidth}{|L|L|}
\hline

DIR
 & 
Directory
\\
\hline
REG
 & 
Regular file
\\
\hline
CHR
 & 
Character special file
\\
\hline
FIFO
 & 
First In First Out
\\
\hline\end{tabulary}


list open files of one user

\begin{Verbatim}[commandchars=\\\{\}]
\PYG{n+nv}{\PYGZdl{} }lsof \PYGZhy{}u user
\end{Verbatim}

find out all the running process of specific port

\begin{Verbatim}[commandchars=\\\{\}]
\PYG{n+nv}{\PYGZdl{} }lsof \PYGZhy{}i TCP:22
\end{Verbatim}

shows only IPv4 and IPv6 network files

\begin{Verbatim}[commandchars=\\\{\}]
\PYG{n+nv}{\PYGZdl{} }lsof \PYGZhy{}i 4
\PYG{n+nv}{\PYGZdl{} }lsof \PYGZhy{}i 6
\end{Verbatim}

list all running process of open files of TCP Port ranges from 1-1024.

\begin{Verbatim}[commandchars=\\\{\}]
\PYG{n+nv}{\PYGZdl{} }lsof \PYGZhy{}i TCP:1\PYGZhy{}1024
\end{Verbatim}

exclude user with ‘\textasciicircum{}’ Character : e.g. to exclude root user

\begin{Verbatim}[commandchars=\\\{\}]
\PYG{n+nv}{\PYGZdl{} }lsof \PYGZhy{}i \PYGZhy{}u\PYGZca{}root
\end{Verbatim}

find out a specific user is looking what files and commands

\begin{Verbatim}[commandchars=\\\{\}]
\PYG{n+nv}{\PYGZdl{} }lsof \PYGZhy{}i \PYGZhy{}u user
\end{Verbatim}

list all network connections : option `-i' shows the list of all network connections `LISTENING \& ESTABLISHED'.

\begin{Verbatim}[commandchars=\\\{\}]
\PYG{n+nv}{\PYGZdl{} }lsof \PYGZhy{}i
\end{Verbatim}

search by PID

\begin{Verbatim}[commandchars=\\\{\}]
\PYG{n+nv}{\PYGZdl{} }lsof \PYGZhy{}p 1
\end{Verbatim}

kill all activity of particular user

\begin{Verbatim}[commandchars=\\\{\}]
\PYG{n+nv}{\PYGZdl{} }\PYG{n+nb}{kill} \PYGZhy{}9 \PYG{l+s+sb}{{}`}lsof \PYGZhy{}t \PYGZhy{}u user\PYG{l+s+sb}{{}`}
\end{Verbatim}


\subsection{netstat}
\label{system/linux/tools:netstat}
listing all the LISTENING Ports of TCP and UDP connections

\begin{Verbatim}[commandchars=\\\{\}]
\PYG{n+nv}{\PYGZdl{} }netstat \PYGZhy{}a \PYG{p}{\textbar{}} more
\end{Verbatim}

listing only TCP (Transmission Control Protocol) port connections

\begin{Verbatim}[commandchars=\\\{\}]
\PYG{n+nv}{\PYGZdl{} }netstat \PYGZhy{}at
\end{Verbatim}

listing only UDP (User Datagram Protocol ) port connections

\begin{Verbatim}[commandchars=\\\{\}]
\PYG{n+nv}{\PYGZdl{} }netstat \PYGZhy{}au
\end{Verbatim}

listing all active listening ports connections

\begin{Verbatim}[commandchars=\\\{\}]
\PYG{n+nv}{\PYGZdl{} }netstat \PYGZhy{}l
\end{Verbatim}

listing all active listening TCP ports

\begin{Verbatim}[commandchars=\\\{\}]
\PYG{n+nv}{\PYGZdl{} }netstat \PYGZhy{}lt
\end{Verbatim}

listing all active listening UDP ports

\begin{Verbatim}[commandchars=\\\{\}]
\PYG{n+nv}{\PYGZdl{} }netstat \PYGZhy{}lu
\end{Verbatim}

Listing all active UNIX listening ports

\begin{Verbatim}[commandchars=\\\{\}]
\PYG{n+nv}{\PYGZdl{} }netstat \PYGZhy{}lx
\end{Verbatim}

Showing Statistics by Protocol

\begin{Verbatim}[commandchars=\\\{\}]
\PYG{n+nv}{\PYGZdl{} }netstat \PYGZhy{}s
\end{Verbatim}

Showing statistics of TCP protocol

\begin{Verbatim}[commandchars=\\\{\}]
\PYG{n+nv}{\PYGZdl{} }netstat \PYGZhy{}st
\end{Verbatim}

Showing statistics of UDP Protocol

\begin{Verbatim}[commandchars=\\\{\}]
\PYG{n+nv}{\PYGZdl{} }netstat \PYGZhy{}su
\end{Verbatim}

Displaying service name with their PID number : option -tp will display “PID/Program Name”

\begin{Verbatim}[commandchars=\\\{\}]
\PYG{n+nv}{\PYGZdl{} }netstat \PYGZhy{}tp
\end{Verbatim}

Displaying Promiscuous mode with -ac switch, netstat print the selected information or refresh screen every five second. Default screen refresh in every second.

\begin{Verbatim}[commandchars=\\\{\}]
\PYG{n+nv}{\PYGZdl{} }netstat \PYGZhy{}ac \PYG{l+m}{5} \PYG{p}{\textbar{}} grep tcp
\end{Verbatim}

Displaying Kernel IP routing table

\begin{Verbatim}[commandchars=\\\{\}]
\PYG{n+nv}{\PYGZdl{} }netstat \PYGZhy{}r
\end{Verbatim}

Showing network interface packet transactions including both transferring and receiving packets with MTU size

\begin{Verbatim}[commandchars=\\\{\}]
\PYG{n+nv}{\PYGZdl{} }netstat \PYGZhy{}i
\end{Verbatim}

Showing Kernel interface table, similar to ifconfig command.

\begin{Verbatim}[commandchars=\\\{\}]
\PYG{n+nv}{\PYGZdl{} }netstat \PYGZhy{}ie
\end{Verbatim}

Displaying multicast group membership information for both IPv4 and IPv6

\begin{Verbatim}[commandchars=\\\{\}]
\PYG{n+nv}{\PYGZdl{} }netstat \PYGZhy{}g
\end{Verbatim}

Print netstat information every few second

\begin{Verbatim}[commandchars=\\\{\}]
\PYG{n+nv}{\PYGZdl{} }netstat \PYGZhy{}c
\end{Verbatim}

Finding un-configured address families with some useful information

\begin{Verbatim}[commandchars=\\\{\}]
\PYG{n+nv}{\PYGZdl{} }netstat \PYGZhy{}\PYGZhy{}verbose
\end{Verbatim}

Find out how many listening programs running on a port

\begin{Verbatim}[commandchars=\\\{\}]
\PYG{n+nv}{\PYGZdl{} }netstat \PYGZhy{}ap \PYG{p}{\textbar{}} grep http
\end{Verbatim}

Displaying RAW Network Statistics

\begin{Verbatim}[commandchars=\\\{\}]
\PYG{n+nv}{\PYGZdl{} }netstat \PYGZhy{}\PYGZhy{}statistics \PYGZhy{}\PYGZhy{}raw
\end{Verbatim}


\subsection{screen}
\label{system/linux/tools:screen}

\subsubsection{To use screen}
\label{system/linux/tools:to-use-screen}
\begin{Verbatim}[commandchars=\\\{\}]
\PYG{n+nv}{\PYGZdl{} }screen
\end{Verbatim}


\subsubsection{When connection lost or crash, to restore previous sessions (very useful !)}
\label{system/linux/tools:when-connection-lost-or-crash-to-restore-previous-sessions-very-useful}
\begin{Verbatim}[commandchars=\\\{\}]
\PYG{n+nv}{\PYGZdl{} }screen \PYGZhy{}r
\end{Verbatim}


\subsubsection{Remove dead screen}
\label{system/linux/tools:remove-dead-screen}
\begin{Verbatim}[commandchars=\\\{\}]
\PYG{n+nv}{\PYGZdl{} }screen \PYGZhy{}wipe
\end{Verbatim}


\subsubsection{Short-cuts}
\label{system/linux/tools:short-cuts}
\begin{tabulary}{\linewidth}{|L|L|L|}
\hline

CTRL + a then CTRL + c
 & 
CTRL + a, c
 & 
Create new window
\\
\hline \multicolumn{2}{|l|}{
CTRL + a then 0
} & 
Select Window 0
\\
\hline \multicolumn{2}{|l|}{
CTRL + a then 1
} & 
Select Window 1
\\
\hline \multicolumn{2}{|l|}{
CTRL + a then 2
} & 
Select Window 2
\\
\hline \multicolumn{2}{|l|}{
CTRL + a then 3
} & 
Select Window 3
\\
\hline \multicolumn{2}{|l|}{
CTRL + a then 4
} & 
Select Window 4
\\
\hline \multicolumn{2}{|l|}{
CTRL + a then 5
} & 
Select Window 5
\\
\hline \multicolumn{2}{|l|}{
CTRL + a then 6
} & 
Select Window 6
\\
\hline \multicolumn{2}{|l|}{
CTRL + a then 7
} & 
Select Window 7
\\
\hline \multicolumn{2}{|l|}{
CTRL + a then 8
} & 
Select Window 8
\\
\hline \multicolumn{2}{|l|}{
CTRL + a then 9
} & 
Select Window 9
\\
\hline \multicolumn{2}{|l|}{
CTRL + a then A
} & 
Set the name of the current window
\\
\hline \multicolumn{2}{|l|}{
CTRL + D
} & 
Close the current Window
\\
\hline\end{tabulary}



\subsubsection{config file .screenrc}
\label{system/linux/tools:config-file-screenrc}
\begin{Verbatim}[commandchars=\\\{\}]
\PYGZsh{} term vt100
\PYGZsh{} termcap sun\PYGZhy{}color bc=
\PYGZsh{} terminfo sun\PYGZhy{}color bc=

defscrollback 3000
startup\PYGZus{}message off
bind \PYGZsq{}\PYGZca{}\PYGZbs{}\PYGZsq{}

termcapinfo xterm ti@:te@

termcap  vt100 \PYGZsq{}AF=\PYGZbs{}E[3\PYGZpc{}dm:AB=\PYGZbs{}E[4\PYGZpc{}dm\PYGZsq{}
terminfo vt100 \PYGZsq{}AF=\PYGZbs{}E[3\PYGZpc{}p1\PYGZpc{}dm:AB=\PYGZbs{}E[4\PYGZpc{}p1\PYGZpc{}dm\PYGZsq{}
termcap  xterm \PYGZsq{}AF=\PYGZbs{}E[3\PYGZpc{}dm:AB=\PYGZbs{}E[4\PYGZpc{}dm\PYGZsq{}
terminfo xterm \PYGZsq{}AF=\PYGZbs{}E[3\PYGZpc{}p1\PYGZpc{}dm:AB=\PYGZbs{}E[4\PYGZpc{}p1\PYGZpc{}dm\PYGZsq{}

\PYGZsh{} test
hardstatus alwayslastline \PYGZdq{}\PYGZpc{}\PYGZob{}bw\PYGZcb{}x: \PYGZpc{}\PYGZhy{}Lw\PYGZpc{}\PYGZob{}= bw\PYGZcb{}\PYGZpc{}50\PYGZgt{}\PYGZpc{}n\PYGZpc{}f* \PYGZpc{}t\PYGZpc{}\PYGZob{}\PYGZhy{}\PYGZcb{}\PYGZpc{}+Lw\PYGZpc{}\PYGZlt{} \PYGZpc{}\PYGZgt{}\PYGZdq{}

\PYGZsh{} Bind F11 and F12 (NOT F1 and F2) to previous and next screen window
bindkey \PYGZhy{}k F1 prev
bindkey \PYGZhy{}k F2 next

\PYGZsh{} Produce a bell sound even when a background window is belling
bell\PYGZus{}msg \PYGZsq{}Bell in \PYGZpc{}\PYGZca{}G\PYGZsq{}
\end{Verbatim}


\subsection{MD5}
\label{system/linux/tools:md5}
\code{md5sum} stands for \textbf{Compute and Check MD5 Message Digest}

md5 checksum (commonly called hash) is used to match or verify integrity of files that may have changed as a result of a faulty file transfer, a disk error or non-malicious interference.

\begin{Verbatim}[commandchars=\\\{\}]
\PYG{n+nv}{\PYGZdl{} }md5sum file
\end{Verbatim}


\subsection{dd}
\label{system/linux/tools:dd}
\code{dd} stands for \textbf{Convert and Copy a file}

Can be used to convert and copy a file and most of the times is used to copy a iso file (or any other file) to a usb device (or any other location), thus can be used to make a ‘Bootlable‘ Usb Stick.

\begin{Verbatim}[commandchars=\\\{\}]
\PYG{n+nv}{\PYGZdl{} }dd \PYG{k}{if}\PYG{o}{=}/home/user/Downloads/debian.iso \PYG{n+nv}{of}\PYG{o}{=}/dev/sdb1 \PYG{n+nv}{bs}\PYG{o}{=}512M\PYG{p}{;} sync
\end{Verbatim}


\subsection{Calendar}
\label{system/linux/tools:calendar}
\begin{Verbatim}[commandchars=\\\{\}]
\PYG{n+nv}{\PYGZdl{} }cal \PYG{l+m}{02} 1835
   February 1835
Su Mo Tu We Th Fr Sa
 \PYG{l+m}{1}  \PYG{l+m}{2}  \PYG{l+m}{3}  \PYG{l+m}{4}  \PYG{l+m}{5}  \PYG{l+m}{6}  7
 \PYG{l+m}{8}  \PYG{l+m}{9} \PYG{l+m}{10} \PYG{l+m}{11} \PYG{l+m}{12} \PYG{l+m}{13} 14
\PYG{l+m}{15} \PYG{l+m}{16} \PYG{l+m}{17} \PYG{l+m}{18} \PYG{l+m}{19} \PYG{l+m}{20} 21
\PYG{l+m}{22} \PYG{l+m}{23} \PYG{l+m}{24} \PYG{l+m}{25} \PYG{l+m}{26} \PYG{l+m}{27} 28
\end{Verbatim}


\subsection{Date}
\label{system/linux/tools:date}
\begin{Verbatim}[commandchars=\\\{\}]
\PYG{n+nv}{\PYGZdl{} }date
Fri May \PYG{l+m}{17} 14:13:29 IST 2013
\end{Verbatim}


\section{Windows}
\label{system/windows:windows}\label{system/windows::doc}

\subsection{Connect to Internet via Ethernet cable (from PC/laptop)}
\label{system/windows:connect-to-internet-via-ethernet-cable-from-pc-laptop}
\textbf{Control Panel} --\textgreater{} \textbf{Network and Internet} --\textgreater{} \textbf{Network Connections}

\textbf{Ctrl} + select local and wireless connections, right click \textbf{Bridge Connections}


\chapter{Mathmatics}
\label{index:mathmatics}

\section{Algebra}
\label{math/algebra::doc}\label{math/algebra:algebra}

\section{Geometry}
\label{math/geometry:geometry}\label{math/geometry::doc}


\renewcommand{\indexname}{Index}
\printindex
\end{document}
